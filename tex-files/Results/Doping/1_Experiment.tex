%! Monte carlo
As described in section \ref{Sec:Methods_pld}, \eqref{Equ:Methods_composition} can be applied to calculate the material composition $\chi$ of the plasma plume when utilizing a \gls{DCS} target.
But as mentioned before, this does not account for the finite area illuminated by the laser pulse.
% To estimate the dopant concentration in dependence on the radius $r_\mathrm{PLD}$, the composition of the ablated material can be calculated via \eqref{Equ:Methods_composition}, but this does not account for the finite area illuminated by the laser pulse.
A simple model for including this effect can be achieved by assuming a target that has an outer composition of pure \cro\ and an inner composition of \cro\ with dopant concentration $x_{D,0}$.
Simulating $N$ randomly distributed points $r_i$ in the range $\Delta r$ around a radial laser position $r_\mathrm{PLD}$ allows the calculation of doping concentration in the plasma $x_D$ as the mean of the $N$ calculated compositions for each $r_i$.
The resulting dependence of $x_D$ on $r_\mathrm{PLD}$ is visualized in Fig.\,\ref{Fig:Results_2_yToComposition}a, where a higher value of $\Delta r$ results in more smeared out graphs.
% A target with inner concentration of \qty{0.01}{\wtpercent} dopant was assumed (Cu-doped target or Zn-doped (low) target), \textbf{but the graph can be scaled to fit the Zn-doped (high) target.}
Those \emph{Monto Carlo} simulations can further be approximated by a linear fit, which was done for $\Delta r=\qty{2}{\mm}$ (blue dotted line in Fig.\,\ref{Fig:Results_2_yToComposition}a).
Henceforth, the different samples fabricated with different radial laser spot positions $r_\mathrm{PLD}$ are parameterized by the expected composition $x_D$ calculated from this linear fit.
% The reason for this is that according to \eqref{Equ:Methods_composition}, when applying $r_\mathrm{PLD}$ smaller than the length of the semi-minor axis of the inner ellipse, no variation in composition would be observed, even though the real finite laser spot size results in a different result.
Note that due to the small concentration of dopant, it was not possible to resolve those fractions via element sensitive measurements: Dr. Daniel Splith kindly performed \gls{edx} measurements that resulted in no signal for either \ce{Cu} or \ce{Zn}.

%! Experiment
To fabricate doped thin films, three different PLD targets were utilized.
Each target was elliptically segmented, with the outer region consisting of pure \ce{Cr2O3} and the inner region consisting of \cro\ with dopant concentration $c_{D,0}$:
\begin{enumerate}
    \item $c_{D,0}=\qty{0.01}{\wtpercent}$ \ce{CuO}, called \emph{CuO-doped},
    \item $c_{D,0}=\qty{0.01}{\wtpercent}$ \ce{ZnO}, called \emph{ZnO-doped (low)} and
    \item $c_{D,0}=\qty{1}{\wtpercent}$ \ce{ZnO}, called \emph{ZnO-doped (high)}.
\end{enumerate}
For each target the \gls{DCS} approach was utilized, i.e.\ several processes were done with fixed laser spot position during deposition, but varying laser spot position \emph{between} processes.
For each process, deposition was done on all of the 4 aforementioned substrate orientations, even though \textit{m}- and \textit{a}-plane samples did not exhibit any substantial conductivity.
This was done to check whether the conductivity of the prismatic orientations could be improved via doping.
The pulse number was \qty{40000}{} for the \qty{0.01}{\wtpercent} targets and \qty{30000}{} for the ZnO-doped (high) target.

\begin{figure}
    \centering
    \begin{tabular}{ll}
        \textbf{(a)} & \textbf{(b)} \figSpace \\
        \includegraphics{g_yToComposition.eps}
        &\includegraphics{2_doped1_growthrate_labelled.pdf}
    \end{tabular}
    
    \caption{
        \textbf{(a)} Predictions for the plasma plume composition $x_D$ using Monte Carlo simulations with $N=\qty{10 000}{}$ and different values for $\Delta r$.
        The blue dotted line is a fit for the graph calculated with $r=\qty{2}{\mm}$.
        \textbf{(b)} Growth rate depending on the process order for the samples fabricated from the ZnO-doped (high) target.
        The laser entrance window was cleaned after each process.
        The applied $r_\mathrm{PLD}$ is indicated next to the marker symbols.
        }
    \label{Fig:Results_2_yToComposition}
\end{figure}