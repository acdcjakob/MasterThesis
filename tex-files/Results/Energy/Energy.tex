\section{Strain Analysis}
    \label{Sec:Results_Energy}
%! Motivation
The structural properties of the thin film, namely its mosaicity and lattice distortion depend crucially on the growth process.
It turned out that the absorption of energy at the laser entrance window alters the growth rate and the crystallinity much more dominantly than the growth temperature or the oxygen partial pressure (cf.\ \ref{Sec:Results_Preliminary}).
A similar effect was observed when targets were used for fabrication that exhibit a non-planar surface and tracks that were carved during previous ablations (cf.\ \ref{Sec:Results_Doping}).
Because the structural properties of the thin film also influence its electrical properties (cf.\ \ref{Sec:Results_Doping}), the following investigations focus on the origin of the observed variations in strain and \textomega-FWHM.
This is further motivated by the observation that a deliberate and controlled variation of laser fluence on the target surface yields a large reduction of \textomega-FWHM as well as a reduced shift of the peak position in the \thetaomega-pattern (Fig.\,\ref{Fig:Results_3_motivation}).
This was achieved by varying the lens position (cf.\ \ref{Fig:Methods_pld}) such that the laser spot size increases, yielding smaller fluence and larger ablation area on the target surface.
\begin{figure}
    \centering
    \includegraphics{3_fluence_motivation.eps}
    \caption{
    \thetaomega-patterns for two \textit{c}-plane samples fabricated with different laser focus on the target.
    The inset displays the diffractograms of the corresponding \textomega-scans performed on the respective reflections.
    The ZnO-doped (low) target was used without a fixed $r_\mathrm{PLD}$ but with uniform ablation on the whole target surface.
    }
    \label{Fig:Results_3_motivation}
\end{figure}

\section{Experiment}