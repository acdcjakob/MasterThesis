%! theta-omega
For the samples deposited on substrates with different orientation, \thetaomega-patterns were recorded (Fig.\,\ref{Fig:Results_1_w6788_2theta}).
For each sample, the expected substrate peaks are observed:
(00.6) and (00.12) for \textit{c}-plane;
(01.2), (02.4), (03.6) and (04.8) for \textit{r}-plane;
(30.0) for \textit{m}-plane;
(11.0) and (22.0) for \textit{a}-plane.
Several smaller peaks also correspond to those reflections but stem from other X-rays than \ce{Cu}-K\textalpha\ (cf.\ Fig.\,\ref{Fig:Results_1_pressure_2theta}).
The mentioned reflections are also observed for the \cro\ thin film, but with a shift in $2\theta$ position similar to the previously investigated \textit{m}-plane samples (Tab.\,\ref{Tab:Results_1_w6788}).
Note that for \textit{r}-plane, the higher order reflections of \cro\ cannot be observed.
It can be concluded that \cro\ grows in the \textalpha-phase on sapphire substrates of different orientation, where the thin film orientation matches the corresponding substrate.
Henceforth, \enquote{\textit{c}-plane \cro} will refer to a \cro\ thin film deposited on \textit{c}-plane oriented $5\times\qty{5}{\mm\squared}$ sapphire substrates, and so on.
\begin{figure}
    \centering
    \includegraphics{1_W6788_2theta_labeled.eps}
    \caption{\thetaomega-patterns of \cro\ thin films deposited on \textit{c}-, \textit{r}-, \textit{m}- and \textit{a}-plane sapphire.}
    \label{Fig:Results_1_w6788_2theta}
\end{figure}
% consider putting [h] to avoid crashing of table with figure
\begin{table}
    \centering
    \caption{Structural parameters, approximate resistivity at room temperature and activation energy for \cro\ thin films of different orientation.}
    \begin{tabular}{ccccc}
        \toprule
        Plane
            & $\epsilon_{zz}$ (\unit{\percent})
            & \textomega-FWHM (\unit{\arcminute}) 
            & $\rho$ (\unit{\ohm\cm})
            & $E_A$ (\unit{\milli\eV})\\
        \midrule
        \textit{c}  &   1.71    &   42.6    &   3       &   57, 34  \\
        \textit{r}  &   0.72    &   38.4    &   120     &   117     \\
        \textit{m}  &   0.55    &   42.6    &   3600    &   240     \\
        \textit{a}  &   1.41    &   32.4    &   4900    &   259     \\
        \bottomrule
    \end{tabular}
    \label{Tab:Results_1_w6788}
\end{table}
%! Omega-scans
For each sample, \textomega-scans were performed on the (00.6), (02.4), (30.0) and (11.0) reflections for \textit{c}-, \textit{r}-, \textit{m}- and \textit{a}-plane, respectively.
The resulting \textomega-FWHMs are in the range of approx.\ \qty{30}{\arcminute} to \qty{40}{\arcminute} (Tab.\,\ref{Tab:Results_1_w6788}).

%! Resistivity
Because the resistivity of all samples was too high to measure Hall effect, only resistivity measurements (cf.~\ref{Sec:Methods_vanDerPauw}) were performed for several temperatures (Fig.\,\ref{Fig:Results_1_w6788_TdH}).
The resistivity depends strongly on the orientation of the thin film, the resistivities at room temperature are listed in Tab.\,\ref{Tab:Results_1_w6788}.
A difference of more than three orders of magnitude between \textit{c}-plane and \textit{a}-plane samples is observed.
The linear behavior of the \textsc{Arrhenius}-plot\footnote{
    Visualization of $f(T)$ as $f'(\tau)$ with $f'=\log f$ and $\tau=1/T$.
}
indicates a thermally activated mechanism for conductivity, and thus semiconductive behavior.
Note that no further conclusions can be drawn on the conduction mechanisms due to the missing carrier concentration and mobility data.
By assuming a behavior of the form
\begin{equation}
    \rho\propto\exp\left(\frac{E_A}{k_BT}\right)\,,
\end{equation}
with \textsc{Boltzmann} constant $k_B$, an activation energy $E_A$ can be estimated.
Those energies are also listed in Tab.\,\ref{Tab:Results_1_w6788}.
For \textit{c}-plane \cro, two linear regimes can be distinguished, favoring a dependence of the form
\begin{equation}
    \rho\propto a\exp\left(\frac{E_{A,1}}{k_BT}\right)
    +b\exp\left(\frac{E_{A,2}}{k_BT}\right)\,,
\end{equation}
thus two activation energies are determined.
\begin{figure}
    \centering
    \includegraphics{1_W6788_resistivityTempDep.eps}
    \caption{
        Temperature dependent resistivity measurements for samples with different orientations.
    }
    \label{Fig:Results_1_w6788_TdH}
\end{figure}