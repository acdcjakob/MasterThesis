In Fig.\,\ref{Fig:Results_3_lensStrain}, the \gls{oop}\ strain for the \textit{r}-plane samples fabricated with varying laser spot size is shown.
The overall strain is with less than \qty{1}{\percent} lower when compared to the \textit{c}-plane samples, exhibiting values up to \qty{4}{\percent} for thin samples.
In particular, the predicted value for \gls{oop}\ strain during pseudomorphic growth of \cro\ on \ce{Al2O3} of \qty{2.41}{\percent} is not reached (cf.\ Tab.\,\ref{tab:d_strained}).
As can be seen in a detailed view (Fig.\,\ref{Fig:App_3_lensStrain_zoomed}), the strain depends on the thickness:
it decreases from \qty{0.8}{\percent} to \qty{0.5}{\percent} for an increment of thickness from \qty{50}{\nm} to \qty{200}{\nm}.
This is in accordance to the bahavior observed for the \textit{c}-plane samples, albeit less pronounced.
Furthermore, for a fixed thickness, decreasing the fluence also results in less strained thin films, which is similar to the behavior of the \textit{c}-plane samples.
The \textomega-FWHM obtained from the (02.4) reflection is depicted in Fig.\,\ref{Fig:Results_3_lensOmega}.
Similar to the \textit{c}-plane samples -- but less pronounced--, increasing the thickness results in less mosaicity, which is also achieved by reducing the fluence.
Note that the overall \textomega-FWHM is between \qtylist{50;90}{\arcminute} which differs for the \textit{c}-plane samples, where a lower fluence yielded samples with $\Delta\omega<\qty{10}{\arcminute}$ (cf. Fig.\,\ref{Fig:Results_3_lensOmega}).
Therefore, increasing the thickness and reducing the fluence by varying laser spot position may increase the crystal quality, but not to an amount comparable to \textit{c}-plane oriented thin films.
