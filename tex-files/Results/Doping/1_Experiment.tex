%! Doped targets
To fabricate doped thin films, three different PLD targets were applied.
Each target was eliptically segmented (cf.~\ref{Sec:Methods_pld}), with the outer region consisting of pure \ce{Cr2O3}.
The inner region (ellipse) was also \cro\ but with a fraction of dopant, namely:
\begin{enumerate}
    \item \qty{0.01}{\wtpercent} \ce{CuO}, called \enquote{CuO-doped},
    \item \qty{0.01}{\wtpercent} \ce{ZnO}, called \enquote{ZnO-doped (low)} and
    \item \qty{1}{\wtpercent} \ce{ZnO}, called \enquote{ZnO-doped (high)}.
\end{enumerate}
For each target, several processes were done with the laser spot position varying on the target.
For each process, deposition was done on all of the 4 aforementioned substrate orientations to check whether the conductivity of the prismatic orientations could be improved.
The composition of the ablated material can be calculated via Equ.\,\ref{Equ:Methods_composition}, but this does not account for the finite area illuminated by the laser pulse.
A simple model for including this effect can be achieved by simulating $N$ several randomly distributed points $r_i$ in the range $\Delta r$ around a radial laser position $r_\mathrm{PLD}$.
Then, the composition $x_D$ can be calculated as the mean of the $N$ calculated compositions for each $r_i$.
The resulting dependence of $x_D$ on $r_\mathrm{PLD}$ is visualized in Fig.\,\ref{Fig:Results_2_yToComposition}a, where a higher value of $\Delta r$ results in more smeared out graphs.
A target with inner concentration of \qty{0.01}{\wtpercent} dopant was assumed (Cu-doped target or Zn-doped (low) target).
The Monto Carlo simulations can further be approximated by a linear fit, which was done for $\Delta r=\qty{2}{\mm}$ (blue dotted line in Fig.\,\ref{Fig:Results_2_yToComposition}a).
Henceforth, the different samples fabricated with different radial laser spot positions $r_\mathrm{PLD}$ are characterized by the expected composition $x_D$ calculated from this linear fit.
The reason for this is that according to Equ.\,\ref{Equ:Methods_composition}, when applying $r_\mathrm{PLD}$ smaller than the length of the semi-minor axis of the inner ellipse, no variation in composition would be observed, even though the real finite laser spot size results in a different result.
Note further that due to the small concentration of dopant, it is not possible to resolve those fractions via element sensitive measurements.
Dr. Daniel Splith kindly performed \gls{edx} measurements that resulted in no signal for either \ce{Cu} or \ce{Zn} that was above the noise level.
\begin{figure}
    \centering
    \begin{tabular}{ll}
        \textbf{(a)} & \textbf{(b)} \figSpace \\
        \includegraphics{g_yToComposition.eps}
        &\includegraphics{2_doped1_growthrate.eps}
    \end{tabular}
    
    \caption{
        \textbf{(a)} Predictions for the plasma plume composition $x_D$ using Monte Carlo simulations with $N=\qty{10 000}{}$ and different values for $\Delta r$.
        The blue dotted line is a fit for the Graph calculated with $r=\qty{2}{\mm}$.
        \textbf{(b)} Growth rate depending on the process order for the samples fabricated from the ZnO-doped (high) target.
        The laser entrance window was cleaned after each process.
        }
    \label{Fig:Results_2_yToComposition}
\end{figure}

%! contacts
To improve the contacts for resistivity measurements, samples were produced using the ZnO-doped (low) target and a fixed $r_\mathrm{PLD}=\qty{3}{\mm}$.
Only \textit{c}- and \textit{r}-plane sapphire substrates were used and the deposition temperature was varied between \qty{560}{\degreeCelsius} and \qty{680}{\degreeCelsius}.
For each growth temperature, subsequent contacting was done with Ti-Al-Au for a \textit{c}-plane and an \textit{r}-plane sample, as well as Ti-Au for a \textit{c}-plane and an \textit{r}-plane sample.
Furthermore, the \textit{c}-plane samples contacted with Ti-Al-Au were compared before and after annealing at \qty{210}{\degreeCelsius} in nitrogen atomosphere.

%! measurements
\thetaomega-scans were performed for every sample, but \textomega-scans for \textit{c}- and \textit{r}-plane samples only.
The thickness was determined using spectroscopic ellipsometry.
Resistivity measurements at room temperature were done using the \textsc{Pauw} method, which was also applied when conducting temperature dependent resistivity measurements on one \textit{c}-plane sample of each target.
Note that the effect of infrequent cleaning of the laser entrance window that was described in \ref{Sec:Results_Preliminary} was discovered during the execution of those experiments, which is why the samples produced from the ZnO-doped (high) target were the only ones for which this effect could be prevented.
Furthermore, \qty{40000}{pulses} were applied for the samples fabricated from the CuO-doped and ZnO-doped (low) target, as well as one sample from the batch made with the ZnO-doped (high) target.
All other samples from this batch were deposited with \qty{30000}{pulses}.