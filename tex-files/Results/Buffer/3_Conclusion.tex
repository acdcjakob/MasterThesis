Phase-pure depositon of \agao\ on \cro\ thin films was achieved for \textit{c}-, \textit{r}-, \textit{m}- and \textit{a}-plane oriented \alo\ substrates.
The orientation of both \cro\ and \agao\ thin film was the same as the respective substrate.
For \textit{ex situ} deposition of \gao, XRD measurements indicate the presence of the \textkappa-phase of \gao, which is less dominant than the \textalpha-phase.
No optimization was performed for the deposition process, but the results serve as a proof of concept that the deposition of phase-pure \agao\ -- especially in the \textit{c}- and \textit{r}-orientation -- is possible on \cro\ buffer layers via PLD.

Especially the deposition of \agao\ in \textit{c}-orientation has been a challenge in previous studies.
Methods like mist-\gls{cvd} or \gls{hvpe} were able to deposit \agao\ on \textit{c}-plane sapphire up to a critical thickness
    \cite{jinno2020}.
However, as pointed out by \textcite{schewski2015}, those thin films exhibited both rotational domains and a contribution of the \textbeta-\gao\ phase
    \cite{shinohara2008,oshima2007}.
Phase-pure deposition without rotational domains on \alo\ can only be achieved pseudomorphically up to a thickness of only three monolayers
    \cite{schewski2015}.
The usage of buffer layers can circumwent this limitation, e.g.\ via growth on \ce{(Al,Ga)2O3} by mist-\gls{cvd}
    \cite{akaiwa2013}
or on \cro\ via \gls{hvpe}
    \cite{polyakov2022,polyakov2022a}.
However, until now, no \gls{pvd} method was able to deposit phase pure \agao\ in \textit{c}-orientation for a thickness larger than a few monolayers.
By exploiting the isostructural properties of \cro, the phase pure deposition of \agao\ in \textit{c}-orientation could be achieved via \acrlong{pld} in this work -- with an \agao\ thickness of \qty{180}{\nm}.
Furthermore, the same approach works for the growth of \agao\ in \textit{r}-, \textit{m}- and \textit{a}-orientation, which allows the deposition of \agao\ on all four common sapphire orientations in one single PLD process.