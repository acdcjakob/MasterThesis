spectroscopic ellipsometry is a method to measure the optical constants and thickness $t$ of a thin film by the change of polarization state upon light reflection.
If the probing light covers several wavelengths at once, one refers to spectroscopic ellipsometry, which will be presented in the following, based on \textcite{fujiwara2007}.
The incoming light can be represented by an electromagnetic wave, decomposed into two components being parallel (p) and perpendicular (s) to the scattering plane:
\begin{equation}
    \mathcal{E} = \mathcal{E}_\mathrm{ip} + \mathcal{E}_\mathrm{is}\,.
\end{equation}
In general, the amplitudes of p- and s-polarized light change in a different manner after reflection, so the overall polarization of the reflected light is changed.
This change is described by the fraction
\begin{equation}
    \rho=\frac{r_\mathrm{p}}{r_\mathrm{s}}:=\tan\Psi\cdot\exp(i\Delta)\,,
\end{equation}
where $r_\mathrm{p}$ and $r_\mathrm{s}$ are the amplitude reflection coefficients\footnote{
    They are defined by the amplitude of incoming (i) and reflected (r) radiation: $r=|\mathcal{E}_\mathrm{r}|/|\mathcal{E}_\mathrm{i}|$.
} for the p- and s-polarized component, respectively.
For simple structures, $\Psi$ is essentially the refractive index $n$, and $\Delta$ represents the extinction coefficient $k$.
In general, they can be calculated from the \textsc{Jones}-matrix -- representing the reflection -- and depend on the angle of incidence as well as the photon energy.

To determine the sample thickness, the spectra of $\Psi$ and $\Delta$ can be generated using a model for the sample structure, which is then fitted to the experimental data.
In this work, this model consists of an \ce{Al2O3} substrate without backscattering from the backside (infinite thickness); a \cro\ thin film of thickness $t$; and a mixed layer of air and \cro, approximating the roughness of the sample.
Because the measured spectra were confined to the visible regime of the thin film (approx. $<\qty{2.8}{\eV}$), one can apply the \textsc{Cauchy} model
    \cite{fujiwara2007},
approximating the refractive index by
\begin{equation}
    n(\lambda)=A+\frac{B}{\lambda^2}+\frac{C}{\lambda^4}+\mathcal{O}\left(\lambda^{-6}\right)\quad,\quad k=0\,.
\end{equation}

The spectroscopic ellipsometry measurements were performed with an \textit{M-2000VI} (J.A.\ Woolam Co., Inc.) and the data was analyzed using the software \textit{WVASE} (J.A.\ Woolam).
The angles of incidence were chosen to be \qty{50}{\degree}, \qty{60}{\degree} and \qty{70}{\degree}, and the modelled spectral range is \qtyrange{0.75}{2.8}{\eV}.
If samples of different orientation were fabricated in the same process, one can assume similar thicknesses, and a measurement with three angles was conducted only for one sample of the batch.
The other samples were measured with only one angle, if the determined thickness did not differ significantly from the more accurately measured one.

Note that the investigated thin films are not isotropic in general, which is why the validity of the determined thickness is checked by profilometer measurements with a \textit{Dektak XT Stylus Profiler} (Bruker Corporation).
The edges of the quadratic substrates are masked by the sample holder during deposition, which is why they are not deposited with the target material.
Consequently, measuring the height profile ranging from an edge to coated regions of the sample yields an approximation for $t$ by the height of the observed step edge.

