\chapter{Introduction}
%! Einordnung
Semiconductor technology today is based on the versatility of silicon: its band gap of \qty{1.1}{\eV} is large enough to allow tailoring its conductivity from semi-insulating to conductive -- on the other hand, it is small enough to enable both \textit{p}- and \textit{n}-type doping.
Even though being present in a vast range of applications, todays technological requirements push silicon to its limits:
power switching applications and high-frequency electronics are strongly dependent on the electrical field at which breakdown occurs
    \cite{higashiwaki2018}.
This critical field strength however is correlated to the band gap of the underlying material
    \cite{slobodyan2022},
which is why \acrfull{uwbg} materials have gained much attention:
Those semiconducors enable devices with smaller dimensions and reduced losses, which can provide the necessary properties which are needed, for example, in the optimization of charging and motor drive efficiency in electric vehicles
    \cite{gupta2024}
-- which itself is an important step towards lowering emissions.

One interesting \acrshort{uwbg} semiconductor is \gao, which turned out to be a viable material for solar-blind UV detectors
    \cite{singhpratiyush2017},
power rectifiers
    \cite{yang2017},
power MOSFETs
    \cite{moser2017}
and gas sensors
    \cite{ogita2001}.
% Its thermodynamically most stable \textbeta-phase ($E_g=4.6-\qty{5.0}{\eV}$) has been extensively studied,
Gallium oxide crystallizes in different polymorphs, of which the \textalpha-phase has the highest band gap of up to \qty{5.3}{\eV}
    \cite{hassa2021a}.
This metastable \gao\ phase crystallizes in the corundum structure, making it isostructural to cost-efficient and readily available sapphire (\alo) substrates.
However, there is still a lattice mismatch of \qty{4.6}{\percent} between \agao\ and \alo, regarding the $a$ lattice constant.
This introduces a large amount of threading dislocations
    \cite{kaneko2012},
which strongly reduces the electron mobiltiy in \agao\ thin films
    \cite{takane2023}.
Additionally, due to similar growth conditions of the \textbeta-phase, phase pure deposition of \agao\ is a challenge for different growth techniques
    \cite{bosi2020}.
Several attempts have been undertaken to improve the crystal quality and reduce the dislocation density in \agao\ thin films, namely epitaxial lateral overgrowth
    \cite{kawara2020}
and growth on patterned substrates
    \cite{nikolaev2020}
or buffer layers
    \cite{jinno2016,jinno2018}.

A promising buffer layer material for \agao\ is chromium oxide (\cro), which is isomorphic to \agao\ with a very low lattice mismatch of only \qty{0.4}{\percent} regarding the $a$ lattice constant -- one order of magnitude lower compared to \alo.
The feasibility of this approach has been demonstrated with \acrfull{hvpe}, a \acrfull{cvd} method, on \textit{c}-sapphire.
    \cite{stepanov2021,polyakov2022,polyakov2022a,butenko2023}.
\cro\ is also interesting due to its defect-induced \textit{p}-type nature, which can be enhanced by doping with lower valent cations
    \cite{arca2011,arca2013,arca2017,farrell2015}.
Because it is virtually impossible to achieve \textit{p}-type \gao\
    \cite{pearton2018},
\cro\ is a candidate for both \textit{p}-type alloys with \agao, similar to \ce{(Ir,Ga)2O3}
    \cite{kaneko2021},
and for \textit{p}-\textit{n} heterostructures with very low lattice mismatch at the interface
    \cite{polyakov2022a}.

%! aim of this work
Because \acrfull{pld} is a promising \acrfull{pvd} method for fabricating \agao\ thin films and devices
    \cite{petersen2023,vogt2023,vogt2024},
the aim of this work is to show the feasibility of preparing \cro\ thin films via \acrshort{pld}.
Therefore, the growth window of thin film growth depending on oxygen partial pressure, growth temperature and substrate orientation is investigated (chapter \ref{Sec:Results_Preliminary}).
Detailled structural analysis via \glspl{RSM} is done for thin films fabricated with various laser energy densities during the deposition (chapter \ref{Sec:Results_Energy}).
Furthermore, a segmented target approach is utilized to aim for tailored conductivities of Zn- and Cu-doped \cro\ thin films (chapter \ref{Sec:Results_Doping}).
Finally, it is demonstrated that \agao\ can be deposited phase pure on \cro\ buffer layers in \textit{c}-, \textit{r}-, \textit{m}- and \textit{a}-orientation (chapter \ref{Sec:Results_Buffer}).
Before analyzing the results of those experiments, a short overview will be given on the here investigated sesquioxides, as well as the basics of \acrfull{XRD} and heteroepitaxy of rhombohedral materials.