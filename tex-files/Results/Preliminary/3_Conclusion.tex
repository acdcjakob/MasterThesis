\textit{m}-plane \cro\ thin films can be deposited over a wide range of oxygen partial pressure of more than two orders of magnitude.
It turned out that the crystal quality correlates mainly with the growth rate, which is presumably caused by a variation of the laser pulse fluence on the target.
Therefore, lower kinetic energy of the plasma species is probably the reason for improved crystallinity and less strain.
Even though the influence of those parameters was less dominant, an oxygen partial pressure of \qty{1e-3}{\milli\bar} and a growth temperature of \qty{750}{\degreeCelsius} are identified as best growth conditions.
Note that those values overlap with the conditions for deposition \agao, which makes ternary solid solutions of chromia with rhombohedral \gao\ feasible.

\textalpha-\cro\ was also deposited on \textit{c}-, \textit{r}- and \textit{a}-plane sapphire, with the thin films crystallizing in the respective orientation.
This is important for heterostructures with \agao\ and could enable growth of rhombohedral \gao\ on all common sapphire cuts via \cro\ buffer layers.
Note that all deposited thin films showed a discrepancy between observed \gls{oop}\ lattice constants and bulk \cro\ literature values (Tab.\,\ref{Tab:sesquiLatticeConstants}).
The conductivity is strongly dependent on the crystal orientation and was very low for the prismatic orientations, but with \qty{0.3}{\siemens\per\cm} three orders of magnitude higher for the basal orientation.