%! theta omega
The \gls{oop}\ strain calculated via \eqref{Equ:Results_oop_strain_def} for all samples of the 1st batch is displayed in Fig.\,\ref{Fig:Results_3_lensStrain}.
Consider the \textit{c}-plane oriented samples of the 1st batch that had a fixed lens position yielding a fluence of approx.\ \qty{2}{\joule\per\cm\squared}, but varying thickness (brown squares in Fig.\,\ref{Fig:Results_3_lensStrain}).
A clear dependence of the \gls{oop}\ strain can be observed: thinner samples yield higher strain.
The layers become relaxed for thicknesses above approx.\ \qty{170}{\nm}.
For low thicknesses, the strain approaches the predicted value for pseudomorphic growth of \cro\ on \ce{Al2O3}, which is \qty{3.90}{\percent} (cf.\ Tab.\,\ref{tab:d_strained}).
\begin{figure}
    \centering
    \includegraphics{3_lensPos_strain.eps}
    \caption{
        Out-of-plane strain calculated from \thetaomega-patterns for all samples from the 1st batch, depending on thickness and laser fluence (false color).
    }
    \label{Fig:Results_3_lensStrain}
\end{figure}
%! RSMs strain
The recorded \glspl{RSM} of the (02.10) reflection can confirm that this observation of \gls{oop}\ strain is due to pseudomorphic growth.
In Fig.\,\ref{Fig:Results_3_cRSMs}, one can observe a shift of $q_\parallel^{(02.10)}$ to higher values for lower thicknesses.
This corresponds to a decrease of the \gls{ip}\ lattice constant, which is the expected behavior for pseudomorphic growth, because the \gls{ip}\ $a$ lattice constant of \textit{c}-oriented \ce{Al2O3} is \qty{0.2}{\angstrom} smaller than for \cro\ (cf.~Tab.\,\ref{Tab:sesquiLatticeConstants}).
The tensile \gls{oop}\ strain observed via \thetaomega-scans can also be confirmed by the fact that the \gls{oop}\ component $q_\perp^{(02.10)}$ is decreasing for thinner samples.
The reduction of signal intensity is attributed to the thickness, but could also be a result of decreasing crystal quality (cf.~Fig.\,\ref{Fig:Results_3_lensOmega}).
\begin{figure}
    \centering
    \includegraphics{3_lensPos_RSM_c.png}
    \caption{
        \glspl{RSM} of the (02.10) reflection for \textit{c}-plane oriented samples with varying thickness.
        The laser spot size was \qty{8}{\mm\squared}, resulting in a fluence of approx.\ \qty{2}{\J\per\cm\squared}.
        The reflection in the upper right corner corresponds to the (02.10) reflection of the sapphire substrate.
    }
    \label{Fig:Results_3_cRSMs}
\end{figure}
\textcite{makushko2022} reported a similar behavior for \textit{c}-plane \cro\ thin films deposited via magnetron sputtering.
They also varied the thin film thickness from \qtyrange{30}{250}{\nm}.
In \gls{TEM} measurements, they could observe that dislocations started forming approx.\ \qty{5}{\nm} away from the \cro--\alo\ interface, indicating a highly strained pseudomorphic layer.
This could also be observed as a very sharp peak in reciprocal space, identified by an alignment of $q_\parallel$ with a corresponding asymmetric substrate reciprocal space vector.
Note that this peak cannot be observed for the here presented samples.
A reason for this could be that \citeauthor{makushko2022} observed this for thin films with thickness of \qty{30}{\nm} only.
Almost every thin film presented here exceeds this critical thickness.
Another reason could be that the probed plane was the (10.\underline{10}) reflection, which 

When looking into the remaining samples that were fabricated with larger laser spot sizes but similar thickness of \qty{50}{\nm} (bluish squares in Fig.\,\ref{Fig:Results_3_lensStrain}), it becomes clear that the \gls{oop} strain is also slightly reduced for lower fluences.
But note that this effect is less dominant when compared to the influence of thickness.

%! omega
In Fig.\,\ref{Fig:Results_3_lensOmega}, the \textomega-FWHM is depicted in dependence on the film thickness and laser fluence for the 1st batch.
As before, consider the samples with smallest laser spot size (largest fluence) first:
increasing the thickness is clearly correlated to a decreasing \textomega-FWHM.
Therefore, thicker samples yield \emph{both} less strained and more crystalline films.
This is an unexpected result, because as shown in section \ref{Sec:Theory_Relaxed}, relaxation is mediated by dislocations which should worsen the crystal quality.
Note that there is an outlier to this behavior for the sample with a thickness of approx.\ \qty{30}{\nm}.
When considering the \textomega-pattern of this sample (Fig.\,\ref{Fig:App_3_cOmegaOutlier}a), it becomes clear that the non-\textsc{Voigt} shape makes the determination of \gls{FWHM} difficult.
Therefore, not too much attention should be paid to this data point.
When considering the samples fabricated with lower fluences (bluish squares in Fig.\,\ref{Fig:Results_3_lensOmega}), a much more dominant influence of laser spot size on the crystallinity can be observed.
\begin{figure}
    \centering
    \includegraphics{3_lensPos_omega.eps}
    \caption{
        \textomega-FWHM for all samples from the 1st batch, depending on thickness and laser fluence (false color).
        The corresponding diffractograms are depicted in Fig.\,\ref{Fig:App_3_lens_omega}.
    }
    \label{Fig:Results_3_lensOmega}
\end{figure}
%! correlation
This can be summarized by stating that the thickness of the samples is the dominant influence on the \gls{oop}\ strain, because the thickest samplest yielded less strain than the thinner samples with lowest fluence (Fig.\,\ref{Fig:Results_3_lensStrain}).
However, for the \textomega-FWHM, the reverse is observed, namely that even the thickest samples (which exhibit better quality than thinner samples of the same lens position) have a much higher \textomega-FWHM when compared to thinner samples fabricated with less fluence.
This can be seen in Fig.\,\ref{Fig:Results_3_lensCorrelation}, where the \textomega-FWHM is visualized depending on the \gls{oop}\ strain of the corresponding sample:
A linear behavior (correlation) is observed for each set fluence; but there are two different regimes in total, with the high-fluence regime generally showing higher \textomega-FWHM.