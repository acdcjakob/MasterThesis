\subsection{Scattering at Lattices}
To elucidate the working principles behind \gls{XRD} as a measurement method, a brief description of reciprocal space and constructive interference will be provided.
Those derivations are based on \textcite{ashcroft1976}.

A periodic point-like structure with translational symmetry (\enquote{\gls{bl}}) can be described by three vectors $\mathbf{a}_i$ that span a so-called \enquote{unit cell}.
Every lattice point $\mathbf{R}$ is a linear combination of those unit cell vectors.
For such a lattice, there exists a so-called \enquote{reciprocal lattice}, which consists of all vectors $\mathbf{K}$ satisfying the condition\footnote{
    The definition of $\textbf{K}$ by (\ref{equ:Theory_DefReciprocal}) is a consequence of demanding that the plane wave described by $f_\mathbf{K}(\mathbf{r})=\exp(i\langle\mathbf{K},\mathbf{r}\rangle)$ has the same symmetry as the \gls{bl}
        \cite{ashcroft1976}.
}:
\begin{equation}\label{equ:Theory_DefReciprocal}
    e^{i\langle\mathbf{K},\mathbf{R}\rangle}=1\,.
\end{equation}
This is again a \gls{bl} with unit cell vectors $\mathbf{a}_j^*$:
\begin{equation}
    \mathbf{K}_{hkl}=h\mathbf{a}_1^*+k\mathbf{a}_2^*+l\mathbf{a}_3^*\,.
\end{equation}
It follows that for any $i,j$:
\begin{equation}
    \langle\mathbf{a}_i^*,\mathbf{a}_j\rangle=2\pi\delta_{ij}\,,
\end{equation}
with the \textsc{Kronecker} delta $\delta_{ij}$.
A major application of reciprocal space vectors is their ability to describe lattice planes.
Any lattice plane can be described by the shortest possible reciprocal space vector $\mathbf{K}_{hkl}$ perpendicular to it.
Consequently, the lattice plane is denoted by $(hkl)$.
The distance between equivalent lattice planes can be calculated via $d_{hkl}=|\mathbf{K}_{hkl}|^{-1}$.
Note that for non-cubic crystals, the lattice plane (hkl) is in general \textit{not} perpendicular to the lattice direction [hkl].

With those preliminarities, the conditions for constructive interference during dif\-frac\-tion of radiation at \glspl{bl} can be derived.
Consider two scattering centers separated by $\mathbf{d}$.
Now consider incoming radiation with wave vector $\mathbf{k}$:
\begin{equation}
        \mathbf{k}=\frac{2\pi}{\lambda}\hat{\mathbf{n}}\,,
\end{equation}
with wavelength $\lambda$ and direction $\hat{\mathbf{n}}$.
For the case of elastic scattering, the outcoming wave vector $\mathbf{k}'$ has the same wavelength $\lambda$ but different direction $\hat{\mathbf{n}}'$.
The phase difference of two photons scattered at the 1st and 2nd scattering center, respectively, can be calculated from their path difference, which reads
\begin{equation}\label{Equ:Theory_PathDifference}
    \langle\mathbf{d},\hat{\mathbf{n}}\rangle+\langle-\mathbf{d},\hat{\mathbf{n}}'\rangle\,.
\end{equation}
\begin{figure}
    \centering
    \includegraphics[width=.5\textwidth]{example-image-a}
    \caption{\textit{Here comes an ashcroft-like image for the scattering geometry to derive Equ.~(\ref{Equ:Theory_PathDifference})}}
\end{figure}
Constructive interference occurs, if the phase difference is an integral multiple of the wavelength, so it must follow that
\begin{align}
    \langle\mathbf{d},(\hat{\mathbf{n}}-\hat{\mathbf{n}}')\rangle&=m\lambda\\
    \Leftrightarrow\langle\mathbf{d},(\hat{\mathbf{k}}-\hat{\mathbf{k}}')\rangle&=2\pi m\,,
    \label{Equ:Theory_ConstructiveInterference}
\end{align}
with $m\in\mathbb{N}$.
Comparing with (\ref{equ:Theory_DefReciprocal}) reveals that $\hat{\mathbf{k}}-\hat{\mathbf{k}}'$ is a reciprocal space vector, because the separation $\mathbf{d}$ of the two scattering centers is a lattice vector.
So constructive interference (observing a reflex) occurs if and only if the scattering geometry (determined by angle of incidence and refraction, as well as wavelength) matches the lattice symmetry in the sense that there is a corresponding lattice translation vector $\mathbf{d}$ fulfilling Equ.~(\ref{Equ:Theory_ConstructiveInterference}).
So from the \enquote{position} of reflexes, one can deduce the lattice symmetry.

Note that this description of X-ray scattering is equivalent to the \gls{bc}:
\begin{equation}\label{Equ:Theory_BraggCondition}
    m\lambda=2d_{hkl}\sin(\theta)\,,
\end{equation}
where the angle of incidence $\theta$ and $\lambda$ are contained in $\hat{\mathbf{k}}-\hat{\mathbf{k}}'$.
Furthermore, when a lattice point is not equivalent to a single atom, but represents several scattering centers, an additional geometrical structure factor has to be taken into account to determine whether a certain geometry allows reflexes.
This is important, e.g., for structures with trigonal symmetry. 
They are described with a conventional hexagonal unit cell, although not every plane $(hkl)$ exhibits constructive interference.

\subsection{X-rays}
Atomic distances in solids are of the order of several \si{\angstrom}, so the radiation for probing those structures must have a similar wavelength, which turns out to be X-rays
    \cite{harrington2021}.
The following description of X-rays is based on \textcite{spiess2009}.

The basis of any X-ray tube are high-energy electrons which are produced by thermionic emission in a cathode, which is usally made out of tungsten\footnote{
    Tungsten is the element with the second highest melting point of around \qty{3400}{\celsius}.
    This ensures a low contamination of the anode with cathode material.
}.
An electric field of several \si{\kV} accelerates the electrons to the anode, where they are stopped such that around \qty{99}{\percent} of their kinetic energy dissipates.
The momentum change of electrons, which are charged particles, leads to emission of \textit{bremsstrahlung}.
Furthermore, the electrons ionize atoms of the anode material which leads to unoccupied electron states.
If those states are filled by electrons with higher quantum number $n$, the difference in energy of those levels is emitted as radiation with a discrete spectrum, called characteristic X-ray.
Important for this work is a part of the characteristic spectrum, called K-radiation, which originates in occupation of empty $1s$-orbitals.
The occupying electron must come from an orbital with angular momentum quantum number $l=1$, i.e.\ a \textit{p}-orbital, because $\Delta l$ cannot be zero.
The radiation is termed K\textalpha- or K\textbeta-radiation, if the previous orbital was $2p$ or $3p$, respectively.
Furthermore, one distinguishes K\textalpha\textsubscript{1}- and K\textalpha\textsubscript{2}-radiation, depending on the magnetic quantum number of the previous orbital, which can be $\frac{3}{2}$ or $\frac{1}{2}$, respectively.
K\textalpha-radiation is desired for probing crystal structures.