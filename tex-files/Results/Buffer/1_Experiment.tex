For the \cro\ thin films, deposition parameters were chosen that yielded the lowest strain and \textomega-FWHM.
Therefore, a lens position of \qty{-1}{\cm} and a laser pulse energy of \qty{450}{\milli\joule} were applied.
Note that \qty{350}{\milli\joule} would result in even better crystallinity, but only at the cost of very low growth rates.
The \gao\ layer was deposited with a lens position of \qty{-1}{\cm} and a laser pulse energy of \qty{650}{\milli\joule}.
This is the configuration that was used for depositing \cro\ in \ref{Sec:Results_Preliminary} and \ref{Sec:Results_Doping}.
By assuming a growth rate of \qty{10}{\pm\per\pulse} -- which is achieved for deposition with similar parameters at a differnt PLD chamber --, \qty{15000}{pulses} were applied to achieve a layer thickness of \qty{150}{\nm}.
The oxygen partial pressure was \qty{3e-4}{\milli\bar} and the temperature was chosen to be approx.\ ???.
Two approaches were chosen to deposit \agao\ of four different orientations \textit{c}, \textit{r}, \textit{m} and \textit{a}:
\begin{enumerate}
    \item Deposition of \cro\ with subsequent analysis of the \cro\ layer under atmospheric conditions.
    Afterwards the \gao\ layer was deposited \textit{ex situ} in a distinct PLD process.
    \item \textit{In situ} deposition of \cro\ and \gao\ without returning to atmospheric conditions inbetween.
    Between both processes, the oxygen pressure was reduced and it was waited for \qty{20}{\min} to achieve thermodynamic equilibrium.
\end{enumerate}

For the \textit{ex situ} samples, \glspl{RSM} were recorded before and after the deposition of the \gao\ layer.
For all samples, \thetaomega-scans were performed in the range of \qtyrange{10}{130}{\degree} after deposition of the \gao\ layer.
Furthermore, \gls{RHEED} patterns were recorded for the surfaces of all samples after deposition of the \gao\ layer.
    \gls{RHEED} is a method to probe the crystal structure of only a few monolayers below the surface of a thin film.
    High-energy electrons are pointed in grazing incidence geometry on the sample surface and the diffracted electrons are detected with a floroescent screen.
    The image of the screen is called the RHEED pattern and it can yield information about the symmetry and crystallinity of the sample surface
        \cite{hafez2022}.

The \thetaomega\ patterns are compared to a reference \cro\ sample, namely sample \texttt{A} from \ref{Sec:Results_Energy}, which was also fabricated with a pulse energy of \qty{450}{\milli\joule}.
The theoretical predictions of the $2\theta$ values for reflections of \textbeta-\gao\ were taken from the \textit{Materials Project}
    \cite[mp-886]{MaterialsProject}.
