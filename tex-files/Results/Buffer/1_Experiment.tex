For the \cro\ thin films, deposition parameters were chosen that yield the lowest strain and smallest \textomega-FWHM.
Therefore, a lens position of \qty{-1}{\cm} and a laser pulse energy of \qty{450}{\milli\joule} were applied.
Note that \qty{350}{\milli\joule} would result in even less strain, but only at the cost of very low growth rates and less crystallinity.
The \gao\ layer was deposited with a lens position of \qty{-2}{\cm} and a laser pulse energy of \qty{650}{\milli\joule}, being known as optimal for \agao\ deposition in this chamber.
% This is the configuration that was used for depositing \cro\ in \ref{Sec:Results_Preliminary} and \ref{Sec:Results_Doping}.
By assuming a growth rate of \qty{10}{\pm\per\pulse} -- which is achieved for deposition with similar parameters at a differnt PLD chamber --, \qty{15000}{pulses} were applied to achieve a layer thickness of approx.\ \qty{150}{\nm}.
The oxygen partial pressure was \qty{3e-4}{\milli\bar} and the heater temperature was chosen to be approx.\ \qty{680}{\degreeCelsius}, which is lower than for \cro\ deposition.
Two approaches were chosen to deposit \agao\ of four different orientations \textit{c}, \textit{r}, \textit{m} and \textit{a}:
\begin{enumerate}
    \item Deposition of \cro\ with subsequent \textit{ex situ} analysis of the \cro\ layer under atmospheric conditions.
    Afterwards the \gao\ layer was deposited in an additional PLD process.
    \item \textit{In situ} deposition of \cro\ and \gao\ without returning to atmospheric conditions inbetween.
    Between both processes, the oxygen pressure was reduced and it was waited for \qty{20}{\min} to achieve thermodynamic equilibrium.
\end{enumerate}

For the \textit{ex situ} samples, \glspl{RSM} around the exptected \textalpha-phase symmetric reflection were recorded before and after the deposition of the \gao\ layer to confirm the formation of the \textalpha-phase.
Furthermore, \thetaomega-scans were performed for all samples after deposition of the \gao\ layer to check whether other phases are present.
To confirm the crystallinity of the \gao\ layer, \gls{RHEED} patterns were recorded for the surfaces of all samples after deposition of the \gao\ layer.
    

The \thetaomega\ patterns are compared to a reference \cro\ sample from chapter \ref{Sec:Results_Energy}, fabricated with a pulse energy of \qty{450}{\milli\joule}.
The theoretical predictions of the $2\theta$ values for reflections of \textbeta-\gao\ were taken from the \textit{Materials Project}
    \cite[mp-886]{MaterialsProject}.

\gls{TEM} and \gls{edx} measurements were performed by Dr.\ J.\ G.\  Fernandez from \textit{Centre for Materials Science and Nanotechnology Physics}, Oslo, who kindly provided the images, shown in Figs.\,\ref{Fig:Results_4_TEM_c}, \ref{Fig:Results_4_TEM_m_100nm} and \ref{Fig:Results_4_TEM_m_zoom}.