\chapter{Summary and Outlook}
\Acrlong{uwbg} materials are increasingly important for applications in high power electronics
    \cite{yang2017,moser2017,gupta2024},
but even though being attractive due to the high bandgap of up to \qty{5.3}{\eV}
    \cite{hassa2021a},
the deposition of corundum structured \agao\ on sapphire is challeng\-ing due to lattice mismatch induced defects
    \cite{kaneko2012}
and parastitic phases
    \cite{bosi2020}.
In this thesis, preliminary investigations were performed on the feasibility of using chromium oxide (\cro) as a buffer layer to enhance the growth of \agao\ via \acrfull{pld}.
Phase pure deposition of \cro\ on \textit{m}-plane sapphire was achieved for a large growth window.
% , spanning over oxygen partial pressures of two orders of magnitude.
With optimized growth conditions, \textit{c}-, \textit{r}-, \textit{m}- and \textit{a}-oriented \cro\ thin films were grown on sapphire substrates of the respective orientation.
\acrfull{XRD} measurements revealed that \cro\ thin films were strained, independent of crystal orientation.
For \textit{c}-oriented samples of low thickness, this strain was found to be due to pseudomorphic growth.
For \textit{r}-, \textit{m}- and \textit{a}-orientations, a dependence of lattice distortion on the plasma dynamics was observed: ablation on worn targets or reduced laser fluences drastically improved the crystal quality regarding strain and mosaicity.
The observed reduction of strain was attributed to relaxed growth, as it was previously observed for \ce{(Al,Ga)2O3} thin films
    \cite{kneiss2021,grundmann2020b}.
However, the slip systems responsible for partially relaxed behavior, indicated by the simultaneous presence of thin film tilt and strain, could not be resolved within the scope of this work.
Advanced measurement techniques like \acrshort{TEM} should be applied to gain more information on how to optimize the growth conditions regarding anisotropic relaxation.

The conductivity of the \cro\ thin films depends strongly on the crystal orientation, where \textit{c}-oriented films exhibited conductivities of up to \qty{0.3}{\siemens\per\cm} at room temperature.
For \textit{a}-oriented films, however, the conductivity was three orders of magnitude lower.
An increase of the conductivity by tailored doping of the \cro\ thin films by doping with \ce{Zn} and \ce{Cu} via a segmented target approach was not successful.
The conductivity of doped thin films did not differ systematically from that of undoped thin films.
% The attempt on improving the conductivity of \cro\ by doping with \ce{Zn} and \ce{Cu} via the segmented target approach did not work out.
However, a strong dependence of conductivity on crystallinity could be observed: increasing \textomega-FWHMs by only a factor of 5 resulted in an improvement of conductivity by two orders of magnitude.
This favors the prediciton
    \cite{mi2018}
of \cro\ being a defect induced \textit{p}-type conductor.
Temperature dependent resistivity measurements also indicate semiconducting behavior.
Future work should focus on more potent dopant materials such as \ce{Mg}, which was reported to increase the conductivity of \cro\
    \cite{uekawa1996}.

The final goal of this work, namely stabilizing phase pure \agao\ in \textit{m}- and \textit{a}-orientation on \cro, was surpassed:
for the first time, the \textalpha-phase of \gao\ could be deposited without parasitic \textbeta- or \textkappa-phase in \textit{c}-orientation with a \acrfull{pvd} method, opening new pathways towards next-generation electronic devices.
Within a single process, the \cro and \agao\ layers were deposited on \textit{c}-, \textit{r}-, \textit{m}- and \textit{a}-sapphire.
A crystallization of the top \agao\ layer was confirmed by by \acrshort{RHEED} measurements.
The phase pure growth was verified by \acrshort{XRD} measurements.
High resolution \acrshort{TEM} images confirmed the growth of \agao\ on the \cro\ and a high-quality interface is observed.
% One single process can be conducted for depositing the \cro\ buffer layer and functional \gao\ layer in all four orientations simultaneously, as it was shown using various characterization methods like \acrfull{RHEED}, Reciprical Space Maps (RSM) and high-reslution \acrshort{TEM}.
% A high-quality interface between \agao\ and \cro\ was observed.
Therefore, the feasability of \cro\ buffers grown by \acrshort{pld} and the subsequent use as buffer for \gao\ thin films was demonstrated within this thesis. 
In future works, the thickness of the buffer layer should be further optimized and conductive \gao\ layers should be deposited on the \cro\ buffers.
% Future studies should focus on further increasing the quality of \agao\ thin films by usage of \cro\ buffer layers, as the here shown samples only serve as a proof of concept.
Furthermore, the feasibility of \textit{p}-\textit{n} heterojunctions of \ce{Mg} doped \cro\ and \textit{c}-oriented \agao\ should be investigated due to the higher conductivity observed for \textit{c}-oriented \cro.
To summarize, this work contributed substantially to the ongoing research on \acrshort{uwbg} materials, in particular on improving the deposition of promising \agao\ using \acrlong{pld}.