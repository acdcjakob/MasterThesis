\gls{RHEED} is a method to probe the crystal structure of only a few monolayers below the surface of a thin film.
High-energy electrons are pointed in grazing incidence geometry on the sample surface and the diffracted electrons are detected with a fluoroescent screen.
The image of the screen is called the RHEED pattern and it can yield information about the symmetry and crystallinity of the sample surface
    \cite{hafez2022}.

\gls{TEM} is a method for providing highly magnified images of solid state samples.
The method utilizes a beam of high energy electrons that is passing through an sufficiently thin sample.
Resolutions of \qty{0.08}{\nm} can be achieved due to the shorter wavelength of electrons compared to visible light
    \cite{schroder2005}.
A scanning \acrshort{TEM} method is \gls{HAADF}, where scattered electrons are detected with an annular dark field detector.
The \acrshort{HAADF} images in this 