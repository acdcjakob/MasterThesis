For both \textit{r}- and \textit{c}-plane, a reduction in growth temperature results in higher crystallinity (not shown).
Lower growth temperatures also yield higher resistivites, as shown in Fig.\,\ref{Fig:Results_2_contacts}.
This confirms the previously observed result, that the crystallinity reduces conductivity.
It has to be noted that several effects influence the change in \textomega-FWHM:
The first process was done at highest temperature, which was then gradually reduced.
So the temperature reduction is convoluted with a process order effect.
This is important, because for every sample, the same radial laser spot position $r_\mathrm{PLD}=\qty{3}{\mm}$ was used on the ZnO-doped (low) target, which led to increasing surface degradation (cf. Fig.\,\ref{Fig:Results_2_photoTarget}b).
This results, as shown before, in lower growth rates and higher crystallinities.
This effect is supported by the fact that the window was cleaned before the first process only, which also adds to the subsequent reduction of laser fluence on the target surface.
Therefore, the effect of temperature variation is covered by those growth rate effects, which makes it difficult to deconvolute the influence of growth temperature on conductivity.
As can also be seen in Fig.\,\ref{Fig:Results_2_contacts}, no significant change in resistivity was observed for both the variation between Ti-Al-Au and Ti-Au contacts, as well as between annealed and as-deposited contacts.
For \textit{r}-plane, the Ti-Al-Au contacts yielded slightly better conductivities.
\begin{figure}
    \centering
    \includegraphics{2_contacts.eps}
    \caption{Resistivity measured at room temperature depending on growth temperature.
    Three different contact types were applied.}
    \label{Fig:Results_2_contacts}
\end{figure}
