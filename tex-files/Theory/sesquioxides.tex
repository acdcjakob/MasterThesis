\glspl{tco} are materials that combine the properties of having low absorption coefficient in the visible spectrum and being conductive at the same time
    \cite{kehoe2016}.
The interest in these materials is motivated by possible usage in portable and flexible electronics, displays, solar cells and more
    \cite{ginley2011}.
Due to the restriction on only a few materials in the industry (e.g. \ce{SnO2} and \ce{In2O3}), investigations of new materials are required
    \cite{ginley2011}.
This includes fabrication of \textit{p}-type \glspl{tco} as well as compounds with even larger band gaps than \qty{3}{eV}, called \gls{uwbg} materials.
A candidate for the latter is \ce{Ga2O3} with its several polymorphs
    \cite{hassa2021a},
where the corundum structured \agao{} gained interest, even though its deposition has to account for parasitic growth of the thermodynamically more stable \textbeta-phase
    \cite{petersen2023}.

At this point, \ce{Cr2O3} comes in handy being a possible \textit{p}-type \gls{tco} as well as being isomorphic to group-III sesquioxide \agao{} with quite similar lattice parameters (cf. Tab.~\ref{Tab:sesquiLatticeConstants}).
This enables the use of \ce{Cr2O3} as a buffer layer between \agao{} and isomorphic \textalpha-\ce{Al2O3} (sapphire) substrates to improve the deposition process
    \cite{stepanov2021}.
Furthermore, \ce{Cr2O3} exhibits increased conductivity upon doping
    \cite{uekawa1996}
and could thus serve as \textit{p}-type component in a \textit{p}-\textit{n}-heterojunction with \agao{}.
In the following, an overview of the two mentioned sesquioxides will be provided with focus on the physical properties being relevant to this work.