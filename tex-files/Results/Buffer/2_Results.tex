%! RHEED
The RHEED patterns of all samples from the \textit{ex situ} and the \textit{in situ} batch are depicted in Fig.\,\ref{Fig:Results_4_RHEED}a and Fig.\,\ref{Fig:Results_4_RHEED}b, respectively.
From the periodic patterns it can be concluded that every surface is crystalline.
Note that the mere observation of crystallinity does not indicate which phase of \gao\ is present on the samples.
\begin{figure}
    \centering
    \begin{tabular}{llll}
        \multicolumn{4}{l}{\textbf{(a)}} \figSpace \\
        \textit{c}-plane & \textit{r}-plane & \textit{m}-plane & \textit{a}-plane \\        
        \includegraphics[width=.22\textwidth]{w6957c.jpg}
        & \includegraphics[width=.22\textwidth]{w6957r.jpg}
        & \includegraphics[width=.22\textwidth]{w6957m.jpg}
        & \includegraphics[width=.22\textwidth]{w6957a.jpg} \figSpace \\
        \multicolumn{4}{l}{\textbf{(b)}} \figSpace \\
        \textit{c}-plane & \textit{r}-plane & \textit{m}-plane & \textit{a}-plane\\
        \includegraphics[width=.22\textwidth]{w6959c.jpg}
        & \includegraphics[width=.22\textwidth]{w6959r.jpg}
        & \includegraphics[width=.22\textwidth]{w6959m.jpg}
        & \includegraphics[width=.22\textwidth]{w6959a.jpg}
    \end{tabular}
    \caption{\gls{RHEED} patterns of the \agao\ surface deposited \textbf{(a)} \textit{ex situ} on \cro\ and \textbf{(b)} \textit{in situ} on \cro.
    Note that the patterns were recorded in arbitrary azimuth which is why they may differ between \textit{ex situ} and \textit{in situ} deposition.
    This is not necessarily a result of different crystal structure.
    }
    \label{Fig:Results_4_RHEED}
\end{figure}

%! c-plane
\subsubsection*{\textit{c}-plane \texorpdfstring{\agao}{a-Ga2O3} grown on \texorpdfstring{\cro}{Cr2O3}}
\begin{figure}
    \centering
    \begin{tabular}{c}
        \multicolumn{1}{l}{\textbf{(a)}}
        \figSpace \\
        \includegraphics{4_thetaomega_c_labelled.pdf}
        \figSpace \\
        \multicolumn{1}{l}{\textbf{(b)}}
        \figSpace \\
        \includegraphics{4_RSM_c.png}
    \end{tabular}
    \caption{
        \textbf{(a)}
        \thetaomega-patterns of the \textit{c}-plane \cro\ reference sample (black), as well as the \textit{ex situ} (blue) and \textit{in situ} (red) buffer layer structures.
        The expected peak positions of \textbeta-\gao\ (purple dotted) and \textkappa-\gao\ (green dashed) from Tab.\,\ref{Tab:Results_4_beta_c} are indicated, as well as the (00.6) reflection of \agao\ (red dotted).
        \textbf{(b)}~\gls{RSM} around the (00.6) reflection recorded before (left) and after (right) \textit{ex situ} deposition of \gao\ on \cro.
    }
    \label{Fig:Results_4_buffer_c}
\end{figure}
To determine the present crystal phases of the \gao\ thin film, in Fig.\,\ref{Fig:Results_4_buffer_c}a, \thetaomega\ patterns of the samples deposited on \textit{c}-plane \alo\ are depicted, namely the \cro\ reference sample (black), as well as the \textit{ex situ} (blue) and \textit{in situ} (red) samples.
The peak at around \qty{39.2}{\degree} of the reference sample is the (00.6) reflection of \cro.
This reflection is also attributed to the peaks at approx.\ \qty{39.4}{\degree} and \qty{39.6}{\degree} of the \textit{ex situ} and \textit{in situ} samples, respectively.
The variation of peak position for the \cro\ layer may originate in the fact that for each process, the thickness may have varied.
As discussed in chapter \ref{Sec:Results_Energy}, this is a crucial factor for the out-of-plane strain and therefore the position of the reflection.
In addition, two overlaying peaks occur at \qty{40.14}{\degree} for both buffer layer processes.
This is attributed to the (00.6) reflection of \agao, because the predicted peak position is at \qty{40.26}{\degree} with a relative intensity
% \footnote{
%     This value indicates the peak intensity of a reflection in powder diffraction patterns and gives a hint to identify peaks.
% }
of 3.37, and is therefore allowed.
The presence of two peaks is attributed to the splitting between \ce{Cu}-K\textalpha\textsubscript{1} and \ce{Cu}-K\textalpha\textsubscript{2} radiation (\textit{K\textalpha\ splitting}).
Note that at approx.\ \qty{86}{\degree}, the higher order (00.12) reflection of \cro\ can be observed more dominantly than the (00.12) reflection of \agao.
This is due to the fact the the ratio of relative intensities of (00.12) to (00.6) is \qty{29.6}{\percent} for \cro\ and only \qty{1.4}{\percent} for \agao. 

For the \textit{in situ} sample, no additional peaks are observed, indicating phase-pure deposition of \textit{c}-plane \gao\ in the \textalpha-phase on \textit{c}-plane \alo.
For the \textit{ex situ} sample, however, a peak occurs at \qty{38.78}{\degree} and both the \textbeta- and \textkappa-phase of \gao\ have a predicted peak at this position, listed in Tab.\,\ref{Tab:Results_4_beta_c}.
The observed peak is attributed to the (004) reflection of the \textkappa-phase.
A lower order of this reflection is also observed at \qty{19.13}{\degree}, which is attributed to the (002) reflection of \textkappa-\gao.
Note that the peak observed at \qty{20.4}{\degree} is also observed for the \cro\ reference sample and can therefore not correspond to any \gao\ phase.
It can be concluded that the \textit{ex situ} growth of \gao\ on a \textit{c}-plane \cro\ buffer layer results in both \agao\ and \textkappa-\gao, with the latter being much less pronounced.
\begin{table}
    \centering
    \caption{
        Selected reflections of \textbeta-\gao\ \cite[mp-886]{MaterialsProject} and \textkappa-\gao\ \cite{hassa2021} and their predicted positions in \thetaomega\ patterns, as well as the observed peak position for \gao\ deposited \textit{ex situ} on \cro\ (blue curve in Fig.\,\ref{Fig:Results_4_buffer_c}a).
    }
    \begin{tabular}{lccc}
        \toprule
        Phase & {$2\theta$} & reflection & measured\\
        \midrule 
        \textbeta-\gao          & \qty{18.89}{\degree}    & $(20\overline{1})$  & \multirow{2}{*}{\qty{19.13}{\degree}}\\
        \textkappa-\gao         & \qty{19.11}{\degree}      & (002) & \\
        \midrule
        \textbeta-\gao          & \qty{38.32}{\degree}     & $(40\overline{2})$ & \multirow{2}{*}{\qty{38.78}{\degree}}\\
        \textkappa-\gao         & \qty{38.77}{\degree}      & (004) \\
        \bottomrule
    \end{tabular}
    \label{Tab:Results_4_beta_c}
\end{table}

In Fig.\,\ref{Fig:Results_4_buffer_c}b, the \glspl{RSM} of the \textit{ex situ} \textit{c}-plane sample are displayed before and after deposition of \gao.
Note that the image is cropped such that no substrate peak is visible.
The previous result is confirmed that the \textalpha-phase of \gao\ has formed on top of the \cro\ layer.
No \textomega-scans were done, but the crystallinity can be estimated by the broadening in $q_\parallel$ direction, which is less dominant in comparison to the \cro\ layer.
Furthermore, the K\textalpha\ splitting indicates a highly crystalline thin film. 

\begin{figure}
    \centering
    \includegraphics{4_TEM_c.pdf}
    \caption{
        \acrshort{HAADF} image of the interface between \textit{c}-plane \cro\ and \agao.
        The $[\overline{2}1.0]$ direction is parallel to the electron beam, i.e.\ the \textit{a}-plane is visible.
        Courtesy of Dr.\ J.\ G.\  Fernandez.
        The image of the \agao\ crystal structure was made with VESTA Ver.~3 \cite{momma2011}.
    }
    \label{Fig:Results_4_TEM_c}
\end{figure}
In Fig.\,\ref{Fig:Results_4_TEM_c}, \acrshort{HAADF} images are shown for the interface between \agao\ and \cro.
The \textit{c}-axis points upwards, and the atom arrangement that is visible corresponds to the \textit{a}-plane.
The similar crystal structure can be clearly seen by comparing the lattice spacing of the upper and lower half plane.
No dislocations form at the interface and the (00.1)-orientation of the \cro\ buffer continues for the \gao\ layer.
The thickness of the \agao\ layer is \qty{180}{\nm}.

\subsubsection*{\textit{m}-, \textit{a}- and \textit{r}-plane \texorpdfstring{\agao}{a-Ga2O3}\ grown on \texorpdfstring{\cro}
{Cr2O3}}


%! m-plane
When comparing the peaks of \textit{m}-plane buffer layer samples to the \cro\ reference sample, note that the (30.0) peak is shifted to lower angles (Fig.\,\ref{Fig:Results_4_buffer_m}a), i.e.\ that the buffer layers are more strained when compared to the reference sample.
Note that for the buffer layers, a new \cro\ target has been used.
This could be the reason for the increased strain as shown in chapter \ref{Sec:Results_Doping}, where target degradation reduced the peak shift in \thetaomega\ patterns.
Furthermore, two peaks can be observed for the buffer layer samples, which may originate in (i) either two peaks for \cro\ and \agao\ each; or (ii) K\textalpha\ splitting of a (30.0) \agao\ reflection on top of the \cro\ layer.
Both explanations are favored by the fact that the expected peak position (red dotted line) lays inbetween both peaks.
The theoretical predictions of the $2\theta$ positions also have similar distance as the two peaks observed (red and green dotted lines), favoring the first explanation.
However, when considering Fig.\,\ref{Fig:Results_4_buffer_m}b, it becomes clear that the origin is a K\textalpha\ splitting, because prior to the deposition of \gao, none of the peaks was present with the observed intensity.
Therefore, the observed peaks must both stem from the \textalpha-phase of \gao.
No other peaks are observed in the \thetaomega\ pattern, therefore only the \textalpha-phase is present.

\begin{figure}[ht]
    \centering
    \begin{tabular}{cc}
    \multicolumn{1}{l}{\textbf{(a)}} & \multicolumn{1}{l}{\textbf{(b)}} \\
        \includegraphics{4_TEM_m_100nm_HAADF.pdf}
        & \includegraphics{4_TEM_m_100nm_EDX.pdf}
    \end{tabular}
    \caption{
        Images of the \textit{m}-plane buffer layer structure:
        \textbf{(a)}~\acrshort{HAADF} image and \textbf{(b)}~spatially resolved \acrshort{edx} data of the \alo, \cro\ and \agao\ layer.
        The \textit{c}-axis is parallel to the electron beam.
        Courtesy of Dr.\ J.\ G.\  Fernandez.
    }
    \label{Fig:Results_4_TEM_m_100nm}
\end{figure}
In Fig.\,\ref{Fig:Results_4_TEM_m_100nm}a, an \acrshort{HAADF} image of the \textit{m}-plane buffer layer structure is depicted.
The [10.0] direction points upwards, and the atom arrangement that is visible corresponds to the \textit{c}-plane.
In the \agao\ layer, threading dislocations can be seen.
The expected composition of the layers is confirmed by spatially resolved \acrshort{edx} measurements (cf.\ Fig.\,\ref{Fig:Results_4_TEM_m_100nm}b).
\begin{figure}
    \centering
    \begin{tabular}{cc}
        \multicolumn{1}{l}{\textbf{(a)}} & \multicolumn{1}{l}{\textbf{(b)}} \\
        \includegraphics{4_TEM_m_interface.pdf}
        & \includegraphics{4_TEM_m_onlyGa.pdf}   
    \end{tabular}
    \caption{
        Images of the \textit{m}-plane buffer layer structure:
        \textbf{(a)}~\acrshort{HAADF} image of the interface between \cro\ and \agao.
        \textbf{(b)}~\acrshort{HAADF} representative image of the \agao\ layer.
        Courtesy of Dr.\ J.\ G.\  Fernandez.
        The image of the \agao\ crystal structure was made with VESTA Ver.~3 \cite{momma2011}.
    }
    \label{Fig:Results_4_TEM_m_zoom}
\end{figure}
A detailed view into the interface between \cro\ and \agao\ is given in Fig.\,\ref{Fig:Results_4_TEM_m_zoom}a.
Similar to the \textit{c}-plane structure, the (10.0) orientation of the buffer layer is continued in the \agao\ layer, which exhibits very good crystal quality, as can be seen in Fig.\,\ref{Fig:Results_4_TEM_m_zoom}b.

%! a-plane
A similar behavior as for the \textit{m}-plane samples can be observed in the \thetaomega\ patterns of \textit{a}-plane samples (Fig.\,\ref{Fig:Results_4_buffer_a}a).
The splitted peak is attributed to the (22.0) reflection of an \agao\ layer and the peak on the left shoulder to the (22.0) reflection of \cro, which is shifted to lower angles in comparison to the reference sample.
This behavior is also observed for the (11.0) reflections of both \agao\ and \cro.
This result is confirmed by the \glspl{RSM} (Fig.\,\ref{Fig:Results_4_buffer_a}b), where two peaks appear due to K\textalpha\ splitting after \gao\ deposition.
The low broadening in $q_\parallel$ direction as well as the K\textalpha\ splitting indicate good crystal quality.


%! r-plane
The \thetaomega\ patterns of the \textit{r}-plane reference and buffer layer samples are depicted in Fig.\,\ref{Fig:Results_4_buffer_r}a.
Due to very close peak positions of the (02.4) reflection for both \cro\ and \agao, a comparison with the reference sample (black) is not as straightforward as for the other orientations.
No additional peak can be identified for the buffer layer samples.
However, the \gls{RHEED} patterns indicated a crystalline surface, which is why the only possible phase of the \gao\ layer is the \textalpha-phase.
This is verified by \glspl{RSM} of the \textit{ex situ} sample in Fig.\,\ref{Fig:Results_4_buffer_r}b:
a significant increment in intensity can be observed at the expected peak position of (02.4) \agao, which is due to the formation of \textalpha-phase \gao\ on \textit{r}-plane \cro.



\begin{figure}
    \centering
    \begin{tabular}{c}
        \multicolumn{1}{l}{\textbf{(a)}}
        \figSpace \\
        \includegraphics{4_thetaomega_m.pdf}
        \figSpace \\
        \multicolumn{1}{l}{\textbf{(b)}}
        \figSpace \\
        \includegraphics{4_RSM_m.png}
    \end{tabular}
    \caption{
        \textbf{(a)}
        \thetaomega-patterns of the \textit{m}-plane \cro\ reference sample  (black), as well as the \textit{ex situ} (blue) and \textit{in situ} (red) buffer layer structures.
        The expected peak positions for the (30.0) reflection of \cro\ (green dotted) and \agao\ (red dotted) are indicated.
        \textbf{(b)}~\gls{RSM} around the (30.0) reflection recorded before (left) and after (right) \textit{ex situ} deposition of \gao\ on \cro.
    }
    \label{Fig:Results_4_buffer_m}
\end{figure}
\begin{figure}
    \centering
    \begin{tabular}{c}
        \multicolumn{1}{l}{\textbf{(a)}}
        \figSpace \\
        \includegraphics{4_thetaomega_a.pdf}
        \figSpace \\
        \multicolumn{1}{l}{\textbf{(b)}}
        \figSpace \\
        \includegraphics{4_RSM_a.png}
    \end{tabular}
    \caption{
        \textbf{(a)}
        \thetaomega-patterns of the \textit{a}-plane \cro\ reference sample (black), as well as the \textit{ex situ} (blue) and \textit{in situ} (red) buffer layer structures.
        The expected peak positions for the (22.0) reflection of \cro\ (green dotted) and \agao\ (red dotted) are indicated.
        \textbf{(b)}~\gls{RSM} around the (22.0) reflection recorded before (left) and after (right) \textit{ex situ} deposition of \gao\ on \cro.
    }
    \label{Fig:Results_4_buffer_a}
\end{figure}
\begin{figure}
    \centering
    \begin{tabular}{c}
        \multicolumn{1}{l}{\textbf{(a)}}
        \figSpace \\
        \includegraphics{4_thetaomega_r_labelled.pdf}
        \figSpace \\
        \multicolumn{1}{l}{\textbf{(b)}}
        \figSpace \\
        \includegraphics{4_RSM_r.png}
    \end{tabular}
    \caption{
        \textbf{(a)}
        \thetaomega-patterns of the \textit{r}-plane \cro\ reference sample (black), as well as the \textit{ex situ} (blue) and \textit{in situ} (red) buffer layer structures.
        The expected peak positions for the (02.4) reflection of \cro\ (green dotted) and \agao\ (red dotted) are indicated.
        \textbf{(b)}~\gls{RSM} around the (02.4) reflection recorded before (left) and after (right) \textit{ex situ} deposition of \gao\ on \cro.
    }
    \label{Fig:Results_4_buffer_r}
\end{figure}