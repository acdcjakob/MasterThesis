The analysis of the data will not be structured into the 1st and 2nd batch, but into the analysis of (i) \textit{c}-plane, (ii) \textit{r}-plane and (iii) \textit{m}- and \textit{a}-plane samples.
In the following, some general remarks on the fabricated samples will be made.
\begin{figure}
    \centering
    \includegraphics{3_lensPos_growthrates.pdf}
    \caption{
    Growth rates of the samples from the 1st batch, depending on the pulse number (top) and depending on the laser fluence on the target for an approx.\ fixed pulse number (bottom).
    The data points are the mean of the four samples with another orientation each, that were obtained from every process.
    The errorbar displays the standard deviation.
    }
    \label{Fig:Results_3_lensGrowthRate}
\end{figure}

In Fig.\,\ref{Fig:Results_3_lensGrowthRate}, a detailed view into the growth rates of the samples of the 1st batch is given.
First of all, for a fixed fluence (fixed lens position), increasing the pulse number decreases the growth rate.
This is expected, because the coating of the laser entrance window increases during the process.
By fixing a pulse number, an increase in growth rate is observed for a regime of decreasing fluence from \qtyrange{2}{1}{\joule\per\cm\squared} (Fig\,\ref{Fig:Results_3_lensGrowthRate} bottom).
This can be explained by the fact that the reduction of fluence is due to increasing laser spot size.
When the fluence is still above the ablation threshold for the target material, an increasing ablation area results in an increasing growth rate.
But at some point the fluence is too low ablate the material and then the growth rate decreases, even though the ablation area increases.
This can be abserved at around \qty{1.2}{\joule\per\cm\squared} in Fig.\,\ref{Fig:Results_3_lensGrowthRate}, which is therefore an estimate for the ablation threshold.

\begin{figure}
    \centering
    \includegraphics{3_pulseEnergy_growthrates.eps}
    \caption{Growth rates of samples from the 2nd batch, depending on laser fluence on the target surface.
    The data points are the mean of thicknesses of the four orientations, similar to Fig.\,\ref{Fig:Results_3_lensGrowthRate}.}
    \label{Fig:Results_3_pulseGrowthRate}
\end{figure}