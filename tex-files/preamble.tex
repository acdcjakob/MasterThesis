\usepackage[utf8]{inputenc}

%% Colors
\usepackage{xcolor}
    \definecolor{shadecolor}{gray}{0.95}

%% images
\usepackage{graphbox}
    % loads graphics
    % allows for "align=c" argument in includegraphics for aligning vertically
% \usepackage{graphicx}
\graphicspath{
    {fastImages}
}
\usepackage{wrapfig}
\usepackage{}
%% layout
\usepackage{lmodern} % thicker font (?)
\usepackage[left=3cm,right=3cm,top=2cm,bottom=2cm]{geometry}
\usepackage{booktabs}
\usepackage{multirow}
% \usepackage{ctable}
\usepackage{framed}
% \usepackage{fancyhdr}
%     \pagestyle{fancy}

%% Text and Writing
% chemistry
\usepackage[version=4]{mhchem}
\usepackage{amssymb}
\usepackage{amsmath}

% units
\usepackage{siunitx}
    \DeclareSIUnit{\angstrom}{\textup{\AA}}
    \DeclareSIUnit{\bar}{bar}
    \sisetup{range-units=single}

% greek
\usepackage{textgreek}

% lists
\usepackage{enumitem}

% quotes
\usepackage{csquotes}

% figures
\usepackage[labelfont=bf]{caption}

%% Referencing
% Glossary
\usepackage[acronym]{glossaries}
% \makeglossaries % run `makeglossaries main' in terminal then compile again
    \newacronym{tco}{TCO}{Transparent Conductive Oxide}
    \newacronym{uwbg}{UWBG}{Ultrawide-bandgap}

    \newacronym{xps}{XPS}{X-ray Photoelectron Spectroscopy}
    \newacronym{XRD}{XRD}{X-ray diffraction}

    \newacronym{vbm}{VBM}{Valence Band Maximum}
    \newacronym{cbm}{CBM}{Conduction Band Minimum}
    \newacronym{dft}{DFT}{Density Functional Theory}
    \newacronym{DOS}{DOS}{density of states}

    \newacronym{cvd}{CVD}{Chemical Vapor Deposition}
    \newacronym{pld}{PLD}{Pulsed Laser Deposition}
    \newacronym{mbe}{MBE}{Molecular Beam Epitaxy}
    \newacronym{hvpe}{HVPE}{Halide Vapor Phase Epitaxy}
    \newacronym{rf}{RF}{radio-frequency}
    \newacronym{dc}{DC}{direct current}

    \newacronym{oop}{\textit{oop}}{out-of-plane}
    \newacronym{ip}{\textit{ip}}{in-plane}

    \newacronym{VCCS}{VCCS}{Vertical Continuous Composition Spread}

    \newacronym{UV}{UV}{ultra-violet}

    \newglossaryentry{pseudomorphic}{
        name = {pseudomorphic},
        description = {\tbd}
    }
    \newglossaryentry{bv}{
        name = {\textsc{Burger}'s vector},
        description = {\tbd}
    }
    \newglossaryentry{bl}{
        name = {\textsc{Bravais} lattice},
        description = {\tbd}
    }
    \newglossaryentry{bc}{
        name = {\textsc{Bragg} condition},
        description = {\tbd}
    }

% interactive
\usepackage[colorlinks=true,citecolor=blue,linkcolor=violet]{hyperref}

% --- own abbrev. ---
\newcommand{\agao}{\textalpha-\ce{Ga2O3}}

% !working on
\newcommand{\tbd}{\colorbox{magenta}{\emph{tbd}}}
\newcommand{\comment}[1]{
    \begin{shaded}
        \textsf{\underline{Comment}:
        \itshape\color{darkgray}
        #1
        }
    \end{shaded}
}

% --- Library stuff ---
\usepackage[style=numeric-comp,maxcitenames=2,sorting=none]{biblatex} %numeric-comp summerizes entries in \cite
\addbibresource{C:/Users/acdcj/OneDrive/Documents/Bibliothek/My Library.bib}
    \AtEveryBibitem{%
        \clearfield{month}%
        \clearfield{day}%
        \clearfield{urlyear}%
        \clearfield{urlmonth}%
        }
    \AtEveryCitekey{%
        \clearfield{month}%
        \clearfield{day}%
        \clearfield{urlyear}%
        \clearfield{urlmonth}%
        }

\usepackage{xpatch}

\xpatchbibmacro{textcite}
{\printnames{labelname}}
{\printnames{labelname} (\printfield{year})}
{}
{}