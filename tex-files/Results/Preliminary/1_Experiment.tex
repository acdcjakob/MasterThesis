Due to the similar crystal structure of \cro\ and \agao, the deposition parameters of the latter were chosen as a starting point to deposit chromia thin films on $10\times\qty{10}{\mm\squared}$ sapphire substrates with \textit{m}-plane orientation
    \cite{petersen2023}.
Namely, a pulse energy of \qty{650}{mJ} and a pulse frequency of \qty{20}{\Hz} were applied for a total of \qty{30000} pulses.
To investigate the influence of deposition parameters, three batches were produced:
\begin{enumerate}
    \item variation of oxygen partial pressure $p(\ce{O2})$ from \qtyrange{8e-5}{1e-2}{mbar} with a fixed heater temperature of \qty{745}{\degreeCelsius},
    \item variation of heater temperature from \qtyrange{725}{765}{\degreeCelsius} with a fixed oxygen partial pressure of \qty{1e-3}{mbar}, and
    \item variation of substrate orientation between \textit{c}- (00.1), \textit{r}- (01.2) \textit{m}- (10.0) and \textit{a}-plane (11.0) $5\times\qty{5}{\mm\squared}$ sapphire substrates\footnote{
        In the following, the \textsc{Bravais}-\textsc{Miller}-indices will be omitted.
        }
    with a fixed oxygen partial pressure of \qty{1e-3}{mbar} and a growth temperature of \qty{715}{\degreeCelsius}.
\end{enumerate}
Photographs of the first two batches are depicted in Fig.\,\ref{Fig:Results_1_samplesPhoto}.
Structural properties of those thin films were determined by \thetaomega-scans, \textomega-scans and \textphi-scans.
The thickness was determined via spectroscopic ellipsometry, and transmission spectra were recorded for two samples of the 1st batch to determine the optical band gap.
By dividing the thickness by the number of applied pulses, the growth rate $g$ can be calculated as is provided in units of \unit{\pm\per\pulse}.
Temperature dependent resistivity measurements were performed only for the samples of the 3rd batch, because all \textit{m}-plane oriented samples showed no conductivity at room temperature.

\begin{figure}
    \centering
    \includegraphics{camera_initial}
    \caption{
        Image of the samples produced at different oxygen partial pressures ($T=\qty{740}{\degreeCelsius}$) and different heater temperatures ($p(\mathrm{O_2})=\qty{1e-3}{\milli\bar}$), as denoted in the image.
    }
    \label{Fig:Results_1_samplesPhoto}
\end{figure}