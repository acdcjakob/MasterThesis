%!
\subsection{Pseudomorphic Growth}
\label{sec:Theory_PseudomorphicGrowth}
% \comment{
%     Ist der folgende Absatz zu lang, dafür dass ich (wahrscheinlich) nur bei \emph{c}-Orientierung pseudomorphic growth beobachte?
%     Aber ich wollte gerne ausrechnen, was denn der out of plane strain \emph{wäre}, falls es so sein sollte, damit ich später argumentieren kann, ob ich relaxed oder pseudomorphic beobachte; oder was dazwischen.
%     Dementsprechend hab ich mich dann gezwungen gefühlt, das ganze noch mal aufzurollen.
%     Vielleicht wäre eine Lösung, (\ref{euq:e3-a}), (\ref{euq:e3-m}) und (\ref{euq:e3-c}) in eine Art appendix zu tun?
%     }
When a body is deformed (\enquote{strained}) from its original state of equilibrium (\enquote{bulk}), forces will arise that tend to return the body to this equilibrium.
Molecular forces are the driving element behind these stresses
    \cite{landau1970}.
In continuum mechanics, stress $\sigma_{ij}$ and strain $\epsilon_{kl}$ are symmetric rank-2 tensors that are linearly connected by the elasticity tensor with components $C_{ijkl}$:
\begin{equation}\label{equ:stress-strain}
    \sigma_{ij}
    = \sum_{k,l} C_{ijkl}\epsilon_{kl}\,.
\end{equation}
Note that \eqref{equ:stress-strain} represents a set of 9 linear equations.

If the \gls{ip}\ lattice constants of two isomorphic compounds match at the interface of a heterostructure, one refers to \enquote{\gls{pseudomorphic}} growth.
This confines a part of the system of equations (\ref{equ:stress-strain}):
\begin{enumerate}
    \item The thin film \gls{ip}\ lattice constants have to match the substrate \gls{ip}\ lattice constants.
    This defines the magnitude of \gls{ip}\ strain of the thin film material.
    \item On the other hand, due to vertical growth, the \gls{oop}\ stress of the thin film is demanded to be zero.
\end{enumerate}
The resulting \gls{oop}\ strain as well as non-diagonal strain components can be derived by solving the system of equations (\ref{equ:stress-strain}) with these two boundary conditions.
In \textcite{grundmann2018}, formulas are derived for the unknown strains in the special case of pseudomorphic heterostructures with threefold rhombohedral symmetry.
For quantitative predictions of those strains, the elasticity tensor $C_{ijkl}$ of the thin film compound has to be known.
Depending on the symmetry of the crystal structure, its components collapse into a lower number of independent entries\footnote{
    Due to symmetry reasons \cite{ashcroft1976}, the nine indices $ij$ of the strain tensor can be unambiguously expressed by one index with six possible values: $11\rightarrow1$, $22\rightarrow2$, $33\rightarrow3$, $23\rightarrow4$, $13\rightarrow5$, $12\rightarrow6$ \cite{grundmann2018}.
    This allows for a $6\times6$-matrix representation of the elasticity tensor $C_{ijkl}\rightarrow C_{\mu\nu}$.
}:
for rhombohedral crystals, six independent components are left \cite{ashcroft1976}.
The entries of the elasticity tensor of the two sesquioxides important for this work are given in Tab.\,\ref{tab:Cr2O3-elasticityConstants}.
% --- elastic constants table
\begin{table}
    \centering
    \caption{The six independent entries of the elasticity tensor for rhombohedral \ce{Cr2O3} \cite{alberts1976} and \textalpha-\ce{Ga2O3} \cite{grundmann2018}.
    All values are in units of \qty{100}{\GPa}.}
    \begin{tabular}{lcccccc}
        \toprule%
        Material & $C_{11}$ & $C_{12}$ & $C_{13}$ & $C_{33}$ & $C_{44}$ & $C_{14}$\\\midrule
        \textalpha-\ce{Cr2O3} & 3.74 & 1.48 & 1.75 & 3.62 & 1.59 & $-0.19$\\
        \textalpha-\ce{Ga2O3}\quad{} & 3.82 & 1.74 & 1.26 & 3.46 & 0.78 & $-0.17$\\
        \bottomrule
    \end{tabular}
    \label{tab:Cr2O3-elasticityConstants}
\end{table}

Because of its direct influence on the \gls{oop}\ lattice plane distance and thus on the \gls{XRD} patterns of \thetaomega-scans (cf.~\ref{Sec:Methods_XRD}), the strain component perpendicular to the sample surface, $\epsilon_{zz}$, is of particular interest.
In the following, the formulas for the case of rhombohedral crystal structures are stated as derived in \textcite{grundmann2018}.
They depend on the respective \gls{ip}\ strains $\epsilon_{xx}$ and $\epsilon_{yy}$ caused by the lattice mismatch between film and substrate.
Note that here, $\mathbf{r}=(x,y,z)$ describes coordinates in the laboratory system -- in contrary to Ref.\,\cite{grundmann2018}, where $\mathbf{r}$ and $\mathbf{r}'$ are used to describe cartesian coordinates in the crystal and laboratory system, respectively.

One derives for (11.0)-plane (\textit{a}-orientation):
\begin{equation}
    \label{euq:e3-a}
    \epsilon_{zz,a}=-\frac{C_{13}\epsilon_{xx,a}+C_{12}\epsilon_{yy,a}}{C_{11}} \,,
\end{equation}
for (10.0)-plane (\textit{m}-orientation):
\begin{equation}
    \label{euq:e3-m}
    \epsilon_{zz,m}=-\frac{C_{13}C_{44}\epsilon_{xx,m}+(C_{12}C_{44}+C_{14}^2)\epsilon_{yy,m}}{C_{11}C_{44}-C_{14}^2} \,,
\end{equation}
and for (00.1)-plane (\textit{c}-orientation):
\begin{equation}
    \label{euq:e3-c}
    \epsilon_{zz,c}=-\frac{2C_{13}}{C_{33}}\epsilon_{yy,c} \,,
\end{equation}
with $\epsilon_{xx,a}=c_S/c_F-1$ and $\epsilon_{yy,a}=a_S/a_F-1$, depending on the lattice parameters of substrate ($a_S$, $c_S$) and film ($a_F$, $c_F$).
Note that 
\begin{align*}
    \epsilon_{xx,a}&= \epsilon_{xx,m}\,,\\
    \epsilon_{yy,a}&= \epsilon_{yy,m}\,,\\
    \epsilon_{yy,c}&= \epsilon_{yy,a}\,.
\end{align*}
For (01.2)-plane (\textit{r}-orientation), more complex relations arise as it was demonstrated by \textcite{grundmann2020}.
% \begin{align}
%     \label{equ:e3-r}
%     \epsilon_{zz,r}=&\frac{\tau_{zzxx}\epsilon_{xx}+\tau_{zzyy}\epsilon_{yy}}{\mu}\\
%     \dots\epsilon_{xx}=&\frac{a_S}{a_F}\cos^2\theta_r+\frac{c_s}{c_F}\sin^2\theta_r-1\\
%     \dots\epsilon_{yy}=&\frac{a_S}{a_F}-1\\
%     \dots\tau_{zzxx}=&\zeta_{zzxx}-2C_{14}(C_{33}+C_{13})\sin4\theta+2C_{14}^2\sin^22\theta\\
%     \dots\tau_{zzyy}=&\zeta_{yxyy}+\frac{C_{14}}{2}[3(\xi_8-C_{13})-4(C_{11}+C_{12}+C_{33})\cos2\theta_r\\
%     &+(\xi_8+3C_{13})\cos4\theta_r]-8\cos\theta_r\sin^3\theta_r\\
%     \dots\mu=&\\
%     \dots\eta=&\,.
% \end{align}
The distance of lattice planes $d$ orthogonal to the sample surface are then strained, such that:
\begin{equation}
    \label{equ:d_strained}
    d_\mathrm{strained}=d(1+\epsilon_{zz})\,.
\end{equation}
Assuming pseudomorphic growth of \ce{Cr2O3} on \ce{Al2O3}, one can compare the strained lattice plance distances to the unstrained bulk values, by utilizing (\ref{equ:d_strained}).
The numerical values, calculated from the lattice constants (Tab.\,\ref{Tab:sesquiLatticeConstants}) and the elasticity tensor (Tab.\,\ref{tab:Cr2O3-elasticityConstants}), are listed in Tab.\,\ref{tab:d_strained}a.
\begin{table}
    % \setlength{\tabcolsep}{12pt}
    \centering
    \caption{
        \textbf{(a)} Comparison of $d$ and $d_\mathrm{str}$, which are the \gls{oop}\ lattice plane distances for bulk \ce{Cr2O3} and pseudomorphic \ce{Cr2O3} on \ce{Al2O3}, respectively.
        The corresponding \gls{oop}-strain $\epsilon_{zz}$ is also given, as well as the corresponding angles of typical reflections in diffraction patterns of \thetaomega-scans.
        \textbf{(b)}~The resulting tilt of the thin film depending on substrate orientation for relaxed growth. The results follow from considerations on the possible slip systems and \glspl{bv}.}
    \begin{tabular}{cccccccccc}
        \toprule
        Orientation
            & \multicolumn{5}{c}{\textbf{(a)} Pseudomorphic}
            &&  \multicolumn{2}{c}{\textbf{(b)} Relaxed}
        \\
        \cmidrule(lr){2-6}
        \cmidrule(lr){8-9}
        (X-ray reflection)
            & $d$ (\si{nm}) & $d_\mathrm{str} (\si{nm})$ & $\epsilon_{zz} (\si{\percent})$
            &$2\theta$ (\si{\degree})&$2\theta_\mathrm{str}$ (\si{\degree})&& $\theta_{T,x}$   & $\theta_{T,y}$
        \\ \midrule
        \textit{c} (00.6)& $13.59$  & $14.12$   & $3.90$    & $39.75$   & $38.20$   &&--&--\\
        \textit{a} (11.0)& $2.48 $  & $2.57 $   & $3.63$    & $36.18$   & $34.87$   &&no&no\\
        \textit{m} (30.0)& $4.30 $  & $4.45 $   & $3.67$    & $65.06$   & $62.49$   &&yes&no\\
        \textit{r} (02.4)& $3.63 $  & $3.72 $   & $2.41$    & $50.19$   & $48.93$   &&yes&no\\
        \bottomrule
    \end{tabular}
    \label{tab:d_strained}
\end{table}

%! -----------------------------
\subsection{Relaxed Growth}
\label{Sec:Theory_Relaxed}
\subsubsection{Dislocations}
When the lattice mismatch is not resolved by adaption to the substrate (cf.~\ref{sec:Theory_PseudomorphicGrowth}), the strain is relieved by formation of \enquote{dislocations}, which faciliate a relaxed growth mode of the film
    \cite{kneiss2021}.
The highest disturbance from equilibrium spacing occurs close to the \enquote{dislocation line} which extends through the material --
far away from this line, the crystallinity is restored.
In which fashion the distortion happens, can be characterized by the \gls{bv} $\mathbf{b}$.
The relation of the \gls{bv} to the dislocation line determines the type of the dislocation:
    if they are orthogonal, one refers to an \emph{edge} dislocation;
    if they are parallel, one refers to a \emph{screw} dislocation.
For a \enquote{perfect} dislocation\footnote{
    Also referred to as \enquote{full} dislocation \cite{grundmann2016}.
}, the \gls{bv} is a lattice translation vector.
Note that in general, dislocations exhibit both edge- and screw-character
    \cite{hull2011}.

Dislocations are not static, but can move (\enquote{glide}) inside the crystal.
The movement happens typically inside a plane which has highest density of atoms (\enquote{glide plane}) and along the \gls{bv} which is responsible for the dislocation
    \cite{hull2011}.
The arrangement of glide plane and direction of movement is called \enquote{slip system}, e.g.\ for hexagonal structures, one finds $\{00.1\}/\langle11.0\rangle$ to be one prevailing slip system
    \cite{hull2011}.

For some slip systems occuring on heterostructures, the relaxation results in an additional tilt of the deposited film.
This happens because a \gls{bv} $\textbf{b}$ has more than one component:
the edge component $b_\parallel$ causes strain relaxation along $b_\parallel$;
but if $\mathbf{b}$ also exhibits a component $b_\perp$ orthogonal to the sample surface and the dislocation line, a tilt angle $\theta_T$ will result between substrate and relaxed film \cite{kneiss2021,grundmann2020b}:
\begin{equation}\label{equ:Theory_tiltDislocation}
    \theta_{T,i}=\epsilon_{ii}\frac{b_{i,\perp}}{b_{i,\parallel}}\,,
\end{equation}
where $i$ denotes the axis of strain relaxation.
This is schematically depicted in Fig.\,\ref{fig:Theory_tiltDislocation}.
Note that \eqref{equ:Theory_tiltDislocation} is formulated for partially relaxed systems in \textcite{kneiss2021}, but here the relaxation parameter $\rho_i$ was omitted, therefore assuming fully relaxed thin films. 

\begin{figure}
    \centering
    \includegraphics{grundmann2016_Fig.4.31.pdf}
    \caption{
        Edge dislocation with \gls{bv} perpendicular to the sample surface.
        The edge dislocation lines are indicated by the red dots and are perpendicular to the image plane.
        Consequently, the direction of the edge \gls{bv} is indicated by the red arrow, whose length $D$ denotes the distance between dislocation lines.
        The green \gls{bv} $\mathbf{b}$ is the orthogonal component $b_\perp$ mentioned in the text.
        }
    \label{fig:Theory_tiltDislocation}
\end{figure}
%! ---------------
\subsubsection{Slip Systems for Sesquioxide Heterostructures}
For heteroepitaxial growth of \ce{(Al_xGa_{1-x})2O3} with low \ce{Al} content on \ce{Al2O3}, studies have been conducted on the prevailing relaxation mechanisms for \textit{r}-oriented
    \cite{grundmann2020b,grundmann2020a},
as well as \textit{a}- and \textit{m}-oriented
    \cite{kneiss2021}
growth directions.
In the following, those results will be summarized.
Note that the $x$-axis points along the $c$-axis for \textit{m}- and \textit{a}-oriented he\-terostructures, and similarly along the projection of the $c$-axis on the sample surface for \textit{r}-oriented heterostructures.

% r-orientation
\paragraph{(01.2)-plane (\textit{r}-orientation)}
The two relevant slip systems are $\{00.1\}/\frac{1}{3}\langle11.0\rangle$ and $\{11.0\}/\frac{1}{3}\langle1\bar{1}.1\rangle$, which contain the \enquote{basal} and \enquote{prismatic} glide plane, respectively
    \cite{grundmann2020b}.
The former allows relaxation along the direction containing the projection of the \textit{c}-axis ($x$-axis), whereas the latter allows relaxation perpendicular to it ($y$-axis).
For the basal system, one can determine two possible independent \glspl{bv} $\mathbf{b}_c$ with differing screw components but otherwise same tilt and edge components $b_{c,\perp}$ and $b_{c,\parallel}$, respectively.
The tilt along $x$-direction can then be calculated via:
\begin{equation}
    \theta_{T,x}=\epsilon_{xx}\frac{b_{c,\perp}}{b_{c,\parallel}}=\frac{1}{\sqrt{3}}\zeta_F\epsilon_{xx}\,.
\end{equation}
with $\zeta_F=c_F/a_F$.
For the prismatic slip system, the possible \glspl{bv} facilitate relaxation along the $y$-direction via $b_{a,\parallel}$.
But in contrast to the basal system, the tilt components $b_{a,\perp}$ cancel out on average, thus resulting in no net tilt along the $y$-direction: $\theta_{T,y}=0$.

% m-orientation
\paragraph{(10.0)-plane (\textit{m}-orientation)}
Neither basal (00.1) nor prismatic (11.0) and (10.0) slip systems can resolve strain along the $x$-axis:
The (00.1)-plane is perpendicular to the surface and $x$-direction, thus the \gls{bv} can only have components in the $y$-$z$-plane.
But for strain release along $x$, the \gls{bv} should have non-zero component in this direction, which cannot be the case.
The prismatic planes, on the other hand, are perpendicular to the surface but parallel to the $x$-axis.
This results in a dislocation line along the $x$-direction.
To release strain, the \gls{bv} would have $x$-component, which does not apply for edge dislocations.
So the prevailing slip system must have (01.2)-plane (\textit{r}-orientation) or (11.2)-plane (\textit{s}-orientation) character, which are called \enquote{pyramidal} slip systems.
Three different \textit{r}-planes contribute to strain release, because there is dislocation line component along the $y$-axis and \gls{bv}'s components along the $x$-axis.
With \eqref{equ:Theory_tiltDislocation} and substituting the possible \glspl{bv} one finds:
\begin{equation}
    \theta_{T,x}=\frac{2}{3}\frac{\sqrt{3}}{\frac{20\zeta}{24+6\zeta^2}+\zeta}\left(\frac{c_S}{c_F}-1\right)
\end{equation}

% a-orientation
\paragraph{(11.0)-plane (\textit{a}-orientation)}
The same argument as for the \textit{m}-oriented hetereostructure holds, why only pyramidal slip systems are possible.
But in this case, only two \textit{r}-planes contribute to strain relaxation, because the third plane is perpendicular to the surface, thus can only exhibit \glspl{bv} without in-plane components which results in no possible edge dislocations.
Furthermore, in this case, the \glspl{bv} of the two remaining \textit{r}-planes have opposite tilt components, i.e. they point outwards and inwards of the surface, respectively.
Regarding \eqref{equ:Theory_tiltDislocation}, this will result in no net tilt of the thin film.

% \begin{table}
%     \centering
%     \caption{The resulting tilt of the thin film depending on substrate orientation. The results follow from considerations on the possible slip systems and \glspl{bv}.}
%     \begin{tabular}{cll}
%         \toprule
%         Orientation         & $\theta_{T,x}$   & $\theta_{T,y}$       \\
%         \midrule
%         (01.2) \textit{r}   & yes               & no                    \\
%         (11.0) \textit{a}   & no                & no                    \\
%         (10.0) \textit{m}   & yes               & no                    \\
%         \bottomrule
%     \end{tabular}
% \end{table}