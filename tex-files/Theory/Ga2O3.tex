\ce{Ga2O3} is a group-III sesquioxide with four different polymorphs, of which \textbeta-\ce{Ga2O3} is the thermodynamically most stable one at ambient conditions
    \cite{schewski2015,hassa2021a,petersen2023}.
The corundum-structured \agao{} phase, which is of more relevance for this work, is isomorphic to \ce{Cr2O3} (cf.~\ref{Sec:Cr2O3}), with lattice parameters as listed in Tab.\,\ref{Tab:sesquiLatticeConstants}.
\agao{} is metastable
    \cite{kaneko2023},
i.e. not favored in the first place, but remains irreversibly after formation, e.g., after phase transition from \textbeta- to \textalpha-phase at high temperatures
    \cite{pearton2018}.
The thermodynamic equilibrium -- which determines the favored phase -- can also be changed by strain due to lattice mismatch occurring during heteroepitaxy\footnote{
    However, the possibility of formation of parasitic \textbeta-phase still has to be taken into account
        \cite{petersen2023}.
    }
    \cite{schewski2015}.
This approach is of particular interest due to the possibility of deposition on cheap\footnote{
    Compared to bulk \textbeta-\ce{Ga2O3} substrates
        \cite{yang2022,kaneko2023}.
}
and readily available sapphire substrates which are isomorphic to \agao{}
    \cite{pearton2018,polyakov2022,kaneko2023}.
Note that deposition of \textbeta-\ce{Ga2O3} on sapphire is also possible, but only with restriction to formation of more than one crystal domain
    \cite{yang2022}.
On the other hand, highly crystalline
    \cite{pearton2018}
\agao{} thin films should be able to be grown without rotational domains
    \cite{yang2022}.

Deposition of \agao{} on sapphire has been done by several deposition techniques, including \cite{yang2022}:
    \gls{hvpe},
    mist \gls{cvd}
        \cite{kaneko2012},
    \gls{mbe}
        \cite{schewski2015},
    Atomic Layer Deposition and
    metalorganic \gls{cvd}.
Phase-pure deposition via \gls{pld} has also been achieved
    \cite{schewski2015,petersen2023}.
Despite being isomorphic to each other, \agao{} and sapphire still exhibit a lattice mismatch of around \qty{4.8}{\percent} along the \textit{a}-axis
    \cite{kaneko2023}.
This induces semicoherent growth with a fairly high dislocation density, which has been reported to be around \qty{7e10}{\per\square\cm}
    \cite{kaneko2012}.
In particular, this becomes a problem regarding carrier mobility which is tremendously hindered by dislocation scattering
    \cite{kaneko2023}.

To overcome the problems of lattice mismatch between sapphire substrates and \agao{} thin films, quasi-continuous gradients from \ce{Al2O3} to \agao{} have been applied, utilizing the capability of alloying the respective compounds
    \cite{jinno2016}.
Furthermore, buffer layers of isomorphic \ce{Cr2O3} have been used to decrease the high dislocation density for deposition on \textit{c}-oriented
    \cite{stepanov2021,polyakov2022a}
as well as \textit{r}-oriented sapphire
    \cite{polyakov2022}.
Deposition on other than \textit{c}-oriented substrates also seems to decrease parasitic phases, because the suppression of crystal facets perpendicular to the principal \textit{c}-axis may increase phase purity
    \cite{jinno2021}.
It has to be noted that despite the difficulties occuring upon lattice mismatch, coherent growth seems to be feasible without buffer layers for different deposition techniques, at least for some monolayers
    \cite{schewski2015}.

% --- electronic structure ---
With \qtyrange{5.0}{5.3}{\eV}
    \cite{yang2022},
\agao{} has the highest band gap of the four polymorphs
    \cite{pearton2018}.
Increasing or decreasing the band gap is possible by alloying with \ce{Al2O3}
    \cite{jinno2021}
or \ce{In2O3}
    \cite{hassa2020},
respectively.
The crystal structure also allows for alloying with other corundum structured compounds
    \cite{yang2022},
in particular other transition metal oxides such as \ce{Cr2O3}
    \cite{polyakov2022,polyakov2022a}.
The conduction band is mainly composed of \ce{Ga}-$4s$ states with an effective electron mass of \qty{0.3}{m_e}.
The valence band is very flat and mainly composed of \ce{O}-$2p$ orbitals, yielding a high effective electron mass and thus strong localization
    \cite{pearton2018}.
Next to band gap engineering, \textit{n}-type doping via \ce{Sn} or \ce{Si} incorporation has been accomplished
    \cite{yang2022}.
