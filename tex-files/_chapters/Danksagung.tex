\chapter*{Danksagung}
\addcontentsline{toc}{chapter}{Acknowledgements}
Ich danke Prof.\ Grundmann für die Betreuung und Korrektur dieser Arbeit sowie für die Möglichkeit, dass ich in der Arbeitsgruppe der Halbleiterphysik forschen durfte.
Weiterhin möchte ich mich bei Dr.\ habil.\ Holger von Wenckstern dafür bedanken, dass er mir stets mit vielen Anregungen zu meiner Forschung zur Seite stand, immer (!) offen für Fragen war und mir regelmäßig positives Feedback zu meinen Ergebnissen gegeben hat.
Für die administrativen Aufgaben bedanke ich mich bei Anja, Birgit und Johanna.

Ganz besonders großer Dank gilt Sofie und Clemens, die mich beide sehr intensiv betreut haben und keine Zeit und Mühen gescheut haben um meine Fragen zu beantworten, mir die verschiedensten Herstellungs- und Messmethoden zu zeigen, meine Arbeit Korrektur zu lesen und Vorschläge für zukünftige Untersuchungen zu geben.
Weiterhin danke ich Gabi, Lukas und Chris für die produktiven Gespräche über Optik und Ellipsometrie;
Daniel, Peter und Micha für die Hilfe mit diversen Laborgeräten;
Holger Hochmuth für die Hilfestellungen an der W-Kammer;
Monika Hahn für die Targetherstellung und Transmissionsmessungen;
Prof.\ Lorenz für die Einweisung in die XRD Messung;
Christiane für die von ihr durchgeführten AFM Messungen;
Jorrit für die Messungen der Laserspotgröße und -energie;
Max für Erklärungen bezüglich Heteroepitaxie;
sowie Dr.\ Javier Fernandez für die TEM Messungen.
Zwar nicht relevant für diese Arbeit, aber dennoch wichtig war auch die Zeit mit Oliver Lahr -- während mein Herz noch für amorphes ZTO schlug, und ich noch nicht ahnen würde dass ich seitenweise über RSMs schreiben würde.
Neben den wissenschaftlichen Unterhaltungen bin ich auch sehr dankbar für die schönen Gespräche die in und außerhalb des gelben Salons geführt wurden.
Neben den genannten Personen möchte ich mich deshalb auch bei
Fabian, Aaron, Arne, Arthur, Jonas, Laurenz, Ron, Yang, Paul, Basti, Hannah, Simon, Tim, Sascha und Dmitry bedanken.
Ich hoffe, dass die HLP auch in Zukunft bei diversen Volleyball-Turnieren stark vertreten sein wird.

Außerdem danke ich meiner Familie, dass sie mich über meine verschiedenen Wege begleitet und unterstützt hat, vom Wechsel als angehender Musiklehrer zum Physik-Bachelor, und dann nochmals während meines Philosophiestudiums während des Physik-Masters.
Daneben haben auch meine Freunde diese Arbeit möglich gemacht, zu denen ich nun auch einige Mitglieder der HLP Arbeitsgruppe zählen kann.