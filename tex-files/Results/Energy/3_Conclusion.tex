The out-of-plane strain and in-plane strain of \cro\ thin films with different crystal orientations were investigated depending on the laser fluence on the target during deposition via \gls{pld}.
The variation in laser fluence was achieved by either increasing the laser spot size or by decreasing the laser pulse energy.
Samples with \textit{c}-orientation grow fully pseudomorphic for low film thicknesses, wheras the samples with \textit{r}-, \textit{m}- and \textit{a}-orientatiin are partially relaxed.
The thickness is a crucial parameter for \textit{c}- and \textit{r}-plane samples, whereas the laser fluence on the target strongly influences the crystal structure of \textit{m}- and \textit{a}-plane samples.
It can be concluded that the thickness of the thin films is more relevant for orientations that have more out-of-plane \textit{c}-axis component.
For \textit{c}- and \textit{r}-orientation, less laser fluence -- no matter whether via larger laser spot sizes or reduced laser pulse energy -- results in less \gls{FWHM} in \textomega-patterns and thus better crystallinity.
For \textit{m}-plane samples, a reversed behavior is observed -- however, the dependence on fluence is much less pronounced.
A layered structure of \textit{a}-plane thin films hardens the comparison of the \textomega-FWHM for this orientaiton.
Furthermore, for \textit{r}-, \textit{m}- and \textit{a}-plane oriented samples, thin film tilts have been observed in the directions that were previously observed for relaxed \agao\ layers on \alo\ 
    \cite{grundmann2020b,kneiss2021}
(cf.\ Tab.\,\ref{tab:d_strained}b).

To understand the dependence between relaxation, crystallinity and thin film tilt for \cro\ \textit{m}-plane thin films, a comparison to similarly structured \agao\ can be made:
In \textcite{kneiss2021}, fully relaxed \textit{m}-plane \textalpha-\ce{(Al_xGa_{1-x})2O3} thin films exhibited a tilt that was dependent on the aluminum content, and therefore on relaxed lattice constants.
For pure \agao\ layers (zero aluminum content), a thin film tilt of \qty{36}{\arcminute} was observed, which is in accordance to the here observed angles of \qty{20}{\arcminute} to \qty{40}{\arcminute} (cf.\ Fig.\,\ref{Fig:Results_3_pulse_ma_strainTilt}a).
In this work, however, due to variations in laser fluence, partially relaxed \textit{m}-plane \cro\ thin films with out-of-plane strain ranging from \qty{0.02}{\percent} to \qty{0.3}{\percent}\footnote{
    According the RSMs. Note that there was a discrepancy between the strain values obtained from \thetaomega\ scans and RSMs.
} could be fabricated (Fig.\,\ref{Fig:Results_3_pulse_ma_strainTilt}a).
It was observed that samples with less out-of-plane strain -- i.e.\ smaller out-of-plane lattice constants -- exhibit smaller thin film tilts.
Note that due to the fact that \alo\ has smaller lattice constants than relaxed \cro, a decrease in lattice constants is directly related to a decreasing mismatch between substrate and thin film.
Therefore, the here reported results are in accordance to \textcite{kneiss2021} in the sense that a reduction in lattice mismatch reduces the thin film tilt.
Furthermore, reducing the laser fluence also results in a slightly higher \textomega-FWHM (Fig.\,\ref{Fig:Results_3_pulseOmega}), which could correspond to an increased dislocation density.
When applying a heteroepitaxial model as displayed in Fig.\,\ref{fig:Theory_tiltDislocation}
    \cite{grundmann2016},
this reduced dislocation density results in less thin film tilt, which is indeed observed.
Therefore, an increasing laser fluence on the PLD target results in both increasing strain and decreasing \textomega-FWHM which manifest in larger thin film tilts.
These effects can be summarized as less partially relaxed layers.

However, note that the importance of reduced laser fluence for more relaxed thin films is not explained yet.
By taking the reduced kinetic energy of the plasma species into account
    \cite{anisimov1996},
one can infer that the growth dynamics are altered in such a way that the formation of dislocations is favored.
In general, note that more sophisticated methods like \gls{TEM} should be applied to get a more detailed view into the formation and density of dislocations, which are fundamental to relaxation and thin film tilt.

No investigations of the influence of laser fluence on the structural properties of \cro\ thin films fabricated by PLD have been done so far in the literature.
The only report on varying laser fluence from \qtyrange{1.6}{3.7}{\J\per\cm\squared} has been concerning the atomic ratio of \ce{Cr} cations to \ce{O} anions for deposition on silicon
    \cite{tabbal2006}.
Most of the \cro\ thin films fabricated via PLD were deposited on \textit{c}-plane sapphire
    \cite{singh2019,arca2017,kehoe2016}.
None of those studies could identify the laser fluence as a crucial paramter influencing the crystallinity of the thin films (Fig.\,\ref{Fig:Results_3_pulseOmega}):
the best \textomega-FWHM of \qty{22}{\arcminute} that was reported by \textcite{singh2019} is larger than the here reported value of \qty{7}{\arcminute}.
For \textit{r}-plane oriented thin films reported by \textcite{punugupati2015}, an out-of-plane strain of \qty{0.57}{\percent} could be identified, but no studies have been performed on the origin of this observation.

To determine the optimal deposition parameters of \cro, one has to take the different orientations into account:
Because the thickness is mostly relevant for \textit{c}-plane oriented thin films, the choice of at least $t=\qty{150}{\nm}$ results in the lowest strain and \textomega-FWHM.
Furthermore, a laser spot size of \qty{10}{\mm\squared} and pulse energy of \qty{300}{\milli\J} would result in the lowest strain for \textit{r}-, \textit{m}- and \textit{a}-orientation.
Note that this, however, results in less crystalline films for those orientations (Fig.\,\ref{Fig:Results_3_pulseOmega}), as well as a very low growth rate.
Therefore, a pulse energy of \qty{450}{\milli\J} is chosen for future depositions due to the best crystal quality while maintaining strain of about \qty{0.1}{\percent} and \qty{0.3}{\percent} for \textit{m}- and \textit{a}-plane respectively.
Those deposition parameters also result in low thin film tilts of about \qty{15}{\arcminute} and \qty{25}{\arcminute} for \textit{r}- and \textit{m}-plane samples, respectively.
The laser fluence on the target is therefore \qty{1.1}{\J\per\cm\squared}, and will be applied when depositing high quality buffer layers for \agao\ thin films:
The reduced compressive in-plane strain is desired to achieve low mismatch between the \cro\ and \agao\ layer, as demonstrated in the following chapter.

% In \textcite{kneiss2021}, a relaxation paramter $\rho_x$ was successfully introduced to quantify the amount of partial relaxation along the $x$-axis for \textit{m}- and \textit{a}-plane oriented \textalpha-\ce{(Al_xGa_{1-x})2O3} thin films.
% Higher values of $\rho_x$ correspond to more pseudomorphic growth and were correlated to smaller tilt angles.
% However, the results here object to this observation:

% The explanation of the thin film tilt by the relaxation model does not seem to be the origin of this discrepancy:
% The axis of thin film tilt was correctly predicted to be only along $x$-direction for \textit{m}- and \textit{r}-plane thin films and along no direction for \textit{a}-plane samples.
% Nevertheless, the result of this observation is that no trade-off has to be made between strain and thin film tilt -- a lens position of \qty{-1}{\cm} and laser pulse energy of \qty{450}{\mJ} yields almost no strain and low thin film tilt while still resulting in a growth rate of up to \qty{4}{\pm\per\pulse}.