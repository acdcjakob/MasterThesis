\section{Strain Analysis}
    \label{Sec:Results_Energy}
%! Motivation
The structural properties of the thin film, namely its mosaicity and lattice distortion depend crucially on the growth process.
It turned out that the absorption of energy at the laser entrance window alters the growth rate and the crystallinity much more dominantly than the growth temperature or the oxygen partial pressure (cf.\ \ref{Sec:Results_Preliminary}).
A similar effect was observed when targets were used for fabrication that exhibit a non-planar surface and tracks that were carved during previous ablations (cf.\ \ref{Sec:Results_Doping}).
Because the structural properties of the thin film also influence its electrical properties (cf.\ \ref{Sec:Results_Doping}), the following investigations focus on the origin of the observed variations in strain and \textomega-FWHM.
This is further motivated by the observation that a deliberate and controlled variation of laser spot size on the target surface yields a large reduction of \textomega-FWHM as well as a reduced shift of the peak position in the \thetaomega-pattern (Fig.\,\ref{Fig:Results_3_motivation}).
This was achieved by varying the lens position (cf.\ \ref{Fig:Methods_pld}) such that the laser spot size increases, yielding smaller fluence and larger ablation area on the target surface.
\begin{figure}[h]
    \centering
    \includegraphics{3_fluence_motivation.eps}
    \caption{
    \thetaomega-patterns for two \textit{c}-plane samples fabricated with different laser focus on the target.
    The inset displays the diffractograms of the corresponding \textomega-scans performed on the respective reflections.
    The ZnO-doped (low) target was used without a fixed $r_\mathrm{PLD}$ but with uniform ablation on the whole target surface.
    }
    \label{Fig:Results_3_motivation}
\end{figure}

\subsection{Experiment}
    %! Sample Fabrication
\subsubsection{Sample Fabrication}
For all following depositions, the laser entrance window was cleaned before each process.
A pure \cro\ target was used for deposition of thin films on $5\times\qty{5}{\mm\squared}$ sapphire substrates in the four aforementioned orientations.
The first batch of samples was produced by only varying the pulse number to achieve a series of thin films with varying thickness but constant laser fluence during deposition.
This is necessary, because the series of thicknesses that was achieved in the prior experiments was correlated to a series of growth rates.
The pulse energy was set to \qty{650}{\milli\joule} and the lens position\footnote{
    Note that the values for the lens position have an arbitrary offset; a value of \qty{0}{\cm} does not correspond to the position where the target surface is in focus.
}
to \qty{-2}{\cm}, the resulting fluence is approx.\ \qty{2}{\joule\per\cm\squared}.
This corresponds to the standard configuration during all previous processes (pink square in Fig.\,\ref{Fig:Methods_fluence}).
This was repeated for three other lens positions, namely \qtylist{0;1;2}{\cm}, resulting in lower fluences:
In Fig.\,\ref{Fig:Methods_fluence}, the yellow circles represent the probed laser fluences.
This set of samples is referred to as the 1st batch.

To investigate the influence of fluence independent of ablation area, a 2nd batch of samples was fabricated with a fixed lens position (\qty{-1}{\cm}) but varying laser pulse energy:
\qtylist{300;450;650;800}{\milli\joule}.
The pulse number was adjusted to achieve approximately same thicknesses.
The achieved fluences are visualized as red triangles in Fig.\,\ref{Fig:Methods_fluence}.

%! Measurements
\subsubsection{Measurements}
For all samples, \thetaomega-scans as well as \textomega-scans were performed.
The reflections probed by the latter were (00.6), (02.4), (30.0) and (22.0) for \textit{c}-, \textit{r}-, \textit{m}- and \textit{a}-plane, respectively.
For some selected samples of different thickness and fluence from the 1st batch, transmission measurements have been performed.
Two of them were investigated via \gls{AFM} to obtain information about the surface morphology.
The thickness of all samples was determined by spectroscopic ellipsometry measurements.
To obtain more information about the relation between in-plane and out-of-plane lattice constants, \glspl{RSM} were performed on selected samples:

\paragraph{\textit{c}-plane}
    For \textit{c}-plane samples, the thickness series of the 1st batch that was fabricated with the largest laser spot size (lowest fluence) was investigated.
    The asymmetric reflection that was used for probing the relaxation process is (02.10), which has an inclination angle of approx.\ \qty{32}{\degree} with respect to the sample surface.
\paragraph{\textit{r}-plane}
    All \textit{r}-plane samples fabricated in the 2nd batch with different laser pulse energies were investigated with \glspl{RSM}.
    For each sample, the $x$-axis of the sample -- containing the projection of the \textit{c}-axis -- is found by performing a \textphi-scan on the (03.0) reflection:
    This set of lattice planes has an inclination with respect to the surface, so the position of the peak in the diffraction pattern of the \textphi-scan reveals the $x$-axis.
    In this azimuth, an \gls{RSM} is recorded around the asymmetric (03.0) reflection and the symmetric (02.4) reflection.
    By rotating $\Delta\phi=\qty{90}{\degree}$, the $y$-axis lays in the scattering plane and another \gls{RSM} is performed around the symmetric (02.4) reflection.
    The twofold measurement of the symmetric reflection is necessary to calculate a possible lattice plane tilt for both $x$- and $y$-direction.
    Note that no shear is calculated due to the asymmetric nature of the (03.0) reflection with respect to the \textit{r}-orientation\footnote{
        For \textit{m}- and \textit{a}-plane rhombohedral structures, the crystal is symmetric under the transformation $\phi\rightarrow\phi+\qty{180}{\degree}$, which is not the case for \textit{r}-plane.
    }.
    After performing the various corrections described in \ref{Sec:Methods_RSM}, the tilt angles can be calculated for both azimuths by
    \begin{eqnarray}
        \theta = \arccos\left(
            \frac{q_\perp}{|\mathbf{q}|}
        \right) \cdot\mathrm{sgn}\left(q_\parallel\right)\,,
        \label{Equ:Results_3_tiltAngle}
    \end{eqnarray}
    with $q_\perp$ and $q_\parallel$ being the \gls{oop}\ and \gls{ip}\ components of the scattering vector $\mathbf{q}$, respectively.
    The \gls{ip}\ and \gls{oop}\ strains are determined by comparing the observed scattering vector to the expected scattering vector for the (03.0) reflection:
    \begin{equation}
        \mathbf{q}_\mathrm{(03.0)} = 
        \left|\mathbf{q}_\mathrm{(03.0)}\right|\cdot
        \begin{pmatrix}
            \cos\alpha_{(03.0)|r}\\
            \sin\alpha_{(03.0)|r}
        \end{pmatrix}\,,
    \end{equation}
    with $|\mathbf{q}_{(03.0)}|$ calculated from \eqref{Equ:Methods_qAbs} and \eqref{Equ:Methods_dhkl}.
    $\alpha_{(03.0)|r}$ denotes the angle between the (03.0) reflection and the normal of the \textit{r}-planes; it can be calculated from \eqref{Equ:Methods_angleWRTc}:
    \begin{equation}
        \alpha_{(03.0)|r}
        = \qty{90}{\degree}-\left(
            \alpha_{(03.0)|c}-\alpha_{(01.2)|c}
        \right)
        = \alpha_{(01.2)|c}
        = \qty{57.62}{\degree}\,.
    \end{equation}
\paragraph{\textit{m}-plane}
    Similar to above, all \textit{m}-plane samples from the 2nd batch were investigated.
    The samples were aligned to the $x$-axis by performing a \textphi-scan on the asymmetric (30.6) reflection, and an \gls{RSM} was recorded afterwards.
    By rotating $\Delta\phi=\qty{180}{\degree}$ while maintaining $2\theta$ and $\omega$, the scattering condition for $(30.\overline{6})$ is probed and an \gls{RSM} was recorded.
    The symmetric reflection (30.0) was also measured in this azimuth.
    The tilt angle and shear angle can be calculated according to \eqref{Equ:Results_3_tiltAngle} and \eqref{Equ:Methods_shearAngle}, respectively.
    The lattice constants can be calculated from the components of the scattering vectors:
    \begin{align}
        a_\perp &= \frac{\sqrt{12}}{q_\perp^{(30.\pm6)}} \,,\\
        a_\perp &= \frac{\sqrt{12}}{q_\perp^{(03.0)}}\,,\\
        c &= \frac{6}{q_\parallel^{(30.\pm6)}} \,.
    \end{align}
    $a_\perp$ denotes the $a$ lattice constant in direction of the normal to the sample surface.
    By rotating $\Delta\phi=\qty{90}{\degree}$, the \textit{y}-axis can be probed via asymmetric reflections $(\overline{4}2.0)$ and (22.0), which differ in the azimuth by $\Delta\phi=\qty{180}{\degree}$.
    A second symmetric reflection (30.0) is recorded in this azimuth.
    Similar to the $x$-axis, the tilt and shear angles, as well as the lattice constants can be calculated:
    \begin{align}
        (4\overline{2}.0):&\quad
            a_\perp = \frac{\sqrt{12}}{q_\perp^{(4\overline{2}.0)}}
            \quad,\quad
            a_\parallel = \frac{2}{q_\parallel^{(4\overline{2}.0)}}\,,\\
        (22.0):&\quad
            a_\perp = \frac{\sqrt{12}}{q_\perp^{(22.0)}}
            \quad,\quad
            a_\parallel = \frac{2}{q_\parallel^{(22.0)}}\,,\\
        (30.0):&\quad
            a_\perp = \frac{\sqrt{12}}{q_\perp^{(03.0)}}\,.
    \end{align}
    $a_\parallel$ denotes the $a$ lattice constant parallel to the $y$-axis.
    For detailed calculations of the former equations, see\ \ref{Sec:App_Calc_mPlane}.
    Note that all 6 measured reflections yield a value for $a_\perp$, and 2 measured reflections each yield 2 values for $c$ and $a_\parallel$, respectively.
    Therefore, for each lattice constant, the mean value is evaluated and the error is estimated by the standard deviation (cf.\ Fig.\,\ref{Fig:Results_3_pulse_ma_strainTilt}a).
\paragraph{\textit{a}-plane}
    All \textit{a}-plane samples from the 2nd batch were investigated and the method is similar to the one applied to the \textit{m}-plane samples.
    The azimuth of the \textit{x}-axis is found by performing a \textphi-scan on the (22.6) reflection, which also served for an \gls{RSM}.
    Rotating by $\Delta\phi=\qty{180}{\degree}$ yields the $(22.\overline{6})$ reflection and (22.0) is also measured.
    Similar to above, the sample is rotated by \qty{90}{\degree} to align to the $y$-axis and two more asymmetric reflections are recorded: (30.0) and (03.0).
    A second \gls{RSM} of (22.0) is also performed.
    This yields the following lattice constants for the $x$-axis:
    \begin{align}
        a_\perp &= \frac{4}{q_\perp^{(22.\pm6)}} \,,\\
        a_\perp &= \frac{4}{q_\perp^{(22.0)}}\,,\\
        c &= \frac{6}{q_\parallel^{(22.\pm6)}} \,,
    \end{align}
    and for the $y$-axis:
    \begin{align}
        (30.0):&\quad
            a_\perp = \frac{2}{q_\perp^{(30.0)}}\cdot\frac{3}{2}
            \quad,\quad
            a_\parallel = \frac{2}{\sqrt{3}q_\parallel^{(30.0)}}\cdot\frac{3}{2}\,,\\
        (03.0):&\quad
            a_\perp = \frac{2}{q_\perp^{(03.0)}}\cdot\frac{3}{2}
            \quad,\quad
            a_\parallel = \frac{2}{\sqrt{3}q_\parallel^{(03.0)}}\cdot\frac{3}{2}\,,\\
        (22.0):&\quad
            a_\perp = \frac{4}{q_\perp^{(22.0)}}\,.
    \end{align}
    For detailed calculations and the origin of the factor $\frac{3}{2}$, see\ \ref{Sec:App_Calc_aPlane}.
    Again, lattice constants obtained from several reflections, the mean and standard deviation are calculated (cf.\ Fig.\,\ref{Fig:Results_3_pulse_ma_strainTilt}b).

    \label{Sec:Results_3_Experiment}
\subsection{Results}
    The analysis of the data will not be structured into the 1st and 2nd batch, but into the analysis of (i) \textit{c}-plane, (ii) \textit{r}-plane and (iii) \textit{m}- and \textit{a}-plane samples.
In the following, some general remarks on the fabricated samples will be made.
\begin{figure}
    \centering
    \includegraphics{3_lensPos_growthrates.pdf}
    \caption{
    Growth rates of the samples from the 1st batch, depending on the pulse number (top) and depending on the laser fluence on the target for an approx.\ fixed pulse number (bottom).
    The data points are the mean of the four samples with another orientation each, that were obtained from every process.
    The errorbar displays the standard deviation.
    }
    \label{Fig:Results_3_lensGrowthRate}
\end{figure}

In Fig.\,\ref{Fig:Results_3_lensGrowthRate}, a detailed view into the growth rates of the samples of the 1st batch is given.
First of all, for a fixed fluence (fixed lens position), increasing the pulse number decreases the growth rate.
This is expected, because the coating of the laser entrance window increases during the process.
By fixing a pulse number, an increase in growth rate is observed for a regime of decreasing fluence from \qtyrange{2}{1}{\joule\per\cm\squared} (Fig\,\ref{Fig:Results_3_lensGrowthRate} bottom).
This can be explained by the fact that the reduction of fluence is due to increasing laser spot size.
When the fluence is still above the ablation threshold for the target material, an increasing ablation area results in an increasing growth rate.
But at some point the fluence is too low ablate the material and then the growth rate decreases, even though the ablation area increases.
This can be abserved at around \qty{1.2}{\joule\per\cm\squared} in Fig.\,\ref{Fig:Results_3_lensGrowthRate}, which is therefore an estimate for the ablation threshold.

\begin{figure}
    \centering
    \includegraphics{3_pulseEnergy_growthrates.eps}
    \caption{Growth rates of samples from the 2nd batch, depending on laser fluence on the target surface.
    The data points are the mean of thicknesses of the four orientations, similar to Fig.\,\ref{Fig:Results_3_lensGrowthRate}.}
    \label{Fig:Results_3_pulseGrowthRate}
\end{figure}
        \subsubsection{\textit{c}-plane: Laser Spot Size Variation}
            %! theta omega
The \gls{oop}\ strain calculated via \eqref{Equ:Results_oop_strain_def} for all samples of the 1st batch is displayed in Fig.\,\ref{Fig:Results_3_lensStrain}.
Consider the \textit{c}-plane oriented samples of the 1st batch, that had a fixed lens position yielding a fluence of approx.\ \qty{2}{\joule\per\cm\squared}, but varying thickness (brown squares in Fig.\,\ref{Fig:Results_3_lensStrain}).
A clear dependence of the \gls{oop}\ strain can be observed: thinner samples yield higher strain.
The layers become relaxed for thicknesses above approx.\ \qty{170}{\nm}.
For low thicknesses, the strain approaches the predicted value for pseudomorphic growth of \cro\ on \ce{Al2O3}, which is \qty{3.90}{\percent} (cf.\ Tab.\,\ref{tab:d_strained}).
\begin{figure}
    \centering
    \includegraphics{3_lensPos_strain.eps}
    \caption{
        Out-of-plane strain calculated from \thetaomega-patterns for all samples from the 1st batch, depending on thickness and laser fluence (false color).
    }
    \label{Fig:Results_3_lensStrain}
\end{figure}
%! RSMs strain
The recorded \glspl{RSM} of the (02.10) reflection can confirm whether this observation of \gls{oop}\ strain is due to pseudomorphic growth.
In Fig.\,\ref{Fig:Results_3_cRSMs}, one can observe a shift of $q_\parallel^{(02.10)}$ to higher values for lower thicknesses.
This corresponds to a decrease of the \gls{ip} lattice constant, which is the expected behavior for pseudomorphic growth, because the \gls{ip} $a$ lattice constant of \textit{c}-oriented \ce{Al2O3} is \qty{0.2}{\angstrom} smaller than for \cro\ (cf.~Tab.\,\ref{Tab:sesquiLatticeConstants}).
The tensile \gls{oop}\ strain observed via \thetaomega-scans can also be confirmed by the fact that the \gls{oop}\ component $q_\perp^{(02.10)}$ is decreasing for thinner samples.
The reduction of signal intensitty is attributed to the thickness, but could also be a result of decreasing crystal quality (cf.~Fig.\,\ref{Fig:Results_3_lensOmega}).
\begin{figure}
    \centering
    \includegraphics{3_lensPos_RSM_c.png}
    \caption{
        \glspl{RSM} of the (02.10) reflection for several \textit{c}-plane oriented samples of the 1st batch with varying thickness.
        The reflection in the upper right corner represents the (02.10) reflection of the sapphire substrate.
    }
    \label{Fig:Results_3_cRSMs}
\end{figure}
When looking into the remaining samples that were fabricated with larger laser spot sizes but similar thickness (bluish squares in Fig.\,\ref{Fig:Results_3_lensStrain}), it becomes clear that the \gls{oop} strain is also slightly reduced for lower fluences.
But note that this effect is less dominant when compared to the influence of thickness.

%! omega
In Fig.\,\ref{Fig:Results_3_lensOmega}, the \textomega-FWHM is depicted depending on the film thickness and laser fluence for the 1st batch.
As before, consider the samples with smallest laser spot size (largest fluence) first:
increasing the thickness is clearly correlated to a decreasing \textomega-FWHM.
Therefore, thicker samples yield both less strained and more crystalline films.
Note that there is an outlier to this behavior for the sample with a thickness of approx.\ \qty{30}{\nm}.
When considering the \textomega-pattern (Fig.\,\ref{Fig:App_3_cOmegaOutlier}a), it becomes clear that the non-\textsc{Voigt} shape makes the determination of \gls{FWHM} difficult.
Therefore, not too much attention should be paid to this data point.
When considering the samples fabricated with lower fluences (bluish squares in Fig.\,\ref{Fig:Results_3_lensOmega}), a much more dominant influence of laser spot size on the crystallinity can be observed.
\begin{figure}
    \centering
    \includegraphics{3_lensPos_omega.eps}
    \caption{
        \textomega-FWHM for all samples from the 1st batch, depending on thickness and laser fluence (false color).
        The corresponding diffractograms are depicted in Fig.\,\ref{Fig:App_3_lens_omega}.
    }
    \label{Fig:Results_3_lensOmega}
\end{figure}
%! correlation
This can be summarzed by stating that the thickness of samples is the dominant influence on the \gls{oop} strain, because the thickest samplest yielded less strain than the thinner samples with lowest fluence (Fig.\,\ref{Fig:Results_3_lensStrain}).
However, for the \textomega-FWHM, it is the other way around, namely that even the thickest samples (which exhibit better quality than thinner samples of same lens position) have a much higher \textomega-FWHM when compared to thinner samples fabricated with less fluence.
This can be seen in Fig.\,\ref{Fig:Results_3_lensCorrelation}, where the \textomega-FWHM is visualized depending on the \gls{oop}\ strain of the corresponding sample:
A linear behavior (correlation) is observed for each set fluence; but there are two different regimes in total, with the high-fluence regime generally showing higher \textomega-FWHM.
\begin{figure}
    \centering
    \includegraphics{3_lensPos_strainOmegaCorrelation.eps}
    \caption{
        Correlation between strain and \textomega-FWHM for all samples from the 1st batch, depending on thickness and laser fluence (false color).
    }
    \label{Fig:Results_3_lensCorrelation}
\end{figure}
        \subsubsection{\textit{c}-plane: Pulse Energy Variation}
            %! Theta Omega
The \gls{oop}\ strain for the \textit{c}-plane oriented samples fabricated with various laser pulse energies, but constant laser spot size, are depicted in Fig.\,\ref{Fig:Results_3_pulseStrain}.
Note that there is still a distribution of thickness from \qtyrange{100}{200}{\nm}, even though the pulse number was adapted to the corresponding laser pulse energy.
The strain is overall smaller ($<\qty{2}{\percent}$) than for the 1st batch, because the 2nd batch contained samples with thickness $t>\qty{100}{\nm}$ which yields smaller strains as seen before.
No systematic dependence on the laser fluence is observed, which may be explained by the still remaining thickness distribution which overlaps the fluence variation.
This effect could be strong enough to overshadow the impact of laser pulse energy, as it was shown in the previous experiment that the thickness is the dominant factor for the \gls{oop}\ strain.
For example, note the sample fabricated with $F=\qty{1.5}{\J\per\cm\squared}$ (dark green square in Fig.\,\ref{Fig:Results_3_pulseStrain}), which exhibits the lowest strain, even though having higher fluence value than other samples.
This can be explained by the fact that with $t=\qty{200}{\nm}$, it is the thickest sample of the batch.
\begin{figure}
    \centering
    \includegraphics{3_pulseEnergy_strain.eps}
    \caption{
        Out-of-plane strain calculated from \thetaomega-patterns for all samples from the 2nd batch, depending on laser fluence and thickness (false color).
    }
    \label{Fig:Results_3_pulseStrain}
\end{figure}

%! Omega
In Fig.\,\ref{Fig:Results_3_pulseOmega}, the \textomega-FWHM is depicted depending on the laser fluence and film thickness for the 2nd batch.
The previously observed relation is confirmed: increasing fluences result in higher \textomega-FWHMs.
Namely, reducing the fluence by a factor of 2 results in a crystal quality improvement by one order of magnitude.
Note that for a fluence of approx.\ \qty{1}{\J\per\cm\squared}, two samples \texttt{A} and \texttt{B} with same thickness of \qty{150}{\nm} exhibit very different \textomega-FWHM of $\Delta\omega_\mathtt{A}=\qty{8}{\arcminute}$ and $\Delta\omega_\mathtt{B}=\qty{49}{\arcminute}$.
The \textomega-patterns are depicted in Fig.\,\ref{Fig:App_3_cOmegaOutlier}b.
Note that both diffractograms have \textsc{Voigt} shape, so the discrepancy may not be attributed to the determination of the \gls{FWHM}.
On the contrary, note that for the whole process \texttt{B}, a determination of FWHM was possible only for the \textit{c}-plane samples\footnote{
    This is why in Fig.\,\ref{Fig:Results_3_pulseOmega}, only the upper left \textit{c}-plane tile has two data points at $F\approx\qty{1}{\J\per\cm\squared}$.
}.
In Fig.\,\ref{Fig:App_3_w6930}, the \textomega-patterns of samples of all orientations from this process are depicted.
The non-\textsc{Voigt} shape for the orientations other than \textit{c}-plane as well as the unexpectedly high \textomega-FWHM for \textit{c}-plane sample indicate that the process yielded samples with poor crystal quality.
The origin of this observation is not entirely clear, but for some samples of this batch, the stepper motor causing the substrate rotation stopped during deposition, resulting in non-uniform deposition.
Whether this was the case for process \texttt{B} is not sure, but since both \texttt{A} and \texttt{B} were conducted with the same process parameters\footnote{
    The pulse number was varying, however, the growth rates $g_\mathtt{A}=\qty{3}{\pm\per\pulse}$ and $g_\mathtt{B}=\qty{3.75}{\pm\per\pulse}$ were quite similar.
},
something irregular must have been occured.

\begin{figure}
    \centering
    \includegraphics{3_pulseEnergy_omega.eps}
    \caption{
        \textomega-FWHM for all samples from the 2nd batch, depending on laser fluence and thickness (false color).
        The corresponding diffractograms are depicted in Fig.\,\ref{Fig:App_3_pulse_omega}. 
    }
    \label{Fig:Results_3_pulseOmega}
\end{figure}
        \subsubsection{\textit{r}-plane: Laser Spot Size Variation}
            In Fig.\,\ref{Fig:Results_3_lensStrain}, the \gls{oop}\ strain for the \textit{r}-plane samples fabricated with varying laser spot size is shown.
The overall strain is with less than \qty{1}{\percent} lower when compared to the \textit{c}-plane samples, exhibiting values up to \qty{4}{\percent} for thin samples.
In particular, the predicted value for \gls{oop}\ strain during pseudomorphic growth of \cro\ on \ce{Al2O3} of \qty{2.41}{\percent} is not reached (cf.\ Tab.\,\ref{tab:d_strained}).
As can be seen in a detailed view (Fig.\,\ref{Fig:App_3_lensStrain_zoomed}), the strain depends on the thickness:
it decreases from \qty{0.8}{\percent} to \qty{0.5}{\percent} for an increment of thickness from \qty{50}{\nm} to \qty{200}{\nm}.
This is in accordance to the bahavior observed for the \textit{c}-plane samples, albeit less pronounced.
Furthermore, for a fixed thickness, decreasing the fluence also results in less strained thin films, which is similar to the behavior of the \textit{c}-plane samples.
The \textomega-FWHM obtained from the (02.4) reflection is depicted in Fig.\,\ref{Fig:Results_3_lensOmega}.
Similar to the \textit{c}-plane samples -- but less pronounced--, increasing the thickness results in less mosaicity, which is also achieved by reducing the fluence.
Note that the overall \textomega-FWHM is between \qtylist{50;90}{\arcminute} which differs for the \textit{c}-plane samples, where a lower fluence yielded samples with $\Delta\omega<\qty{10}{\arcminute}$ (cf. Fig.\,\ref{Fig:Results_3_lensOmega}).
Therefore, increasing the thickness and reducing the fluence by varying laser spot position may increase the crystal quality, but not to an amount comparable to \textit{c}-plane oriented thin films.

        \subsubsection{\textit{r}-plane: Pulse Energy Variation}
            %! thetaomega strain
In Fig.\,\ref{Fig:Results_3_pulseStrain}, the \gls{oop}\ strain is depicted for varying laser pulse energies (2nd batch).
Independent of thickness, the fluence determines the strain of the thin films.
The overall strain is below \qty{0.4}{\percent}, and thereby comparable to the samples obtained from processes in the 1st batch with larger laser spot sizes.
%! RSM: strain
\begin{figure}
    \centering
    \includegraphics{3_pulseEnergy_completeStrain_r.eps}
    \caption{
        In-plane and out-of-plane strain for the \textit{r}-plane samples from the 2nd batch, calculated from the peak positions of the \glspl{RSM} described in section \ref{Sec:Results_3_Experiment} (left).
        Tilt along the $x$-axis (purple ordinate) and $y$-axis (blue ordinate), determined from symmetric reflections (right).
        Note the different scaling of the ordinates, indicating less tilt along the $y$-axis.
    }
    \label{Fig:Results_3_pulse_r_StrainTilt}
\end{figure}
A detailed view on the strain for those samples is given in Fig.\,\ref{Fig:Results_3_pulse_r_StrainTilt} which is based on the evaluation of \glspl{RSM} that were performed as described in section \ref{Sec:Results_3_Experiment}.
The \gls{oop}\ strain was calculated from both asymmetric (\textcolor{col-brightGreen}{$\blacktriangle$}) and symmetric (\textcolor{col-brightOrange}{$\blacksquare$}) reflections.
The latter is equivalent to the calculation from the peak position in \thetaomega\ diffraction patterns.
It can be observed that the increasing tensile \gls{oop}\ strain comes along with an increasing \gls{ip}\ compressive strain (\textcolor{col-purple}{$\blacktriangleleft$}), ranging from \qty{-0.2}{\percent} to \qty{-0.8}{\percent}.
Therefore, the \gls{oop} strain may be attributed to a partial pseudomorphic growth mode, because the \ce{Al2O3} lattice constants are smaller than the ones for \cro.
The compressive strain is then due to an aligning of in-plane lattice constants.

Note that the values for \gls{oop}\ strain obtained from \thetaomega-scans (cf.\ Fig.\,\ref{Fig:Results_3_pulseStrain}) are only qualitatively confirmed:
the strain measured from the symmetric \gls{RSM} is approx.\ 0.2 percentage points below the value obtained from \thetaomega-scans.
A comparison of both methods is given in Fig.\,\ref{Fig:Results_3_r_strainDiscrepancy}, where both a \thetaomega\ pattern and a symmetric \acrshort{RSM} of the (02.4) reflection are depicted for one sample ($F=\qty{1.1}{\J\per\cm\squared}$), as well as the calculated strain for all samples with different laser pulse energy.
\begin{figure}[ht]
    \centering
    \includegraphics{3_misc_pulse_r_discrepancy.png}
    \caption{
        Out-of-plane strain for the \textit{r}-plane samples fabricated with varying laser pulse energy (right).
        For the sample with $F=\qty{1.1}{\J\per\cm\squared}$, the \thetaomega\ pattern (top left) and \gls{RSM} (bottom left) are depicted.
        The black diamond ($\blacklozenge$) marks the position of the (02.4) reflection in the 1D \thetaomega\ pattern and 2D \gls{RSM}.
    }
    \label{Fig:Results_3_r_strainDiscrepancy}
\end{figure}
The origin of this discrepancy may lie in on of the corrections that was applied the \glspl{RSM}, but not to the \thetaomega\ patterns.
However, the correction of the substrate peak position in 2D reciprocal space (rotation and stretching, cf.\ section \ref{Sec:Methods_RSM}) corresponds to a shift of the whole 1D \thetaomega\ pattern to match the substrate peak.
The latter was done for the evaluation of \thetaomega-scans on which Fig.\,\ref{Fig:Results_3_pulseStrain} is based.
But the correction of thin film tilt which is done for \glspl{RSM} was not done for the \thetaomega-scans.
This can be seen in Fig.\,\ref{Fig:Results_3_r_strainDiscrepancy}, where the \thetaomega\ pattern corresponds to a line with $q_\parallel=\mathrm{const.}=0$ in the reciprocal space.
The peak on this line has the same $q_\perp$ coordinate as the RSM peak not corrected to thin film tilt (both visualized as black diamonds, $\blacklozenge$).
But if the (02.4) peak is rotated by the value of thin film tilt (counterclockwise), the $q_\perp$ component slightly increases.
Therefore it follows that
$$q_\perp^\mathrm{RSM, corrected}
>q_\perp^\mathrm{RSM, uncorrected}
=q_\perp^{2\theta\mathrm{-}\omega}\,,$$
which results in a smaller \gls{oop}\ lattice constant obtained from RSMs.
Therefore, the \gls{oop}\ strain is smaller, when determined from symmetric RSMs.
Note that a \textomega-optimization prior to a \thetaomega-scan is done for correcting a tilt of the \textit{substrate}, which is different from the correction of thin film tilt.

But even though this is a significant difference in evaluation between \thetaomega-patterns and \glspl{RSM}, the discrepancy between both methods does not change significantly when the thin film tilt decreases from \qty{40}{\arcminute} to \qty{10}{\arcminute}.
So further analysis has to be done for the applied evaluation methods.
In general it has to be noted that the precision of the \gls{oop}\ strain obtained from \glspl{RSM} depends on (i) the peak position of the reflection, (ii) the peak position of the corresponding substrate peak (for substrate correction) and (iii) the peak position of the asymmetric peaks (for shear correction, not done for \textit{r}-plane).
These values are subject to a certain amount of uncertainity, which results in an ill-defined error.
% Those positions were not obtained by fitting a 2D \textsc{Voigt} profile to the \glspl{RSM}, but by reading the peak position \enquote{by hand}.
% This may result in an undefined error.

{\sloppy % fix problems with the word "squares"
Another observation is that the \gls{oop}\ strain obtained from symmetric (\textcolor{col-brightOrange}{$\blacksquare$}) and asymmetric reflections (\textcolor{col-brightGreen}{$\blacktriangle$}) aligns for the two samples fabricated with higher fluences only (cf.\ Fig.\,\ref{Fig:Results_3_pulse_r_StrainTilt}).
The discrepancy observed for the lower fluences is unexpected.
In Fig.\,\ref{Fig:Res_3_RSMs_r}, all symmetric and asymmetric RSMs are displayed.
For the samples fabricated with \qtylist{650;800}{\milli\J}, accounting for the thin film tilt will result in a counterclockwise rotation of the reciprocal space.
Therefore, the observed (03.0) reflection (\textcolor{red}{$\blacksquare$}) has a smaller $q_\perp$ component compared to the predicted ($\blacklozenge$) peak position (after rotation).
This results in tensile (positive) strain -- which is in accordance with the values obtained from both symmetric RSM reflections and \thetaomega\ patterns.
For the samples with \qtylist{300;450}{\milli\J}, however, the thin film tilt is sufficiently small to result in compressive (negative) strain, i.e.\ the rotation of reciprocal space does not result in a smaller out-of-plane component $q_\perp$ of the thin film (\textcolor{red}{$\blacksquare$}) compared to the bulk value ($\blacklozenge$).
Note that the definition of positive or negative strain is strongly depending on what is defined as bulk value.
Because in Fig.\,\ref{Fig:Results_3_pulse_r_StrainTilt}a, in- and out-of-plane strain align for low fluences, it is possible that the assumed bulk constant is too large, and the observed compressive strain of \qty{-0.2}{\percent} actually corresponds to an unstrained thin film.
% and is probably due to an error in evaluation of the \glspl{RSM}.
% This is supported by the fact that for those two data points, the strain obtained from asymmetric reflections is almost exactly mirroring the value obtained from the symmetric reflections.
% 
\par}
\begin{figure}
    \centering
    \includegraphics{3_misc_r_RSMs.png}
    \caption{
        Reciprocal space maps of four \textit{r}-plane oriented thin films fabricated with different laser pulse energy.
        The probed reflections are symmetric (02.4) (left) and asymmetric (30.0) (right).
        The peak with larger $q_\perp$ component corresponds to the substrate.
        For the RSMs of the asymmetric reflections, the expected peak position ($\blacklozenge$) as well as the observed peak positions (\textcolor{red}{$\blacksquare$}) are indicated.
        Note that the RSMs are already corrected such that the substrate peak aligns with the expected position.
        A thin film tilt is indicated by the nonzero in-plane component of the symmetric (02.4) reflection.
        For determination of thin film lattice constants, the RSM is rotated by the observed thin film tilt, which is not visualized in these images.
    }
    \label{Fig:Res_3_RSMs_r}
\end{figure}

%! RSM: tilt
As predicted by \textcite{grundmann2020b}, partially relaxed \textit{r}-plane thin films should exhibit a tilt of the thin film with respect to the substrate.
This tilt is indeed observed along the $x$-axis for all values of fluence, ranging from approx.\ \qty{10}{\arcminute} to \qty{40}{\arcminute} (\textcolor{col-purple}{$\blacktriangleleft$} in Fig.\,\ref{Fig:Results_3_pulse_r_StrainTilt}).
A corresponding tilt along the $y$-axis is not observed: there, the tilt angles are two orders of magnitude lower and below \qty{0.4}{\arcminute} (\textcolor{blue}{$\blacktriangleright$}).
This is in agreement with elasticity theory which predicts a tilt only along the $x$-axis, because the prismatic slip systems responsible for relaxation along the $y$-axis yield tilt components of the \gls{bv} that cancel out on average (cf.\ section \ref{Sec:Theory_Relaxed}).
But note that the thin film tilt increases for higher fluences, which also results in a higher \gls{oop}\ strain.
This observation is unexpected, because the thin film tilt is a result of \emph{relaxation}, whereas strain is a result of partial \emph{pseudomorphic} growth.
So according to strain, higher fluences result in less relaxed layers -- according to tilt, higher fluences result in more relaxed layers.
This result indicates that an interplay of both processes is present and that for growth modes that exhibit only partially relaxed behavior, more sophisticated models for the relaxation mechanism must be applied.

%! omega
The \textomega-FWHM of the \textit{r}-plane samples of the 2nd batch is approx.\ \qty{50}{\arcminute} and has no significant dependence on both fluence or thickness (Fig.\,\ref{Fig:Results_3_pulseOmega}).
This confirms the previously obtained result for the samples fabricated with varying laser spot sizes.
        \subsubsection{\textit{m}- and \textit{a}-plane: Laser Spot Size Variation}
            %! theta-omega strain
In Fig.\,\ref{Fig:Results_3_lensStrain}, the \gls{oop}\ strain for the \textit{m}- and \textit{a}-plane oriented samples of the 1st batch (laser spot size variation) is depicted.
The maximum strain which is reached for high fluences is approx.\ \qtylist{0.8;1.5}{\percent} for \textit{m}- and \textit{a}-plane, respectively.
Those values are far below the predicted values for pseudomorphic growth, which are \qty{3.67}{\percent} (\textit{m}-plane) and \qty{3.63}{\percent} (\textit{a}-plane).
This indicates relaxed growth.
In Fig.\,\ref{Fig:App_3_lensStrain_zoomed} it can be seen that for higher thicknesses, the strain reduces only very slightly.
The \textit{m}-plane outlier at a thickness of \qty{25}{\nm} can be explaind by the very low peak intensity of the (30.0) reflection in the \thetaomega\ pattern, which causes a larger uncertainty for this value.
Overall, the fluence is the determining parameter for the strain, allowing strain values of down to \qty{0}{\percent} for \textit{a}-plane samples.

%! omega
In Fig.\,\ref{Fig:Results_3_lensOmega}, the \textomega-FWHM for both \textit{m}- and \textit{a}-plane samples is depicted.
For \textit{m}-plane, the \textomega-FWHM is approx.\ \qty{50}{\arcminute} for all fluences and thicknesses -- only a small decrease for higher thicknesses is oberserved.
As a result, in Fig.\,\ref{Fig:Results_3_lensCorrelation}, two regimes of high and low fluence can be distinguished, where each regime itself comes with a correlation indicating better crystallinity with less strain.
However, alltogether, a slight negative correlation can be observed, i.e.\ better crystallinity comes at the cost of higher strain.
For \textit{a}-plane samples, no significant dependence on fluence can be observed.
On the contrary, there seems to be an increase in \textomega-FWHM for increasing thicknesses up to approx.\ \qty{100}{\nm}.
This behavior differs from all other orientations observed and could be attributed to an unusual shape of the \textomega-patterns.
In Fig.\,\ref{Fig:Results_3_lens_a-weirdOmega}, such a pattern is depicted and has clearly no \textsc{Voigt}-shape.
Rather, the pattern consists of an exponential tail (linear in logarithmic intensity axis) for about \qty{1.5}{\degree} and a very sharp 2nd peak with a small \gls{FWHM} on top of it.
This shape is observed for almost every \textit{a}-plane sample, as can be seen by the various diffractograms shown in Fig.\,\ref{Fig:App_3_lens_omega}.
The sharp peak is located at the maximum of the underlying broader peak and can therefore not be attributed to \ce{Al2O3} or another phase of \cro, because then it would not shift together with the (22.0) peak of the \textalpha-phase of \cro.
This is also supported by the fact that no anomaly is observed in the \thetaomega-patterns (not shown).
This behavior has previously been observed for \ce{ZnO} thin films grown on \textit{c}-plane silicon
    \cite{cho2005,durand2011}.
There, the broader peak contribution is attributed to a degradation at the interface region
    \cite{cho2005}
and the sharp peak is present for a weakly disordered film, where the misfit to the substrate is damped through the underlying film.
As shown by \textcite{durand2011}, the \textomega-FWHM of the narrow peak is limited by the instrumental resolution -- they could resolve a broadening of 5 arcseconds for this peak.
This behavior has also been observed for \textit{c}-plane oriented \cro\ thin films fabricated by \acrshort{pld}
    \cite{arca2017}.
In Fig.\,\ref{Fig:Results_3_lens_a-weirdOmega}b, several \textomega-patterns are depicted for thin films of different thickness.
For a layer thickness of \qty{30}{\nm}, a broad peak with an overlaying higher intensity peak is observable, which is not narrow but exhibits a plateau with a width of approx.\,\qty{0.25}{\degree}.
For increasing film thicknesses, the FWHM of both the broader and narrow peak decrease, confirming the growth of a higher crystalline layer.
Henceforth, the crystallinity of \textit{a}-plane \cro\ thin films should be assessed by a two-layer model when determining the FWHM of the corresponding \textomega\ patterns.

% Note that the diffractograms stem from different samples, which were also fabricated with different laser fluences, which makes the comparison difficult and the origin of the 2nd peak not entirely clear.
% This could explain the broad spread of \textomega-FWHM (\qty{15}{\arcminute} to \qty{80}{\arcminute}) for \textit{a}-plane samples as well as that the \textomega-FWHM follows a different relation to fluence and thickness when compared to the other orientations.
\begin{figure}
    \centering
    \begin{tabular}{cc}
        \multicolumn{1}{l}{\textbf{(a)}}
        & \multicolumn{1}{l}{\textbf{(b)}} \figSpace \\
        \includegraphics{3_misc_lens_a_weirdOmega.eps}
        & \includegraphics{3_misc_lens_a_weirdOmega_thickness.eps}
    \end{tabular}
    \caption{
        \textomega-pattern of \textit{a}-plane samples from the 1st batch: \textbf{(a)} a sample with linear representation and \textbf{(b)} samples with varying thickness in logarithmic representation.
    }
    \label{Fig:Results_3_lens_a-weirdOmega}
\end{figure}

        \subsubsection{\textit{m}- and \textit{a}-plane: Pulse Energy Variation}
            In Fig.\,\ref{Fig:Results_3_pulseStrain}, the \gls{oop}\ strain for the \textit{m}- and \textit{a}-plane oriented samples of the 1st batch is depicted.
The strain ranges from
    \qty{0.15}{\percent} (\qty{0.3}{\percent})
to
    \qty{0.45}{\percent} (\qty{1.1}{\percent})
for \textit{m}-plane (\textit{a}-plane) samples.
A clear dependence on fluence can be observed, whereby the thickness has no influence at all.
%! m-plane strain
The complete in- and out-of-plane strains for \textit{m}-plane samples are depicted in Fig.\,\ref{Fig:Results_3_pulse_ma_strainTilt}a.
With increasing \gls{oop}\ strain, the \gls{ip}\ strain also increases, which indicates a pseudomorphic growth mode.
Note that the \gls{ip} strain is, in the range of uncertainity, the same along both $x$- and $y$-direction, even though both axes are not equivalent.
As for \textit{r}-plane samples (cf.\ Fig.\,\ref{Fig:Results_3_pulse_r_StrainTilt}), the \gls{oop}\ strain is systematically smaller by 0.15 percentage points, compared to the values obtained from peak positions in \thetaomega-patterns (Fig.\,\ref{Fig:Results_3_pulseStrain}).
\thetaomega-scans probe for symmetric reflections only, which is why shear stresses cannot be corrected by this method.
However, those strain angles are rather small (cf.\ Fig.\,\ref{Fig:Results_3_pulse_ma_strainTilt}a), which is why it is unplausible that this is the origin of the discrepancy.
Moreover, as discussed for the \textit{r}-plane samples, the thin film tilt is suspected to be too small to yield this effect.
\begin{figure}
    \centering
    \begin{tabular}{c}
        \multicolumn{1}{l}{\textbf{(a)}} \figSpace \\
        \includegraphics{3_pulseEnergy_completeStrain_m.eps} \figSpace \\
        \multicolumn{1}{l}{\textbf{(b)}} \figSpace \\
        \includegraphics{3_pulseEnergy_completeStrain_a.eps}
        
    \end{tabular}
    
    \caption{Left: In-plane (along $x$- and $y$-axis) and out-of-plane strain for the \textbf{(a)} \textit{m}-oriented and \textbf{(b)} \textit{a}-oriented samples from the 2nd batch, calculated from the peak positions of the \glspl{RSM} described in~\ref{Sec:Results_3_Experiment}.
    Right: Shear (circles) and tilt (triangles) of thin films determined by asymmetric and symmetric reflections, respectively.
    The values were determined along $x$-axis and $y$-axis.
    This is represented by the purple and blue ordinates in \textbf{(a)}, respectively.
    Note the different scaling of the ordinates, indicating less tilt along the $y$-axis.
    Due to the same scale of tilt along $x$- and $y$-axis for \textbf{(b)}, only one ordinate is used to represent the data.
    }
    \label{Fig:Results_3_pulse_ma_strainTilt}
\end{figure}

%! m-plane tilt/shear
As predicted by \textcite{kneiss2021}, a significant tilt of the thin film is observed along the $x$-direction (purple triangles in Fig.\,\ref{Fig:Results_3_pulse_ma_strainTilt}b), which ranges from \qty{20}{\arcminute} to \qty{40}{\arcminute} and increases with higher fluences.
Furthermore, a small shear of up to \qty{2}{\arcminute} is observed along this axis.
On the contrary, a thin film tilt (blue triangles) below \qty{1}{\arcminute} and a small shear tilt are observed in $y$-direction.
This is also in accordance with the predicted slip systems (cf.~\ref{Sec:Theory_Relaxed}), which should result in no net tilt along the $y$-axis.
However, as it is the case for the \textit{r}-oriented samples, this thin film tilt -- acting as an indicator for relaxation -- increases with higher fluences.
But this objects to the observation of decreasing relaxation with higher fluences due to increasing \gls{ip}\ and \gls{oop}\ strain. 
Again, this draws to the conclusion that a more sopisticated description of partially relaxed layers is in need.

%! a-plane strain
In a qualititive sense, a similar behavior for \textit{a}-plane samples is observed when investigating both in- and out-of-plane strain (Fig.\,\ref{Fig:Results_3_pulse_ma_strainTilt}b).
However, the \gls{ip}\ strain is significantly larger along the $y$-direction for layers with less strain:
for $F=\qty{0.7}{\J\per\cm\squared}$, the thin film is 0.66 percentage points more strained along the $y$-axis.
This discrepancy reduces for thin films that are more stressed in total.
Furthermore, as for the other orientations, the \gls{oop}\ strain is systematically lower when compared to the values obtained from \thetaomega-patterns, namely by 0.15 percentage points.
The reasons for this effect are presumably the same as for the \textit{m}-plane oriented samples.
%! a-plane tilt / shear
The shear and tilt angles for \textit{a}-plane samples are depicted in Fig.\,\ref{Fig:Results_3_pulse_ma_strainTilt}b.
As predicted by \textcite{kneiss2021}, no tilt is observed along both the $x$-axis ($<\qty{1.5}{\arcminute}$) and $y$-axis ($<\qty{0.7}{\arcminute}$).
A small shear angle of around \qty{1.5}{\arcminute} is observed along the $y$-axis.

%! Omega
Finally, the mosaicity for both \textit{m}- and \textit{a}-plane samples is depicted in Fig.\,\ref{Fig:Results_3_pulseOmega}.
For \textit{m}-plane oriented samples, the \textomega-FWHM is approx.\ \qty{40}{\arcminute} and decreases slightly with higher fluences, which is in contrast to the increasing strain.
For \textit{a}-plane samples, a large spread similar to the samples of the 1st batch (cf.\ Fig.\,\ref{Fig:Results_3_lensOmega}) is observed.
But, again, this observation is probably due to the specific shape of \textomega-patterns for the \textit{a}-plane samples.
In Fig.\,\ref{Fig:App_3_pulse_omega}, the diffractograms are depicted and a severe deviation from the \textsc{Voigt} shape can be identified.
Therefore, determination of the \gls{FWHM} is hindered and the large spread of values explained.
\subsection{Conclusion}
    The out-of-plane strain and in-plane strain of \cro\ thin films with different crystal orientations were investigated depending on the laser fluence on the target during deposition via \gls{pld}.
The variation in laser fluence was achieved by either increasing the laser spot size or by decreasing the laser pulse energy.
Samples with \textit{c}-orientation grow fully pseudomorphic for low film thicknesses, wheras the samples with \textit{r}-, \textit{m}- and \textit{a}-orientatiin are partially relaxed.
The thickness is a crucial parameter for \textit{c}- and \textit{r}-plane samples, whereas the laser fluence on the target strongly influences the crystal structure of \textit{m}- and \textit{a}-plane samples.
It can be concluded that the thickness of the thin films is more relevant for orientations that have more out-of-plane \textit{c}-axis component.
For \textit{c}- and \textit{r}-orientation, less laser fluence -- no matter whether via larger laser spot sizes or reduced laser pulse energy -- results in less \gls{FWHM} in \textomega-patterns and thus better crystallinity.
For \textit{m}-plane samples, a reversed behavior is observed -- however, the dependence on fluence is much less pronounced.
A layered structure of \textit{a}-plane thin films hardens the comparison of the \textomega-FWHM for this orientaiton.
Furthermore, for \textit{r}-, \textit{m}- and \textit{a}-plane oriented samples, thin film tilts have been observed in the directions that were previously observed for relaxed \agao\ layers on \alo\ 
    \cite{grundmann2020b,kneiss2021}
(cf.\ Tab.\,\ref{tab:d_strained}b).

To understand the dependence between relaxation, crystallinity and thin film tilt for \cro\ \textit{m}-plane thin films, a comparison to similarly structured \agao\ can be made:
In \textcite{kneiss2021}, fully relaxed \textit{m}-plane \textalpha-\ce{(Al_xGa_{1-x})2O3} thin films exhibited a tilt that was dependent on the aluminum content, and therefore on relaxed lattice constants.
For pure \agao\ layers (zero aluminum content), a thin film tilt of \qty{36}{\arcminute} was observed, which is in accordance to the here observed angles of \qty{20}{\arcminute} to \qty{40}{\arcminute} (cf.\ Fig.\,\ref{Fig:Results_3_pulse_ma_strainTilt}a).
In this work, however, due to variations in laser fluence, partially relaxed \textit{m}-plane \cro\ thin films with out-of-plane strain ranging from \qty{0.02}{\percent} to \qty{0.3}{\percent}\footnote{
    According the RSMs. Note that there was a discrepancy between the strain values obtained from \thetaomega\ scans and RSMs.
} could be fabricated (Fig.\,\ref{Fig:Results_3_pulse_ma_strainTilt}a).
It was observed that samples with less out-of-plane strain -- i.e.\ smaller out-of-plane lattice constants -- exhibit smaller thin film tilts.
Note that due to the fact that \alo\ has smaller lattice constants than relaxed \cro, a decrease in lattice constants is directly related to a decreasing mismatch between substrate and thin film.
Therefore, the here reported results are in accordance to \textcite{kneiss2021} in the sense that a reduction in lattice mismatch reduces the thin film tilt.
Furthermore, reducing the laser fluence also results in a slightly higher \textomega-FWHM (Fig.\,\ref{Fig:Results_3_pulseOmega}), which could correspond to an increased dislocation density.
When applying a heteroepitaxial model as displayed in Fig.\,\ref{fig:Theory_tiltDislocation}
    \cite{grundmann2016},
this reduced dislocation density results in less thin film tilt, which is indeed observed.
Therefore, an increasing laser fluence on the PLD target results in both increasing strain and decreasing \textomega-FWHM which manifest in larger thin film tilts.
These effects can be summarized as less partially relaxed layers.

However, note that the importance of reduced laser fluence for more relaxed thin films is not explained yet.
By taking the reduced kinetic energy of the plasma species into account
    \cite{anisimov1996},
one can infer that the growth dynamics are altered in such a way that the formation of dislocations is favored.
In general, note that more sophisticated methods like \gls{TEM} should be applied to get a more detailed view into the formation and density of dislocations, which are fundamental to relaxation and thin film tilt.

No investigations of the influence of laser fluence on the structural properties of \cro\ thin films fabricated by PLD have been done so far in the literature.
The only report on varying laser fluence from \qtyrange{1.6}{3.7}{\J\per\cm\squared} has been concerning the atomic ratio of \ce{Cr} cations to \ce{O} anions for deposition on silicon
    \cite{tabbal2006}.
Most of the \cro\ thin films fabricated via PLD were deposited on \textit{c}-plane sapphire
    \cite{singh2019,arca2017,kehoe2016}.
None of those studies could identify the laser fluence as a crucial paramter influencing the crystallinity of the thin films (Fig.\,\ref{Fig:Results_3_pulseOmega}):
the best \textomega-FWHM of \qty{22}{\arcminute} that was reported by \textcite{singh2019} is larger than the here reported value of \qty{7}{\arcminute}.
For \textit{r}-plane oriented thin films reported by \textcite{punugupati2015}, an out-of-plane strain of \qty{0.57}{\percent} could be identified, but no studies have been performed on the origin of this observation.

To determine the optimal deposition parameters of \cro, one has to take the different orientations into account:
Because the thickness is mostly relevant for \textit{c}-plane oriented thin films, the choice of at least $t=\qty{150}{\nm}$ results in the lowest strain and \textomega-FWHM.
Furthermore, a laser spot size of \qty{10}{\mm\squared} and pulse energy of \qty{300}{\milli\J} would result in the lowest strain for \textit{r}-, \textit{m}- and \textit{a}-orientation.
Note that this, however, results in less crystalline films for those orientations (Fig.\,\ref{Fig:Results_3_pulseOmega}), as well as a very low growth rate.
Therefore, a pulse energy of \qty{450}{\milli\J} is chosen for future depositions due to the best crystal quality while maintaining strain of about \qty{0.1}{\percent} and \qty{0.3}{\percent} for \textit{m}- and \textit{a}-plane respectively.
Those deposition parameters also result in low thin film tilts of about \qty{15}{\arcminute} and \qty{25}{\arcminute} for \textit{r}- and \textit{m}-plane samples, respectively.
The laser fluence on the target is therefore \qty{1.1}{\J\per\cm\squared}, and will be applied when depositing high quality buffer layers for \agao\ thin films:
The reduced compressive in-plane strain is desired to achieve low mismatch between the \cro\ and \agao\ layer, as demonstrated in the following chapter.

% In \textcite{kneiss2021}, a relaxation paramter $\rho_x$ was successfully introduced to quantify the amount of partial relaxation along the $x$-axis for \textit{m}- and \textit{a}-plane oriented \textalpha-\ce{(Al_xGa_{1-x})2O3} thin films.
% Higher values of $\rho_x$ correspond to more pseudomorphic growth and were correlated to smaller tilt angles.
% However, the results here object to this observation:

% The explanation of the thin film tilt by the relaxation model does not seem to be the origin of this discrepancy:
% The axis of thin film tilt was correctly predicted to be only along $x$-direction for \textit{m}- and \textit{r}-plane thin films and along no direction for \textit{a}-plane samples.
% Nevertheless, the result of this observation is that no trade-off has to be made between strain and thin film tilt -- a lens position of \qty{-1}{\cm} and laser pulse energy of \qty{450}{\mJ} yields almost no strain and low thin film tilt while still resulting in a growth rate of up to \qty{4}{\pm\per\pulse}.