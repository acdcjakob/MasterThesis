The out-of-plane strain and in-plane strain of \cro\ thin films of different crystal orientation were investigated depending on the laser fluence on the target during deposition via \gls{pld}.
Samples with \textit{c}-orientation grow fully pseudomorphic for low film thicknesses, wheras the samples of different orientation only show partial pseudomorphic growth.
The thickness is for \textit{c}- and \textit{r}-plane samples a crucial parameter, whereas the laser fluence on the target strongly influences the crystal structure of \textit{m}- and \textit{a}-plane samples.
This indicates that the thickness of the thin films is more relevant for orientations that have more out-of-plane \textit{c}-axis component, because for the prismatic orientations, the \textit{c}-axis is completely in-plane.
Overall, less laser fluences -- no matter whether via larger laser spot sizes or reduced laser pulse energy -- result in less \gls{FWHM} in \textomega-patterns.
For \textit{r}-, \textit{m}- and \textit{a}-plane oriented samples, thin film tilts has been observed in the directions that were predicted by theoretical calculations
    \cite{grundmann2020b,kneiss2021}
(cf.\ Tab.\,\ref{tab:d_strained}b).

In \textcite{kneiss2021}, a relaxation paramter $\rho_x$ was successfully introduced to quantify the amount of partial relaxation along the $x$-axis for \textit{m}- and \textit{a}-plane oriented \textalpha-\ce{(Al_xGa_{1-x})2O3} thin films.
Higher values of $\rho_x$ correspond to more pseudomorphic growth and were correlated to smaller tilt angles.
However, the results for \cro\ object to this observation:
an increasing thin film tilt is correlated to an \emph{increasing} in- and out-of-plane strain (cf.\ Fig.\,\ref{Fig:Results_3_pulse_ma_strainTilt}), i.e.\ to more pseudomorphic growth.
The explanation of the thin film tilt by the relaxation model does not seem to be the origin of this false prediction:
The axis of thin film tilt was correctly predicted to be only along $x$-direction for \textit{m}- and \textit{r}-plane thin films and along no direction at all for \textit{a}-plane samples!
Nevertheless, the result of this observation is that no trade-off has to be made between strain and thin film tilt -- a lens position of \qty{-1}{\cm} and laser pulse energy of \qty{450}{\mJ} yields almost no strain and low thin film tilt while still resulting in a growth rate of up to \qty{4}{\pm\per\pulse}.