\glspl{tco} are materials that combine the properties of having low absorption coefficient in the visible spectrum and being conductive at the same time
    % \cite{kehoe2016}.
    \cite{ginley2011}.
The interest in these materials is motivated by possible applications in portable and flexible electronics
    \cite{guillen2011},
displays
    \cite{singh2024},
solar cells
    \cite{chavan2023}
and more
    \cite{ginley2011}.
% Due to the restriction on only a few materials in the industry (e.g. \ce{SnO2} and \ce{In2O3}), investigations of new materials are required
%     \cite{ginley2011}.
Not only the focus on a small number of materials in the past (\ce{SnO2}, \ce{In2O3}, \ce{ZnO})
    \cite{ginley2011},
but also the scarcity and concerns about availability\footnote{
    China controls \qty{75}{\percent} of the world's indium reserves, and limited the export of this material already in the past \cite{candelise2011}.
} of fundamental compounds
    \cite{candelise2011}
increases the demand for new materials.
This includes fabrication of \textit{p}-type \glspl{tco} as well as compounds with even larger band gaps than \qty{3}{eV}, called \gls{uwbg} materials.
Because the critical electrical field -- at which breakdown occurs -- is depending on the band gap
    \cite{slobodyan2022},
\gls{uwbg} materials can serve for high-power electronic devices as well as for deep \gls{UV} optoelectronics
    \cite{wong2021}.
Candidates for this material class are group-III sesquioxides\footnote{
    A sesquioxide is an oxide with formula unit \ce{Me2O3}, where \ce{Me} is a metal with oxidation state $+3$.
    Transition metal sesquioxides are, e.g., \ce{Y2O3}, \ce{Rh2O3} or \ce{In2O3}.
},
of which \ce{Ga2O3} with its several polymorphs
    \cite{hassa2021a}
recently gained interest in the scientific community -- in particular the metastable corundum structured \agao.
% , even though its deposition has to account for parasitic growth of the thermodynamically more stable \textbeta-phase \cite{petersen2023}.
% At this point, \ce{Cr2O3} comes in handy

The sesquioxide \cro, being a possible \textit{p}-type \gls{tco}, is isomorphic to group-III sesquioxide \agao{} with quite similar lattice parameters (cf.\ Tab.\,\ref{Tab:sesquiLatticeConstants}).
This enables the use of \ce{Cr2O3} as a buffer layer between \agao{} and isomorphic \textalpha-\ce{Al2O3} (sapphire) substrates to improve the deposition process
    \cite{stepanov2021}.
The lower band gap also makes band gap engineering with isostructural sesquioxides possible
    \cite{kaneko2010}
and finally, \ce{Cr2O3} exhibits increased conductivity upon doping
    \cite{uekawa1996}
and could thus serve as \textit{p}-type component in a \textit{p}-\textit{n}-heterojunction with \agao{}.
In the following, an overview of the two mentioned sesquioxides will be presented with focus on the physical properties being relevant to this work.