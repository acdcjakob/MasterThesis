%! RHEED
The RHEED patterns of all samples from the \textit{ex situ} and \textit{in situ} batch are depicted in Fig.\,\ref{Fig:Results_4_RHEED}a and Fig.\,\ref{Fig:Results_4_RHEED}b, respectively.
From the periodic patterns it can be concluded that every surface is crystalline.
Note that the mere observation of crystallinity does not indicate which phase of \gao\ is present on the samples.
\begin{figure}
    \centering
    \begin{tabular}{llll}
        \multicolumn{4}{l}{\textbf{(a)}} \figSpace \\
        \textit{c}-plane & \textit{r}-plane & \textit{m}-plane & \textit{a}-plane \\        
        \includegraphics[width=.22\textwidth]{w6957c.jpg}
        & \includegraphics[width=.22\textwidth]{w6957r.jpg}
        & \includegraphics[width=.22\textwidth]{w6957m.jpg}
        & \includegraphics[width=.22\textwidth]{w6957a.jpg} \figSpace \\
        \multicolumn{4}{l}{\textbf{(b)}} \figSpace \\
        \textit{c}-plane & \textit{r}-plane & \textit{m}-plane & \textit{a}-plane\\
        \includegraphics[width=.22\textwidth]{w6959c.jpg}
        & \includegraphics[width=.22\textwidth]{w6959r.jpg}
        & \includegraphics[width=.22\textwidth]{w6959m.jpg}
        & \includegraphics[width=.22\textwidth]{w6959a.jpg}
    \end{tabular}
    \caption{\gls{RHEED} patterns of the \agao\ surface deposited \textbf{(a)} \textit{ex situ} on \cro\ and \textbf{(b)} \textit{in situ} on \cro.
    Note that the patterns were recorded in arbitrary azimuth which is why they may differ between \textit{ex situ} and \textit{in situ} deposition.
    This is not necessarily a result of different crystal structure.
    }
    \label{Fig:Results_4_RHEED}
\end{figure}

%! c-plane
\begin{figure}
    \centering
    \begin{tabular}{c}
        \multicolumn{1}{l}{\textbf{(a)}}
        \figSpace \\
        \includegraphics{4_thetaomega_c.pdf}
        \figSpace \\
        \multicolumn{1}{l}{\textbf{(b)}}
        \figSpace \\
        \includegraphics{4_RSM_c.png}
    \end{tabular}
    \caption{
        \textbf{(a)}
        \thetaomega-patterns of the \textit{c}-plane \cro\ reference sample \texttt{A} (black), as well as the \textit{ex situ} (blue) and \textit{in situ} (red) buffer layer structures.
        \textbf{(b)}~\gls{RSM} around the (00.6) reflection recorded before (left) and after (right) \textit{ex situ} deposition of \gao\ on \cro.
    }
    \label{Fig:Results_4_buffer_c}
\end{figure}
In Fig.\,\ref{Fig:Results_4_buffer_c}a, \thetaomega\ patterns are depicted of the \cro\ reference sample \texttt{A} (black), as well as of the \textit{ex situ} (blue) and \textit{in situ} (red) samples.
The peak at around \qty{39.2}{\degree} of the refefence sample is the (00.6) reflection of \cro.
This reflection is also attributed to the peaks at approx.\ \qty{39.4}{\degree} and \qty{39.6}{\degree} of the \textit{ex situ} and \textit{in situ} samples, respectively.
The variation of peak position for the \cro\ layer may originate in the fact that for each process, the thickness may have varied which was determined to be a crucial factor for the out-of-plane strain.
Despite this peak, two overlaying peaks occur at \qty{40.14}{\degree} for both buffer layer processes.
This is attributed to the (00.6) reflection of \agao, because the predicted peak position is at \qty{40.26}{\degree} with a relative intensity\footnote{
    This value indicates the peak intensity of a reflection in powder diffraction patterns and gives a hint to identify peaks.
} of 3.37, and is therefore allowed.
The presence of two peaks is attributed to the splitting between \ce{Cu}-K\textalpha\textsubscript{1} and \ce{Cu}-K\textalpha\textsubscript{2} radiation (\textit{K\textalpha\ splitting}).
Note that at approx.\ \qty{86}{\degree}, the higher order (00.12) reflection of \cro\ can be observed more dominantly than the (00.12) reflection of \agao.
This is due to the fact the the ratio of relative intensities of (00.12) to (00.6) is \qty{29.6}{\percent} for \cro\ and only \qty{1.4}{\percent} for \agao. 

For the \textit{in situ} sample, no additional peaks are observed, indicating phase-pure deposition of \textit{c}-plane \gao\ in the \textalpha-phase on \textit{c}-plane \alo.
For the \textit{ex situ} sample, however, a peak occurs at \qty{38.71}{\degree} and the \textbeta-phase of \gao\ has three different predicted peaks at this position, listed in Tab.\,\ref{Tab:Results_4_beta_c}.
The observed peak is attributed to the $(40\overline{2})$ reflection, because at \qty{19.06}{\degree}, another peak is observed that can only correspond to the $(20\overline{1})$ reflection of \textbeta-\gao (Tab.\,\ref{Tab:Results_4_beta_c}), which favors the identification of $(40\overline{2})$ as a higher order reflection.
Note that the peak observed at \qty{20.4}{\degree} is also observed for the \cro\ reference sample and can therefore not correspond to any \gao\ phase.
\begin{table}
    \centering
    \caption{
        Selected reflections of \textbeta-\gao\ and their predicted positions in \thetaomega\ patterns as well as their relative intensities.
        Data taken from Ref.\,\cite[mp-886]{MaterialsProject}
    }
    \begin{tabular}{llS}
        \toprule
        {$2\theta$} & reflection & {relative intensity} \\
        \midrule 
        \qty{18.89}{\degree}    & $(20\overline{1})$    & 8.7\\
        \midrule
        \qty{38.3}{\degree}     & $(31\overline{1})$    & 57.0 \\
        \qty{38.32}{\degree}    & $(40\overline{2})$    & 3.6 \\
        \qty{38.36}{\degree}    & (202)                 & 2.5 \\
        \bottomrule
    \end{tabular}
    \label{Tab:Results_4_beta_c}
\end{table}

In Fig.\,\ref{Fig:Results_4_buffer_c}b, the \glspl{RSM} of the \textit{ex situ} sample are displayed.
Note that the image is cropped such that no substrate peak is visible.
The previous result is confirmed that another phase has formed on top of the \cro\ layer.
No \textomega-scans were done, but the crystallinity can be estimated by the broadening in $q_\parallel$ direction, which is less dominant in comparison to the \cro\ layer.
Furthermore, the K\textalpha\ splitting indicates a highly crystalline thin film. 

%! r-plane
\begin{figure}
    \centering
    \begin{tabular}{c}
        \multicolumn{1}{l}{\textbf{(a)}}
        \figSpace \\
        \includegraphics{4_thetaomega_r.pdf}
        \figSpace \\
        \multicolumn{1}{l}{\textbf{(b)}}
        \figSpace \\
        \includegraphics{4_RSM_r.png}
    \end{tabular}
    \caption{
        \textbf{(a)}
        \thetaomega-patterns of the \textit{r}-plane \cro\ reference sample \texttt{A} (black), as well as the \textit{ex situ} (blue) and \textit{in situ} (red) buffer layer structures.
        \textbf{(b)}~\gls{RSM} around the (02.4) reflection recorded before (left) and after (right) \textit{ex situ} deposition of \gao\ on \cro.
    }
    \label{Fig:Results_4_buffer_r}
\end{figure}
The \thetaomega\ patterns of the \textit{r}-plane reference and buffer layer samples are depicted in Fig.\,\ref{Fig:Results_4_buffer_r}a.
Due to very close peak positions of the (02.4) reflection for both \cro\ and \agao, a comparison with the reference sample is not as straightforward as for the other orientations.
No additional peak can be identified for the buffer layer samples.
However, the \gls{RHEED} patterns indicated a crystalline surface, which is why the only possible phase of the \gao\ layer is the \textalpha-phase.
This is verified by \glspl{RSM} of the \textit{ex situ} sample in Fig.\,\ref{Fig:Results_4_buffer_r}b:
a significant increment in intensity can be observed at the expected peak position of (02.4) \agao, which is due to the formation of \textalpha-phase \gao\ on \textit{r}-plane \cro.


%! m-plane
\begin{figure}
    \centering
    \begin{tabular}{c}
        \multicolumn{1}{l}{\textbf{(a)}}
        \figSpace \\
        \includegraphics{4_thetaomega_m.pdf}
        \figSpace \\
        \multicolumn{1}{l}{\textbf{(b)}}
        \figSpace \\
        \includegraphics{4_RSM_m.png}
    \end{tabular}
    \caption{
        \textbf{(a)}
        \thetaomega-patterns of the \textit{m}-plane \cro\ reference sample \texttt{A} (black), as well as the \textit{ex situ} (blue) and \textit{in situ} (red) buffer layer structures.
        \textbf{(b)}~\gls{RSM} around the (30.0) reflection recorded before (left) and after (right) \textit{ex situ} deposition of \gao\ on \cro.
    }
    \label{Fig:Results_4_buffer_m}
\end{figure}
When comparing the peaks of \textit{m}-plane buffer layer samples to the \cro\ reference sample, note that the (30.0) peak is shifted to lower angles (Fig.\,\ref{Fig:Results_4_buffer_m}a).
Furthermore, two peaks can be observed for the buffer layer samples, which may originate in (i) either two peaks for \cro\ and \agao\ each; or (ii) K\textalpha\ splitting of a (30.0) \agao\ reflection on top of the \cro\ layer.
Both explanations are favored by the fact that the expected peak position (red dotted line) lays inbetween both peaks.
The theoretical predictions of the $2\theta$ positions also have similar distance as the two peaks observed (red and green dotted lines), favoring the first explanation.
However, when considering Fig.\,\ref{Fig:Results_4_buffer_m}b, it becomes clear that the origin is a K\textalpha\ splitting, because prior to the deposition of \gao, none of the peaks was present with the observed intensity.
Therefore, the observed peaks must both stem from the \textalpha-phase of \gao.
No other peaks are observed in the \thetaomega\ pattern, therefore a phase pure \textalpha-phase is present.

%! a-plane
A similar behavior as for the \textit{m}-plane samples can be observed in the \thetaomega\ patterns of \textit{a}-plane samples (Fig.\,\ref{Fig:Results_4_buffer_a})a.
The splitted peak is attributed to the (22.0) reflection of an \agao\ layer and the peak on the left shoulder to the (22.0) reflection of \cro, which is shifted to lower angles in comparison to the reference sample.
This behavior is also observed for the (11.0) reflections of both \agao\ and \cro.
This result is confirmed by the \glspl{RSM} (Fig.\,\ref{Fig:Results_4_buffer_a}b), where two peaks appear due to K\textalpha\ splitting after \gao\ deposition.
The low broadening in $q_\parallel$ direction as well as the K\textalpha\ splitting indicate good crystal quality.
\begin{figure}
    \centering
    \begin{tabular}{c}
        \multicolumn{1}{l}{\textbf{(a)}}
        \figSpace \\
        \includegraphics{4_thetaomega_a.pdf}
        \figSpace \\
        \multicolumn{1}{l}{\textbf{(b)}}
        \figSpace \\
        \includegraphics{4_RSM_a.png}
    \end{tabular}
    \caption{
        \textbf{(a)}
        \thetaomega-patterns of the \textit{a}-plane \cro\ reference sample \texttt{A} (black), as well as the \textit{ex situ} (blue) and \textit{in situ} (red) buffer layer structures.
        \textbf{(b)}~\gls{RSM} around the (22.0) reflection recorded before (left) and after (right) \textit{ex situ} deposition of \gao\ on \cro.
    }
    \label{Fig:Results_4_buffer_a}
\end{figure}

