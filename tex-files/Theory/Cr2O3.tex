% ----------------------
% --- chromium oxide ---
% ----------------------
Chromia (\ce{Cr2O3}) is a sesquioxide composed of the transition metal chro\-mi\-um and oxygen.
Among other chromium oxides (e.g. metallic \ce{CrO2}, toxic \ce{CrO3} etc.), it is the thermodynamically most stable phase 
    \cite{lebreau2014,robbert1998,al-kuhaili2007},
making it the abundant chromium oxide on earth
    \cite{mi2018}.
\ce{Cr2O3} occurs mainly in the \textalpha-phase (described below), but a cubic spinel \textgamma-phase with random missing \ce{Cr} point defects has also been reported
    \cite{robbert1998}.
Henceforth, \enquote{\ce{Cr2O3}} will refer to the \textalpha-phase.

% Miscellaneous
As coating material, \ce{Cr2O3} is commonly used due to its high hardness and resistance against corrison \cite{singh2019,al-kuhaili2007}, also explaining its use-case as component of stainless steel to form passive films \cite{lebreau2014}.
\ce{Cr2O3} absorbs electromagnetic waves with wavelengths smaller than \qty{400}{\nm}, making it opaque in the \acrshort{UV}-spectrum \cite{cheng1996,guillen2021}.
It is transparent in the visible spectrum with a transmittance of \qty{50}{\percent} at \qty{800}{\nm} \cite{cheng1996}.

% crystal structure
\ce{Cr2O3} crystallizes in the corundum structure, which has trigonal symmetry (space group $R\bar{3}c$) and belongs to the hexagonal crystal family.
One unit cell contains six formula units, i.e. 12 chromium cations and 18 oxygen anions
    \cite{lebreau2014}.
The oxygen atoms arrange in a hexagonal close-packed manner, where two thirds of the formed octahedrons are filled with \ce{Cr} atoms
    \cite{catti1996}.
The unit cell is spanned by a principal axis, called \textit{c}-axis\footnote{The spins of the \ce{Cr} atoms along this direction are alternating $\uparrow$ and $\downarrow$, making the crystal antiferromagnetic.}, and a hexagonal basal plane with lattice constant \textit{a}.
The numerical values for those lattice parameters differ depending on the publication \cite{mi2018,stepanov2021,finger1980}, and we will use the values in Tab.~\ref{Tab:sesquiLatticeConstants}.
\begin{table}
    \centering
    \begin{tabular}{ccc|l}
        &\textit{a}&\textit{c}&Ref.\\\hline
        \textalpha-\ce{Al2O3}&&&\\
        \textalpha-\ce{Cr2O3}&$\qty{4.96}{\angstrom}$&$\qty{13.59}{\angstrom}$&\citeauthor{mi2018} \citeyear{mi2018}\\
        \textalpha-\ce{Ga2O3}&&
    \end{tabular}
    \caption{Lattice constants of selected corundum structured compounds.}
    \label{Tab:sesquiLatticeConstants}
\end{table}

% growth techniques
Several techniques were applied for depositing chromia thin films, including:
\gls{cvd} \cite{cheng2000,cheng2001a,cheng2001} on silicon and glass,
thermal evaporation on platinum \cite{robbert1998},
electron-beam evaporation on glass \cite{al-kuhaili2007},
\gls{rf} sputtering on sapphire \cite{stepanov2021,polyakov2022a,polyakov2022},
reactive \gls{dc} sputtering on glass \cite{guillen2021}, and
reactive \cite{caricato2010} and non-reactive \cite{singh2019} \gls{pld} on silicon and glass, respectively.

% ----------------------------
% --- electronic structure ---
% ----------------------------
\paragraph{Electronic Structure}
Experimental and theoretical studies reveal that chromia exhibits a band gap of \qtyrange{3.2}{3.4}{\eV}
    \cite{mi2018,robbert1998,lebreau2014}.
% Numerical calculations locate the smallest direct band gap at the $M$-point (Brillouin zone edge in $\langle10.0\rangle$-direction).
This predicts insulating behavior, classified as both Mott-Hubbard type and charge-transfer type, which are models to describe the electronic behavior of compounds containing transition metals with partly filled $3d^n$ orbitals\footnote{$n=5$ for \ce{Cr}}
    \cite{catti1996,mi2018,lebreau2014}:
\gls{dft} calculations show that the \ce{Cr}-$3d$ states are almost solely responsible for electronic states in the conduction band and that they are also present in the valence band
    \cite{mi2018,lebreau2014}.
Thus, $3d\longrightarrow 3d$ band transitions are possible, favoring the Mott-Hubbard model of this compound
    \cite{lebreau2014}.
Furthermore, the \ce{O}-$2p$ states are mainly present in the valence band, at similar energies as the \ce{Cr}-$3d$ states, which leads to hybridization and thus favoring the charge-transfer model
    \cite{lebreau2014}.

% p-type of undoped Cr2O3 due to defects
However, several studies agree on \ce{Cr2O3} being a semiconductor with \textit{p}-type conductivity\footnote{\citeauthor{cheng2001a} (\citeyear{cheng2001a}) actually find \ce{Cr2O3} to be insulating. It is noted that they examined \ce{Cr2O3} as an \qty{2}{\nm} thick oxide surface on \ce{CrO2} films deposited by \gls{cvd}.} at room temperature and atmospheric conditions
    \cite{kofstad1980,cheng1996,caricato2010,lebreau2014,mi2018,singh2019,polyakov2022a}.
Calculating the impact of different crystal point defects on the band structure may give insight into these observations.
Indeed, when considering a missing chromium atom (\enquote{vacancy} $V_{\ce{Cr}}$) the band structure changes in two ways:
The band gap itself is reduced \cite{mi2018}, but not in a way that it would make excitations of valence electrons into the conduction band much more probable than at room temperature.
But additionally, there is a new band introduced slightly above the Fermi level (cf. Fig.~\ref{Fig:lebreau2014_acceptorLevel}), which acts as an unoccupied acceptor level
    \cite{mi2018}.
This defect state is mainly composed of \ce{O}-$2p$ orbitals of the oxygen anions surrounding the vacancy
    \cite{lebreau2014}.
From a more intuitive point of view, a missing neutral chromium atom effectively removes the \ce{Cr^3+} cation as well as three electrons bound to the adjacent \ce{O^2-} anions, thus creating three holes \cite{lebreau2014} explaining the \textit{p}-type conductivity.
\begin{figure}
    \centering
    \includegraphics[width=.5\textwidth]{lebreau2014_acceptorLevel}
    \caption{Calculated \acrfull{DOS} of chromia, taking a $V_{\ce{Cr}}$ into account.
    The arrow marks the new acceptor level.
    Image taken from \cite{lebreau2014}.}
    \label{Fig:lebreau2014_acceptorLevel}
\end{figure}

Note that there are also other possible defects with different effects:
a chromium Frenkel point defect describes a \ce{Cr} atom leaving it's position and occupying a formerly unoccupied oxygen-octahedron.
This Frenkel defect actually creates a new band right below the Fermi level, acting as an occupied donor level
    \cite{lebreau2014}.
A similar defect state is introduced by oxygen vacancies
    \cite{mi2018}.
But note that the Fermi level is located only slightly above the \gls{vbm} and thus the new occupied donor level is not significantly closer to the \gls{cbm}, which means that electrons still have to overcome the band gap energy to get into conducting states.
This favors the formation of holes via \ce{Cr} vacancies (\ce{O}-$2p$ acceptor states) rather than electrons via \ce{Cr} Frenkel defects and \ce{O} vacancies (\ce{Cr}-$3d$ donor states).