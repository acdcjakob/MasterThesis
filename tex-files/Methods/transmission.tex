For optical characterization, spectral transmission measurements were conducted:
Mo\-no\-chro\-ma\-tic light is used to compare a reference beam with the intensity of radiation transmitting a sample of thickness $t$.
The fraction of both is defined as the transmittance $T(E)$, which depends on the photon energy $E=\hbar\omega$.
In a simple model, neglecting reflection on the sample surface as well as thin film interference
    \cite{manifacier1977},
the absorption coefficient $\alpha$ can be defined via the \textsc{Lambert}-\textsc{Beer}-law:
\begin{equation}
    T\propto e^{-\alpha t}\,,
\end{equation}
which states an exponential decay of intensity with increasing thickness $t$.
The absorption coefficient is based on the transition rate between valence band and conduction band at photon energy $E$.
This rate depends on the energy-dependent coupling matrix element for this transition, as well as the joint electron and hole \gls{DOS}
    \cite{zanatta2019}.
If matrix element and \gls{DOS} are assumed to be constant and parabolic, respectively
    \cite{tauc2005},
then $\alpha=0$ for energies below the band gap, and for direct transitions it follows \cite{zanatta2019}:
\begin{equation}
    \alpha\propto (E-E_g)^{1/2}\,.
    \label{Equ:Tauc_direct}
\end{equation}
For indirect transitions, phonons with energy $\hbar\Omega$ must be considered which leads to
\begin{equation}
    \alpha\propto (E+\hbar\Omega-E_g)^2\,.
    \label{Equ:Tauc_indirect}
\end{equation}
In general, it has to be noted that those equations are affected by several approximations:
    no \textsc{Coulomb} interaction between holes and electrons is considered;
    defect states that could enable absorption below $E_g$ are neglected;
    the parabolic nature of the \gls{DOS} is only true for $k\approx0$ and not for large energies $\hbar\omega\gg E_g$
    \cite{zanatta2019}.
Due to those considerations, the \enquote{band gap} constant occuring in \eqref{Equ:Tauc_direct} and \eqref{Equ:Tauc_indirect} will be referred to as \emph{optical gap} $E_\mathrm{opt}$, or \emph{optical absorption edge}.
Both equations allow extrapoliting an $\alpha^{1/\eta}$ vs.\ $E$ plot to the abscissa to determine the optical gap, with $\eta=\frac{1}{2}$ and $\eta=2$ for direct and indirect transitions, respectively
    \cite{zanatta2019}.
According to \textcite{zanatta2019}, these two methods are often mislabeled as \textsc{Tauc} plots $\alpha^{1/\eta}$ vs.\ $E$ and were applied to determine the band gap of crystalline solids
    \cite{cheng1996,al-kuhaili2007,singh2019,farrell2015}.
But actually, those methods are based on the description of \emph{crystalline} solids, whereas the original \enquote{\textsc{Tauc} plot} was an empirical description of \emph{amorphous} germanium, leading to an $(\alpha E)^{1/2}$ vs.\ $E$ plot
    \cite{tauc2005}.
Therefore, applying the \textsc{Tauc} method to crystals is inappropriate -- it can be only used if localized energy states are present, as in amorphous solids or nanoparticles
    \cite{dolgonos2016}.
In this work, a direct band gap of \cro\ is assumed and therefore the $\alpha^2$ vs.\ $E$ plot is utilized.
This is described in more detail in chapter \ref{Sec:Results_Preliminary}.

Due to the fact that the intersection of 
\begin{equation*}
    \alpha=\mathrm{const.}\cdot(E-E_\mathrm{opt})^{1/2}
\end{equation*}
with the abscissa does not depend on the value of $\mathrm{const.}$, the $\alpha^2$ vs.\ $E$ plots are visualized with arbitrary units on the ordinate.

%! experiment
The measurements were conducted in a spectral range of \qtyrange{200}{2800}{\nm}, realized by two light sources with a \textit{Lambda 40} (Perkin-Elmer).
The spectra are corrected with a corresponding reference substrate.
The measurements were done by M.\ Hahn.