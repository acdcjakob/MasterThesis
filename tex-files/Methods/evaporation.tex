The ohmic contacts for electrical characterization of \ce{Cr2O3} thin films were deposited by means of thermal evaporation, which is a \gls{pvd} method.
This method is well established for contacting \agao\ thin films
    \cite{vogt2023}.
Experiments on the ohmic behavior of evaporated contacts on \cro\ were conducted and linear $U$ vs.\ $I$ curves were obtained for conducting samples (not shown).
The thermal evaporation method utilizes a \emph{boat} made of a material with high melting temperature (tungsten or molybdenum) that is loaded with the target material in form of powder or filament.
A vacuum chamber is used to achieve pressures of around \qty{5e-5}{\milli\bar}.
A high current is driven through the contacted boat, such that resistive heating ensures melting of the target material and subsequent evaporation.
The evaporated material spreads out due to the pressure gradient and condensates on the sample which is mounted to a rotating holder.
A metal mask ensures that only the corners of the sample are contacted.

The contacts are either a stacking of titanium, aluminum, and gold (henceforth referred to as \emph{Ti-Al-Au}) or only titanium and gold (\emph{Ti-Au}).
The thickness of each material layer is around \qty{30}{\nm}, measured with a crystal oscillator during the process.
The currents used for evaporating \ce{Ti}, \ce{Al} and \ce{Au} are \qty{60}{\ampere}, \qty{50}{\ampere} and \qty{45}{\ampere}, respectively.