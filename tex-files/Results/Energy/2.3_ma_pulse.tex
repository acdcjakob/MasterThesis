In Fig.\,\ref{Fig:Results_3_pulseStrain}, the \gls{oop}\ strain for the \textit{m}- and \textit{a}-plane oriented samples of the 1st batch is depicted.
The strain ranges from
    \qty{0.15}{\percent} (\qty{0.3}{\percent})
to
    \qty{0.45}{\percent} (\qty{1.1}{\percent})
for \textit{m}-plane (\textit{a}-plane) samples.
A clear dependence on fluence can be observed, whereby the thickness has no influence at all.
%! m-plane strain
The complete in- and out-of-plane strains for \textit{m}-plane samples are depicted in Fig.\,\ref{Fig:Results_3_pulse_ma_strainTilt}a.
With increasing \gls{oop}\ strain, the \gls{ip}\ strain also increases, which indicates a pseudomorphic growth mode.
Note that the \gls{ip} strain is, in the range of uncertainity, the same along both $x$- and $y$-direction, even though both axes are not equivalent.
As for \textit{r}-plane samples (cf.\ Fig.\,\ref{Fig:Results_3_pulse_r_StrainTilt}), the \gls{oop}\ strain is systematically smaller by 0.15 percentage points, compared to the values obtained from peak positions in \thetaomega-patterns (Fig.\,\ref{Fig:Results_3_pulseStrain}).
\thetaomega-scans probe for symmetric reflections only, which is why shear stresses cannot be corrected by this method.
However, those strain angles are rather small (cf.\ Fig.\,\ref{Fig:Results_3_pulse_ma_strainTilt}a), which is why it is unplausible that this is the origin of the discrepancy.
Moreover, as discussed for the \textit{r}-plane samples, the thin film tilt is suspected to be too small to yield this effect.
\begin{figure}
    \centering
    \begin{tabular}{c}
        \multicolumn{1}{l}{\textbf{(a)}} \figSpace \\
        \includegraphics{3_pulseEnergy_completeStrain_m.eps} \figSpace \\
        \multicolumn{1}{l}{\textbf{(b)}} \figSpace \\
        \includegraphics{3_pulseEnergy_completeStrain_a.eps}
        
    \end{tabular}
    
    \caption{Left: In-plane (along $x$- and $y$-axis) and out-of-plane strain for the \textbf{(a)} \textit{m}-oriented and \textbf{(b)} \textit{a}-oriented samples from the 2nd batch, calculated from the peak positions of the \glspl{RSM} described in~\ref{Sec:Results_3_Experiment}.
    Right: Shear (circles) and tilt (triangles) of thin films determined by asymmetric and symmetric reflections, respectively.
    The values were determined along $x$-axis and $y$-axis.
    This is represented by the purple and blue ordinates in \textbf{(a)}, respectively.
    Note the different scaling of the ordinates, indicating less tilt along the $y$-axis.
    Due to the same scale of tilt along $x$- and $y$-axis for \textbf{(b)}, only one ordinate is used to represent the data.
    }
    \label{Fig:Results_3_pulse_ma_strainTilt}
\end{figure}

%! m-plane tilt/shear
As predicted by \textcite{kneiss2021}, a significant tilt of the thin film is observed along the $x$-direction (purple triangles in Fig.\,\ref{Fig:Results_3_pulse_ma_strainTilt}b), which ranges from \qty{20}{\arcminute} to \qty{40}{\arcminute} and increases with higher fluences.
Furthermore, a small shear of up to \qty{2}{\arcminute} is observed along this axis.
On the contrary, a thin film tilt (blue triangles) below \qty{1}{\arcminute} and a small shear tilt are observed in $y$-direction.
This is also in accordance with the predicted slip systems (cf.~\ref{Sec:Theory_Relaxed}), which should result in no net tilt along the $y$-axis.
However, as it is the case for the \textit{r}-oriented samples, this thin film tilt -- acting as an indicator for relaxation -- increases with higher fluences.
But this objects to the observation of decreasing relaxation with higher fluences due to increasing \gls{ip}\ and \gls{oop}\ strain. 
Again, this draws to the conclusion that a more sopisticated description of partially relaxed layers is in need.

%! a-plane strain
In a qualititive sense, a similar behavior for \textit{a}-plane samples is observed when investigating both in- and out-of-plane strain (Fig.\,\ref{Fig:Results_3_pulse_ma_strainTilt}b).
However, the \gls{ip}\ strain is significantly larger along the $y$-direction for layers with less strain:
for $F=\qty{0.7}{\J\per\cm\squared}$, the thin film is 0.66 percentage points more strained along the $y$-axis.
This discrepancy reduces for thin films that are more stressed in total.
Furthermore, as for the other orientations, the \gls{oop}\ strain is systematically lower when compared to the values obtained from \thetaomega-patterns, namely by 0.15 percentage points.
The reasons for this effect are presumably the same as for the \textit{m}-plane oriented samples.
%! a-plane tilt / shear
The shear and tilt angles for \textit{a}-plane samples are depicted in Fig.\,\ref{Fig:Results_3_pulse_ma_strainTilt}b.
As predicted by \textcite{kneiss2021}, no tilt is observed along both the $x$-axis ($<\qty{1.5}{\arcminute}$) and $y$-axis ($<\qty{0.7}{\arcminute}$).
A small shear angle of around \qty{1.5}{\arcminute} is observed along the $y$-axis.

%! Omega
Finally, the mosaicity for both \textit{m}- and \textit{a}-plane samples is depicted in Fig.\,\ref{Fig:Results_3_pulseOmega}.
For \textit{m}-plane oriented samples, the \textomega-FWHM is approx.\ \qty{40}{\arcminute} and decreases slightly with higher fluences, which is in contrast to the increasing strain.
For \textit{a}-plane samples, a large spread similar to the samples of the 1st batch (cf.\ Fig.\,\ref{Fig:Results_3_lensOmega}) is observed.
But, again, this observation is probably due to the specific shape of \textomega-patterns for the \textit{a}-plane samples.
In Fig.\,\ref{Fig:App_3_pulse_omega}, the diffractograms are depicted and a severe deviation from the \textsc{Voigt} shape can be identified.
Therefore, determination of the \gls{FWHM} is hindered and the large spread of values explained.