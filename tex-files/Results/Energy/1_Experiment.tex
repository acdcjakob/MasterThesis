%! Sample Fabrication
\subsubsection{Sample Fabrication}
For all following depositions, the laser entrance window was cleaned before each process.
A pure \cro\ target was used for deposition of thin films on $5\times\qty{5}{\mm\squared}$ sapphire substrates in the four aforementioned orientations.
The first batch of samples was produced by only varying the pulse number to achieve a series of thin films with varying thickness but constant laser fluence during deposition\footnote{
    The series of thicknesses that was achieved in the prior experiments was correlated to a series of growth rates.
}.
The pulse energy was set to \qty{650}{\milli\joule} and the lens position\footnote{
    Note that the values for the lens position have an arbitrary offset; a value of \qty{0}{\cm} does not correspond to the position where the target surface is in focus.
}
to \qty{-2}{\cm}, the resulting fluence is approx.\ \qty{2}{\joule\per\cm\squared}.
This corresponds to the standard configuration during all previous processes (pink square in Fig.\,\ref{Fig:Methods_fluence}).
This was repeated with fewer pulse number variations for three other lens positions, namely \qtylist{0;1;2}{\cm}, resulting in lower fluences:
In Fig.\,\ref{Fig:Methods_fluence}, the yellow circles represent the probed laser fluences.
This set of samples is referred to as the 1st batch.

To investigate the influence of fluence independent of ablation area, a 2nd batch of samples was fabricated with a fixed lens position (\qty{-1}{\cm}) but varying laser pulse energy:
\qtylist{300;450;650;800}{\milli\joule}.
The pulse number was adjusted to achieve approximately same thicknesses.
The achieved fluences are visualized as red triangles in Fig.\,\ref{Fig:Methods_fluence}.

%! Measurements
\subsubsection{Measurements}
For all samples, \thetaomega-scans as well as \textomega-scans were performed.
The reflections probed by the latter were (00.6), (02.4), (30.0) and (22.0) for \textit{c}-, \textit{r}-, \textit{m}- and \textit{a}-plane, respectively.
For some selected samples of different thickness and fluence from the 1st batch, transmission measurements have been performed.
The thickness of all samples was determined by spectroscopic ellipsometry measurements.
To obtain more information about the relation between in-plane and out-of-plane lattice constants, \glspl{RSM} were performed on selected samples:

\paragraph{\textit{c}-plane}
    For \textit{c}-plane samples, the thickness series of the 1st batch that was fabricated with the largest laser spot size (lowest fluence) was investigated.
    The asymmetric reflection that was used for probing the relaxation process is (02.10), which has an inclination angle of approx.\ \qty{32}{\degree} with respect to the sample surface.
\paragraph{\textit{r}-plane}
    All \textit{r}-plane samples fabricated in the 2nd batch with different laser pulse energies were investigated.
    For each sample, the $x$-axis of the sample -- containing the projection of the \textit{c}-axis -- is found by performing a \textphi-scan on the (03.0) reflection:
    This set of lattice planes has an inclination with respect to the surface, so the position of the peak in the diffraction pattern of the \textphi-scan reveals the $x$-axis.
    In this azimuth, an \gls{RSM} is recorded around the asymmetric (03.0) reflection and the symmetric (02.4) reflection.
    By rotating $\Delta\phi=\qty{90}{\degree}$, the $y$-axis lays in the scattering plane and another \gls{RSM} is performed around the symmetric (02.4) reflection.
    The twofold measurement of the symmetric reflection is necessary to calculate a possible lattice plane tilt for both $x$- and $y$-direction.
    Note that no shear is calculated due to the asymmetric nature of the (03.0) reflection with respect to the \textit{r}-orientation\footnote{
        For \textit{m}- and \textit{a}-plane rhombohedral structures, the crystal is symmetric under the transformation $\phi\rightarrow\phi+\qty{180}{\degree}$, which is not the case for \textit{r}-plane.
    }.
    After performing the various corrections described in \ref{Sec:Methods_RSM}, the tilt angles can be calculated for both azimuths by
    \begin{eqnarray}
        \theta = \arccos\left(
            \frac{q_\perp}{|\mathbf{q}|}
        \right) \cdot\mathrm{sgn}\left(q_\parallel\right)\,,
        \label{Equ:Results_3_tiltAngle}
    \end{eqnarray}
    with $q_\perp$ and $q_\parallel$ being the \gls{oop}\ and \gls{ip}\ components of the scattering vector $\mathbf{q}$, respectively.
    The \gls{ip}\ and \gls{oop}\ strains are determined by comparing the observed scattering vector to the expected scattering vector for the (03.0) reflection:
    \begin{equation}
        \mathbf{q}_\mathrm{(03.0)} = 
        \left|\mathbf{q}_\mathrm{(03.0)}\right|\cdot
        \begin{pmatrix}
            \cos\alpha_{(03.0)|r}\\
            \sin\alpha_{(03.0)|r}
        \end{pmatrix}\,,
    \end{equation}
    with $|\mathbf{q}_{(03.0)}|$ calculated from \eqref{Equ:Methods_qAbs} and \eqref{Equ:Methods_dhkl}.
    $\alpha_{(03.0)|r}$ denotes the angle between the (03.0) reflection and the normal of the \textit{r}-planes; it can be calculated from \eqref{Equ:Methods_angleWRTc}:
    \begin{equation}
        \alpha_{(03.0)|r}
        = \qty{90}{\degree}-\left(
            \alpha_{(03.0)|c}-\alpha_{(01.2)|c}
        \right)
        = \alpha_{(01.2)|c}
        = \qty{57.62}{\degree}\,.
    \end{equation}
\paragraph{\textit{m}-plane}
    Similar to above, all \textit{m}-plane samples from the 2nd batch were investigated.
    The samples were aligned to the $x$-axis by performing a \textphi-scan on the asymmetric (30.6) reflection, and an \gls{RSM} was recorded afterwards.
    By rotating $\Delta\phi=\qty{180}{\degree}$ while maintaining $2\theta$ and $\omega$, the scattering condition for $(30.\overline{6})$ is probed and an \gls{RSM} was recorded.
    The symmetric reflection (30.0) was also measured in this azimuth.
    The tilt angle and shear angle can be calculated according to \eqref{Equ:Results_3_tiltAngle} and \eqref{Equ:Methods_shearAngle}, respectively.
    The lattice constants can be calculated from the components of the scattering vectors:
    \begin{align}
        a_\perp &= \frac{\sqrt{12}}{q_\perp^{(30.\pm6)}} \,,\\
        a_\perp &= \frac{\sqrt{12}}{q_\perp^{(03.0)}}\,,\\
        c &= \frac{6}{q_\parallel^{(30.\pm6)}} \,.
    \end{align}
    $a_\perp$ denotes the $a$ lattice constant in direction of the normal to the sample surface.
    By rotating $\Delta\phi=\qty{90}{\degree}$, the \textit{y}-axis can be probed via asymmetric reflections $(\overline{4}2.0)$ and (22.0), which differ in the azimuth by $\Delta\phi=\qty{180}{\degree}$.
    A second symmetric reflection (30.0) is recorded in this azimuth.
    Similar to the $x$-axis, the tilt and shear angles, as well as the lattice constants can be calculated:
    \begin{align}
        (4\overline{2}.0):&\quad
            a_\perp = \frac{\sqrt{12}}{q_\perp^{(4\overline{2}.0)}}
            \quad,\quad
            a_\parallel = \frac{2}{q_\parallel^{(4\overline{2}.0)}}\,,\\
        (22.0):&\quad
            a_\perp = \frac{\sqrt{12}}{q_\perp^{(22.0)}}
            \quad,\quad
            a_\parallel = \frac{2}{q_\parallel^{(22.0)}}\,,\\
        (30.0):&\quad
            a_\perp = \frac{\sqrt{12}}{q_\perp^{(03.0)}}\,.
    \end{align}
    $a_\parallel$ denotes the $a$ lattice constant parallel to the $y$-axis.
    For detailed calculations of the former equations, see\ \ref{Sec:App_Calc_mPlane}.
    Note that all 6 measured reflections yield a value for $a_\perp$, and 2 measured reflections each yield 2 values for $c$ and $a_\parallel$, respectively.
    Therefore, for each lattice constant, the mean value is evaluated and the error is estimated by the standard deviation (cf.\ Fig.\,\ref{Fig:Results_3_pulse_ma_strainTilt}a).
\paragraph{\textit{a}-plane}
    All \textit{a}-plane samples from the 2nd batch were investigated and the method is similar to the one applied to the \textit{m}-plane samples.
    The azimuth of the \textit{x}-axis is found by performing a \textphi-scan on the (22.6) reflection, which also served for an \gls{RSM}.
    Rotating by $\Delta\phi=\qty{180}{\degree}$ yields the $(22.\overline{6})$ reflection and (22.0) is also measured.
    Similar to above, the sample is rotated by \qty{90}{\degree} to align to the $y$-axis and two more asymmetric reflections are recorded: (30.0) and (03.0).
    A second \gls{RSM} of (22.0) is also performed.
    This yields the following lattice constants for the $x$-axis:
    \begin{align}
        a_\perp &= \frac{4}{q_\perp^{(22.\pm6)}} \,,\\
        a_\perp &= \frac{4}{q_\perp^{(22.0)}}\,,\\
        c &= \frac{6}{q_\parallel^{(22.\pm6)}} \,,
    \end{align}
    and for the $y$-axis:
    \begin{align}
        (30.0):&\quad
            a_\perp = \frac{2}{q_\perp^{(30.0)}}\cdot\frac{3}{2}
            \quad,\quad
            a_\parallel = \frac{2}{\sqrt{3}q_\parallel^{(30.0)}}\cdot\frac{3}{2}\,,\\
        (03.0):&\quad
            a_\perp = \frac{2}{q_\perp^{(03.0)}}\cdot\frac{3}{2}
            \quad,\quad
            a_\parallel = \frac{2}{\sqrt{3}q_\parallel^{(03.0)}}\cdot\frac{3}{2}\,,\\
        (22.0):&\quad
            a_\perp = \frac{4}{q_\perp^{(22.0)}}\,.
    \end{align}
    For detailed calculations and the origin of the factor $\frac{3}{2}$, see\ \ref{Sec:App_Calc_aPlane}.
    Again, lattice constants obtained from several reflections, the mean and standard deviation are calculated (cf.\ Fig.\,\ref{Fig:Results_3_pulse_ma_strainTilt}b).
